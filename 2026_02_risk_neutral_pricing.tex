\documentclass{beamer}
\usetheme{afm}

\title{Arbitrage-Free Pricing Theory}
\subtitle{Let's refresh some useful concept}
\course{Advanced Financial Modelling}
\author{\href{mailto:matteo.sani@unisi.it}{Matteo Sani}}

\begin{document}
	\begin{frame}[plain]
		\maketitle
	\end{frame}        

\section{Risk Measures}
\begin{frame}{Estimating Risk}
	\begin{itemize}
		\item The assignment of the correct price to a derivative is not the whole story.
		\item Usually it is important to assess the \emph{interest rate risk} of a portfolio; in other words check how its value moves with the rates (its \emph{sensitivity}).
		\item There are several metrics to estimate the risk and different ways to measure its actual value:
		\begin{itemize}
			\item \textbf{analytical}: involves deriving a closed-form expression which can be complex (and not always feasible);
			\item \textbf{numerical}: one bumps the yield curve calibration instruments and reprice the derivative. The change in price gives the risk;
			\item \textbf{curve Jacobians}: when yield curves have been calibrated, the solver slope 'Jacobian' can be kept and used to give the change in derivative price for a change in forwards/discount factors;
			\item \textbf{algorithmic differentiation}: is becoming the gold standard. It can be used to compute risk "automatically" along with the price. It produces "exact" and fast risk estimates, without the noise of numerical bumping.
		\end{itemize}
		
	\end{itemize}
\end{frame}


\begin{frame}{PV01}
	\begin{itemize}
		\item<1-> \textcolor{maincolor}{PV01} is computed by changing the value of the fixed coupon by 1~bp and evaluating the impact on the NPV. It is equal to the discounted value of the cash-flows for a rate of 0.01\% for all periods of the fixed leg
		\begin{equation}
			\textbf{PV01} = 0.01\% \frac{\partial \textbf{PFS}}{\partial K} = 0.01\% \sum_{i=\alpha+1}^\beta\tau_iP(t,T_i)
		\end{equation}
		\item<2-> PV01 is essentially the Annuity ($A_{\alpha,\beta}$). It measures how the NPV changes if the fixed coupon $K$ changes. 
		\item<3-> It is used for pricing and sales margins.This is a useful measure for dealers calculating the exact P\&L generated by applying a spread (or margin) to a fixed rate away from the mid-market rate.
	\end{itemize}
\end{frame}

\begin{frame}{DV01}
	\begin{itemize}
		\item<1-> If you want to know the \emph{linear} risk associated to a Interest Rate Swap if the market actually moves, you can make a slightly different calculation than above.
		\item<2-> The \textcolor{maincolor}{Dollar value of a basis point} measures how the NPV changes if the market rates change (an \textbf{up-shift} move of 1~bp in the forward rate curve).
		\item<3-> In the \emph{real} scenario the fixed rate is, well, fixed and floating rates move, so you can consider what happens to the NPV if every rate $r_i$ is changed in parallel by the same amount
		
		Crucial Correction: In your DV01 formula, you used a down-shift. In most US/UK markets, DV01 is a positive number representing the loss for a 1bp up-shift (for a receiver). Ensure you define the sign convention clearly for the exam.
		
		\begin{equation*}
			\textbf{DV01} = \sum_j \frac{\partial \textbf{PFS}}{\partial r_j} = -\sum_{j}\tau_jP_j+\sum_{j}\left(K\sum_{i=\alpha+1}^\beta\tau_i\frac{\partial P_i}{\partial r_j} - \sum_{k=\alpha+1}^\beta L_k\tau_k\frac{\partial P_k}{\partial r_j}\right)
		\end{equation*}
		\item<4-> In the multi-curve framework DV01 would be calculated for the forecast curve, and for the discounting curve, resulting in two actual DV01 measurements.
	\end{itemize}
\end{frame}

\begin{frame}{DV01 Numerical Calculation}
	\begin{itemize}
		\item<1-> Assume the P\&L on a Swap could be approximated by its linear P\&L plus its convexity
		\begin{equation*}
			\Delta \textbf{PFS}(\delta r)\approx \frac{\partial \textbf{PFS} }{\partial r}\delta r + \frac{1}{2}\frac{\partial^2 \textbf{PFS}}{\partial r^2}\delta r^2
		\end{equation*}
		\item<2-> Then bumping $\delta r$ by $\pm 1$~bp and dividing by 2 eliminates the convexity element and very accurately approximates the DV01
		\begin{equation*}
			\textbf{DV01} = \frac{\Delta \textbf{PFS}(+1~bp)-\Delta \textbf{PFS}(-1~bp)}{2}=\frac{\partial \textbf{PFS} }{\partial r}
		\end{equation*}
		\item<3-> Another common method of calculation is to use a single bumped curve by, say, $\frac{1}{100}$th of a bp, and scale the result by 100 (less accurate, since the convexity is marginalised and not eliminated)
		\begin{equation*}
			100\Delta \textbf{PFS}\left(\frac{1}{100}~bp\right)=\frac{\partial \textbf{PFS}}{\partial r}+\frac{1}{200}\frac{\partial^2\textbf{PFS}}{\partial r^2}
		\end{equation*}
	\end{itemize}
\end{frame}


\begin{frame}{Market Rate Sensitivity}
	While DV01 assumes a parallel shift across the entire term structure, \textbf{Market Rate Sensitivity} (often called 'Bucket Risk' or 'Pillar Risk') provides a granular view of risk. 
	This process involves shocking each individual market instrument used in the bootstrap (e.g., specific 3M Futures, 2Y Swaps, 5Y Swaps) by 1 basis point independently. 
	This results in a vector of sensitivities that accounts for the fact that the curve does not always move in parallel; it may twist or butterfly. 
	Because the Swap NPV is a \textbf{convex} function of these underlying rates, the sum of these bucketed risks will closely approximate, but not perfectly match, the parallel DV01.
\end{frame}


\begin{homework}
	\begin{frame}{\textcolor{white}{Homework}}
		\begin{itemize}
			\item[white] Consider a 2-year Interest Rate Swap on a notional of 1M, with a fixed rate of 5\% and paying Libor rate annually. The term-structure of interest rates is flat at 5\%.
			Estimate the DV01 of the swap numerically.  
		\end{itemize}
	\end{frame}
\end{homework}










%When moving from linear derivatives (Swaps) to options on those derivatives (Swaptions), calculating the Delta becomes significantly more complex. In a Swaption, the Delta measures the sensitivity of the option price to a change in the underlying swap rate.
%\begin{frame}{The Delta of a Swaption}
%	
%	\begin{itemize}
%		\item Unlike a Swap, where Delta is constant (linear), a Swaption's Delta is non-linear and depends on the "moneyness" of the option.
%		\item The value of a European Swaption is typically given by the \textbf{Black-76 formula}:
%		\begin{equation*}
%			V_{pay} = A_{\alpha,\beta}(t) \left[ S_{\alpha,\beta}(t) \Phi(d_1) - K \Phi(d_2) \right]
%		\end{equation*}
%		\item The Delta is the derivative of this price with respect to the forward swap rate S:
%		\begin{equation*}
%			\Delta = \frac{\partial V}{\partial S} = A_{\alpha,\beta}(t) \Phi(d_1)
%		\end{equation*}
%		\item \textbf{The Complication:} In the market, we often distinguish between:
%		\begin{itemize}
%			\item \textbf{Sticky-Strike Delta}: Assumes volatility σ stays constant as the rate moves.
%			\item \textbf{Sticky-Delta (Shadow Delta)}: Accounts for the \emph{Volatility Smile}. As the rate moves, the implied volatility usually changes (skew), requiring an adjustment: ∂S∂V​+∂σ∂V​∂S∂σ​.
%		\end{itemize}
%	\end{itemize}
%\end{frame}


\section{Cap and Floor}
\begin{frame}{Caps and Floors}
	\begin{itemize}
		\item<1-> A \textcolor{maincolor}{Cap} is a Payer IRS in which the payment is done only if the payoff is positive. Its value is the expectation of 
		\begin{equation}
			\sum_{i=\alpha+1}^{\beta}D(t,T_i)N\tau_i\max\left[L(T_{i-1},T_i)-K,0\right]
			\label{eq:cap}
		\end{equation} 
		\item<2-> The cap allows investors which have a debt at a variable rate to buy insurance against high rates in the future.
		\item<3-> A \textcolor{red}{Floor} is the same kind of object but analogous to a Receiver IRS:
		\begin{equation}
			\sum_{i=\alpha+1}^{\beta}D(t,T_i)N\tau_i\max\left[K-L(T_{i-1},T_i),0\right]
			\label{eq:floor}
		\end{equation} 
	\end{itemize}
\end{frame}

\subsection{Caplets}
\begin{frame}{Caplet and Floorlet}
	\begin{itemize}
		\item<1-> Considering each element of the sum in \cref{eq:cap} or \cref{eq:floor} we see that Cap/Floor can be split into forward starting options over a floating rate, called \textcolor{red}{Caplet/Floorlet}.
		\item<2-> A Caplet/Floorlet payoff is defined as
		\begin{equation*}
			D(t,T_i)N\tau_i\max\left[L(T_{i-1},T_i)-K,0\right]
		\end{equation*} 
		and its value is given by
		\begin{equation}
			\textbf{Cpl}(t,T_{i-1},T_i,\tau,N,K)=\expect{Q}\left[e^{-\int_t^{T_i}r_s ds}N\tau(L(T_{i-1},T_i)-K)^+ | \mathcal{F}_t\right]
		\end{equation}
		\item<3-> This can also be written
		\begin{equation*}
			\textbf{Cpl}=N\expect{Q}\left[e^{-\int_t^{T_{i-1}}r_s ds}\tau P(T_{i-1},T_i)(L(T_{i-1},T_i)-K)^+ | \mathcal{F}_t\right]
		\end{equation*}
	\end{itemize}
\end{frame}

\begin{frame}{Caplets as ZCB Options}
	\begin{itemize}
		\item<1-> Using the LIBOR rate definition we get
		\begin{equation*}
			\begin{aligned}
				\textbf{Cpl} &=N\expect{Q}\left[e^{-\int_t^{T_{i-1}}r_s ds}P(T_{i-1},T_i)\left(\frac{1}{P(T_{i-1},T_i)}-1-K\tau\right)^+ \Big\rvert \mathcal{F}_t\right] \\
				& = N\expect{Q}\left[e^{-\int_t^{T_{i-1}}r_s ds}\left(1-(1+K\tau)P(T_{i-1},T_i)\right)^+ | \mathcal{F}_t\right]
			\end{aligned}
		\end{equation*}
		\item<2-> Multiplying by $\frac{1}{1+K\tau}$ we finally get
		\begin{equation}
			\boxed{\textbf{Cpl}=N(1+K\tau)\expect{Q}\left[e^{-\int_t^{T_{i-1}}r_s ds}\left(\frac{1}{1+K\tau}-P(T_{i-1},T_i)\right)^+ \Big\rvert \mathcal{F}_t\right]}
		\end{equation}
	\end{itemize}
\end{frame}

\begin{frame}{Caplets as ZCB Options}
	\begin{block}{Intepretation}
		\begin{enumerate}
			\item Caplets can be seen as \textcolor{red}{put options} on ZCBs
			\begin{equation*}
				\textbf{Cpl}=N'\expect{Q}\left[D(t,T_{i-1})\left(K'-P(T_{i-1},T_i)\right)^+ \Big\rvert \mathcal{F}_t\right]
			\end{equation*}
			\item Similarly floorlets are \textcolor{red}{call options} on ZCBs
			\begin{equation}
				\textbf{Flr}=N'\expect{Q}\left[D(t, T_{i-1})\left(P(T_{i-1},T_i)-K'\right)^+ \Big\rvert \mathcal{F}_t\right]
			\end{equation}
		\end{enumerate}
	\end{block}
 It highlights that a Caplet is not just a "rate option" but a Put option on the Bond price. If the bond price drops (rates rise), the Put gains value. This is a very powerful way to link interest rate volatility to bond volatility.
\end{frame}

\section{The Black Model}
\begin{frame}{The Black Model - Overview}
	\begin{itemize}
		\item The \textcolor{maincolor}{Black Model} extends the Black-Scholes formula to \emph{caplets} and \emph{swaptions}. % and \emph{bond options}. %It uses the forward coordinates, not the spot ones; this last is not a minor issue indeed.
		\item The main difference with respect to the Black-Scholes set up is that \textcolor{maincolor}{forward rates $F(t;T_{i-1},T_i)$ (or swap rates $S_{\alpha\beta}(t)$) are log-normally distributed}, rather than the spot price of the underlying. 
		\item It should be stated that Black’s formulas did not originally correspond to prices that arise from the application of martingale pricing theory to some particular model. 
		\item The advent of the \emph{market models} will provide a belated justification for these formulas but we shall see that the justifications are mutually inconsistent. 
		\item Indeed, it may be shown that if forward rates have deterministic volatilities then it is not possible for swap rates to also have deterministic volatilities. Therefore Black’s formulas for caplets and swaptions cannot both hold within the same model. 
	\end{itemize}
\end{frame}

\begin{frame}{Pricing Caps with the Black-76 Formula}
	\begin{block}{Definition}
		\begin{equation}
			\begin{aligned}
				\textbf{Cap}_{Bl}(0, \tau,N,K,\sigma_{\alpha,\beta}) &= N\sum_{i=\alpha+1}^{\beta}\textbf{Cpl}_{Bl}(T_i, \tau,K,\sigma_{\alpha,\beta}) = \\ &=N\sum_{i=\alpha+1}^{\beta}\tau P(0,T_i) \textbf{Bl}(K,F(0;T_{i-1},T_i),v_i)
			\end{aligned}
			\label{eq:cap_black}
		\end{equation}
		where
		\begin{equation*}
			\begin{gathered}
				\boxed{\textbf{Bl}(K,F_i,v_i)=F\Phi(d_1(K,F_i,v_i)) - K\Phi(d_2(K,F_i,v_i))} \\
				d_{1,2} = \frac{\log{\cfrac{F_i}{K}} \pm \cfrac{v_i^2}{2}}{2} \\[0.2cm]
				v_i = \sigma_{\alpha,\beta}\sqrt{T_{i-1}}
			\end{gathered}
		\end{equation*}
	\end{block}
\end{frame}

\begin{frame}{Problems with the Black Model}
	\begin{itemize}
		\item It is widely used in practice as the metric by which traders translated volatilities into prices until rates became too low and the model collapsed under the assumption of positive rates.
		\item In the Black model \textcolor{red}{negative rates are not allowed}.
		\begin{equation*}
			d_{1,2} = \frac{\boxed{\log{\frac{F}{K}}} \pm \cfrac{v^2}{2}}{2} 
		\end{equation*}
%		Peer Note: In your formula for d1,2​, you have a stray 2 in the denominator: d1,2​=vlog(F/K)±v2/2​. Ensure you check if that extra 2 in your snippet was a typo or a specific notation. Usually, it's just v (the total volatility σT​).
		
		
		but in the last years in the inter-bank market it was not so unusual to find prices for -1\% strike floors.
		\item Moreover in the Black model the empirical evidence of the "smile" (volatility vs $K$) is not accounted for, i.e. $\sigma$ is a constant. Two caps identical but for the strike need a different volatility to recover two different market prices if one uses Black formula.
		
%		Slide: The Black Model Consistency
%		You make a crucial point: Log-normality cannot hold simultaneously for both Caplets and Swaptions. Since a Swap rate is a weighted average of Forward rates, and the sum of log-normals is not log-normal, the market effectively uses "inconsistent" models for different desks. This is a great "A-grade" exam question topic.
		
	\end{itemize}
\end{frame}

\begin{frame}{The Practitioner Solutions}
\begin{itemize}
	\item \textbf{"Smile" issue}: the model is used with \textcolor{maincolor}{different input volatilities for different strikes}. In practice it is a mapping of implied volatilities into prices and vice versa.
	\item \textbf{Non-positive rates}: could have switched to a different dynamics, like ABM, where negative values are "normal". But there are large differences in the tails between normal and log-normal distributions.
	So Black model has been \textcolor{maincolor}{shifted}. The technique was already known but in the last years has become crucial to shift the lower bound of prices admitted by the model.
\end{itemize}
\end{frame}

\begin{frame}{Shifted Log-normal Model for Caplets}
\begin{itemize}
	\item Slide: Shifted Log-normal (Displaced Diffusion)
%	The formula Bl(K+α,F+α,vT​) is the standard market fix. By shifting the "zero" floor to −α, you allow F to go negative while the argument inside the log stays positive.
	
	\item It can be shown that Black formula provides valid solutions if strike and forward rate are \textcolor{maincolor}{shifted}. For a $(T,S)$ caplet with strike $K$ we get
	\begin{equation}
		\textbf{Cpl}_{Bl}(t,T,S,\tau,K,v_T,\alpha) = P(t,S) Bl(K+\alpha,F(t;T,S)+\alpha,v_T)
	\end{equation}
	where $d_1$ and $d_2$ read as before and instead $v_i$ is now given by
	\begin{equation*}
		v_i = \sigma^{\text{shifted}}\sqrt{T_{i-1}}
	\end{equation*}
	\item The market quotes of $\sigma^{\text{shifted}}$ refer to shifts $\alpha$ of the order of [2\%,3\%].
\end{itemize}
\end{frame}

\begin{homework}
	\begin{frame}{\textcolor{white}{Homework}}
		\begin{itemize}
			\item[white] Prove that the difference between the price of a Cap and a Floor with the same strike K and the same maturity structure is equal to the value of a Payer Interest Rate Swap (IRS).
		\end{itemize}
	\end{frame}
\end{homework}

%\begin{frame}{The Volatility Hump}
%\begin{itemize}
%	\item Empirical studies have pointed out two very important facts:
%	\begin{itemize}
%		\item the first one is that interest rates volatility can depend on the level of the interest rates themselves;
%		\item moreover the volatility function is increasing in the short end of the curve, and decreasing in the long end, with an \textcolor{red}{humped} type movement.
%	\end{itemize}  
%	\begin{columns}
%		\column{0.45\linewidth}
%		\item Uncertainty is bigger in the intermediate region and lower in the front of the maturity spectrum. For long maturities volatility tends to decay.
%		\item When the hump does not appear it is regarded as \emph{stressed market}.
%		%\item There is a financial explanation for this feature.
%		\column{0.45\linewidth}
%		\includegraphics[width=1.1\linewidth]{cap_vola}
%	\end{columns}
%\end{itemize}
%\end{frame}














\section{Swaptions}
\begin{frame}{Swaptions Overview}
\begin{itemize}
	\item<1-> The purchaser of an \textcolor{red}{European swaption} has the right but not the obligation to enter into a swap contract, at a given future time, called the \textcolor{red}{swaption maturity}.
	\item<2-> There are two types of \textcolor{red}{swaptions} (as the underlying swaps): a \emph{payer} in which at maturity the buyer could become the fixed-rate payer and a \emph{receiver} where she could become the fixed-rate receiver.
	\item <3-> The strike price of the swaption determines the fixed rate of the underlying swap.
	\item<4-> A swaption provides protection for a borrower as it ensures a maximum fixed interest rate payable in the future. Furthermore, it gives her the flexibility, if the rate does not rise above the swaption strike rate at expiry, to choose not to exercise it and take advantage of the lower market rates.
\end{itemize}
\end{frame}

\begin{frame}{Strategic View and Risks}
\begin{itemize}
	\item<1-> Usually, the swaption maturity coincides with the first reset date of the underlying IRS (the Interest Rate Swap length is called the \textcolor{red}{tenor} of the swaption).	
	\item<2-> If you are on the buyer side (you are long payer swaption) which is your view on rates ? Why ?
	\item<2->  The buyer of a Payer Swaption is Bearish on bond prices (Bullish on rates). They want rates to rise above $K$.
	\item<3-> Are there any risks associated with a Swaption?
	\begin{itemize}
		\item<4-> The primary risk with a Swaption occurs after you have exercised your right and proceeded with the Swap. Should interest rate movements be different to your expectations, the Swap may have the opposite effect to what you were trying to achieve with the transaction. 
		\item<5-> If interest rates do not rise above the strike on the exercise date, you have not obtained any benefit from the premium paid for the purchase of the Swaption. The premium is the cost of obtaining protection against a rise in interest rates.
	\end{itemize}
\end{itemize}
\end{frame}

\begin{frame}{Swaption Payoff: The Summation Problem}
\begin{itemize}
	\item<1-> The discounted payoff of a payer Swaption (with maturity $T_\alpha$) is given, recalling the value of a payer IRS (\cref{eq:swap_as_sum_fra}) by
	\begin{equation}
		\textbf{PSw}=D(t,T_\alpha)\left(\sum_{i=\alpha+1}^\beta P(T_\alpha,T_i)\tau_i (F(T_\alpha;T_{i-1},T_i) - K)\right)^+
		\label{eq:swaption_payoff_std}
	\end{equation}
	\item<2-> Unfortunately this payoff \textcolor{red}{cannot} be easily decomposed into elementary parts (as done for Cap/Floor). Indeed the \emph{positive part operator} $(\cdot)^+$ is "outside" the summation (while for Caps it is "inside").
	\item<3-> Nevertheless it can be simplified by writing it in a different way...
	%		\begin{equation}
		%			ND(t,T_\alpha)\left(S_{\alpha,\beta}(T_\alpha)-K\right)^+\sum_{i=\alpha+1}^\beta \tau_i P(t,T_i)
		%		\end{equation}
\end{itemize}
\end{frame}

\begin{frame}{Swaption Payoff: The Annuity Measure}
	\begin{itemize}
		\item<1-> Recall that the Par Swap Rate $S_{\alpha,\beta}(t)$ is the rate that sets the NPV of a swap to zero at time $t$. We can express the Payer Swap (PFS) value as:
		\begin{equation*}
			\textbf{PFS}(t) = A(t) \left( S_{\alpha,\beta}(t) - K \right)
		\end{equation*}
		where $A(t)$ is the \textcolor{maincolor}{Annuity} (or Level) factor:
		\begin{equation*}
			\boxed{A(t) = \sum_{i=\alpha+1}^\beta \tau_i P(t,T_i)}
		\end{equation*}
		\item<2-> By modeling the swap rate $S_{\alpha,\beta}(t)$ as the stochastic variable, the Swaption price at time $t$ is the expectation of:
		\begin{equation}
			\textbf{PSw} = \mathbb{E}^{\mathbb{Q}} \left[ D(t, T_\alpha) A(T_\alpha) \max(S_{\alpha,\beta}(T_\alpha) - K, 0) \right]
		\end{equation}
		\item<3-> This formulation is intuitive: it is an option on the swap rate, scaled by the value of the fixed-leg cash flows (the annuity).
	\end{itemize}
\end{frame}

%\begin{frame}{Swaption Payoff}
%\begin{itemize}
%	\item<1-> Recall that we have expressed the swap payoff also as (\cref{eq:swap_payoff_with_swap_rate})
%	\begin{equation*}
%		\textbf{PFS}=\sum_{i=\alpha+1}^\beta \tau_i P(t,T_i)(S_{\alpha,\beta}-K) = A(S_{\alpha,\beta}-K)
%	\end{equation*}
%	\item<2-> If we look at the swaption payoff through this expression modeling as stochastic variable directly $S_{\alpha,\beta}(t)$, instead of each forward rate $F(t;T_{i-1},T_i)$, we can write the swaption price as the expectation of
%	\begin{equation}
%		\boxed{\textbf{PSw}=\expect{Q}\left[D(t,T_\alpha)A\max(S_{\alpha,\beta}(T_\alpha)-K, 0)\right]}
%	\end{equation}
%	which looks like easier and more intuitive than previous \cref{eq:swaption_payoff_std}.
%\end{itemize}
%\end{frame}

\begin{frame}{Swaption Characterization}
\begin{itemize}	
	\item We characterize the payoff in three different ways
	\begin{enumerate}
		\item The swaption is said to be \textcolor{maincolor}{at-the-money} (ATM) if
		\begin{equation*}
			K = K_{ATM} = S_{\alpha,\beta}(0) = \frac{P(0,T_\alpha)-P(0,T_\beta)}{\sum_{i=\alpha+1}^\beta \tau_i P(0,T_i)}
		\end{equation*}
		where $T_\alpha$ is the maturity of the swaption, and $T_\beta$ the last payment date of the underlying swap (the first being $T_{\alpha+1})$. That is when the strike is equal to the swap forward rate $S_{\alpha,\beta}$.
		\item The payer swaption is \textcolor{red}{in-the-money} if $K<K_{ATM}$ and \textcolor{red}{out-of-the-money} otherwise.
		\item The opposite holds for the receiver swaption.
	\end{enumerate}
	\item ATM swaptions are quoted for maturities ranging between $1m$ and $30y$, and for tenors between $1y$ and $30y$.
\end{itemize}
\end{frame}

\begin{frame}{Swaption as an Option on a Swap}
	\begin{itemize}
		\item So the forward rates are the chosen state variable, also the correlation between them is needed...
		\item Market practice: approximation formula (see chapter 6 of Brigo-Mercurio) the definitive reference for this issue.
		\item Clearly here a model which accounts for terminal correlations needed.
		\item Which is the relationship between a Cap and a payer swaption with the same payment and roll dates ?
	\end{itemize}

\end{frame}

\begin{frame}{An Example}
\begin{itemize}
	\item \textbf{A} has raised a $10y$ loan with floating interest rates fixed every three months (IBOR + margin).
	\item \textbf{A} wants to \emph{hedge the loan against rising interest rates but also to benefit from the floating rate}, i.e. should interest rates not rise above a certain level (the swaptions strike-rate $K$).
	\item The purchase of a payer swaption could hedge this risk: 
	\begin{columns}
		\column{0.45\linewidth}
		\begin{itemize}
			\item \textbf{interest rates increase}: \textbf{A} may exercise the swaption and be a party of a swap as a payer of a fixed interest rate;
			\item \textbf{swap-rate below $K$}: it will not be exercised and \textbf{A} will continue to have floating-rate funding.
		\end{itemize}
		\column{0.45\linewidth}
		\includegraphics[width=1.\linewidth]{images/swaption_example}
	\end{columns}
\end{itemize}
\end{frame}

\begin{homework}
\begin{frame}{\textcolor{white}{Homework}}
	\begin{itemize}
		\item[white] In order to hedge your position against interest rate movements which kind of contract would you use: 
		\begin{itemize}
			\item[white] a Swap
			\item[white] a Cap/Floor
			\item[white] a Swaption
		\end{itemize}
		List the pros and cons about each one and declare your favourite.
	\end{itemize}
\end{frame}
\end{homework}

\begin{frame}{Black Formula for Swaptions}
	\begin{itemize}
		\item The market standard for pricing European Swaptions is the \textbf{Black-76 formula}, assuming $S_{\alpha,\beta}$ follows a log-normal process.
		\item The price of a Payer Swaption is given by:
		\begin{equation}
			\boxed{\textbf{PSw}_{Bl} = N \cdot A(0) \left[ S_{\alpha,\beta}(0)\Phi(d_1) - K\Phi(d_2) \right]}
		\end{equation}
		where:
		\begin{equation*}
			\begin{gathered}
				d_{1} = \cfrac{\log\left(\frac{S_{\alpha,\beta}(0)}{K}\right) + \frac{v^2}{2}}{v}, \quad d_2 = d_1 - v \\[0.2cm] 
				v = \sigma_{\alpha,\beta}\sqrt{T_\alpha}
			\end{gathered}
		\end{equation*}
		\item Note: $A(0)$ is the current value of the annuity, and $\sigma_{\alpha,\beta}$ is the \emph{implied volatility} for the specific maturity $T_\alpha$ and tenor $T_\beta - T_\alpha$.
	\end{itemize}
\end{frame}

\begin{frame}{Swaptions Volatility Calibration}
\begin{itemize}
	\item Swaption volatilities are quoted for different maturities and tenors (length of the underlying swap).
	\item Both for ATM and away from ATM on both sides ("swaption smile").
	\item So swaptions have an additional dimension with respect to caps: the quotes are parametrized according to 
	\begin{itemize}
		\item maturites;
		\item tenors;
		\item strikes.
	\end{itemize}
	%\item They have also a different \emph{delta} effect on your book.
	%		\item Volatility trade between caps ans swaption: WEDGE
\end{itemize}
\end{frame}

\begin{frame}{Swaption Volatility Calibration}
\begin{center}
	\includegraphics[width=1.\linewidth]{images/atm_vol}
\end{center}
\end{frame}

\begin{frame}{Swaption Volatility Calibration}
\begin{center}
	\includegraphics[width=1.\linewidth]{images/skews}
\end{center}
\end{frame}

\begin{frame}{Swaption Volatility Calibration}
After we have constructed the volatility matrix we can fit the "smile" at each (expiry, tenor) pair.

This is done assuming the volatilities evolves according to the SABR model. An approximated solution has been given by \emph{Hagan et al.} which is the industry standard for fitting the "Volatility Smile".

The parameters in the Hagan formula control the shape of the curve:
\begin{itemize}
\item $\alpha$: determines the overall level of volatility (the "At-The-Money" level);
\item $\beta$:determines the backbone of the smile (usually fixed at 0.5 or 1 to represent log-normal or normal-like behavior);
\item $\rho$: controls the Skew (slope). A negative $\rho$ creates a downward slope (volatility higher for lower strikes);
\item $\nu$: controls the Convexity (the "Smile" or curvature).
\end{itemize}
\begin{eqnarray} \sigma _{B}(K,f) &=&\frac{\textcolor{red}{\alpha} \left\{ 1+\left[ \frac{\left( 1-\textcolor{red}{\beta} \right) ^{2}}{24}\frac{\textcolor{red}{\alpha} ^{2}}{(fK)^{1-\textcolor{red}{\beta}}}+\frac{1}{4}\frac{\textcolor{red}{\rho \beta \nu\alpha}}{(fK)^{(1-\textcolor{red}{\beta})/2}}+\frac{2-3\textcolor{red}{\rho} ^{2}}{24}\textcolor{red}{\nu}^{2}\right] T\right\} }{(fK)^{(1-\textcolor{red}{\beta})/2}\left[ 1+\frac{(1-\textcolor{red}{\beta})^{2}}{24}\ln ^{2} \frac{f}{K}+\frac{(1-\textcolor{red}{\beta})^{4}}{1920}\ln^{4}\frac{f}{K}\right] } \times \frac{z}{\chi (z)} \notag \\ z &=&\frac{\textcolor{red}{\nu}}{\textcolor{red}{\alpha}}(fK)^{(1-\textcolor{red}{\beta})/2}\ln \frac{f}{K} \notag \\ \chi (z) &=&\ln \left[ \frac{\sqrt{1-2\rho z+z^{2}}+z-\textcolor{red}{\rho}}{1-\textcolor{red}{\rho}}\right] . \notag \end{eqnarray}
\end{frame}

\begin{frame}{Swaption Volatility Calibration}
\begin{center}
	\includegraphics[width=0.65\linewidth]{images/10y_10y}
\end{center}
\end{frame}

\begin{frame}{Swaption Volatility Calibration}
\begin{center}
	\includegraphics[width=0.3\linewidth]{images/alpha}
	\includegraphics[width=0.3\linewidth]{images/rho}
	\includegraphics[width=0.3\linewidth]{images/nu}
\end{center}
\end{frame}

%\begin{frame}{Differences between Caps and Swaptions}
%\begin{itemize}
%	\item<1-> Caps can be decomposed into more elementary products: \textcolor{red}{caplets}. Value each caplet one by one and then add their prices
%	\begin{itemize}
%		\item each forward rate can be modeled as a random variable;
%		\item \textbf{no joint action of forward LIBOR rates is involved.}
%	\end{itemize}
%	\item<2-> Unfortunately this is not possible with swaptions. The swap rate is \textcolor{red}{essentially a weighted average of forward rates} $S=\frac{\sum F(t;T_{i-1},T_i)}{\sum P(t,T_i)}$, hence its volatility should depend on each forward rate volatilities \textbf{as well as} their correlations.
%	\item<3-> If you take as "fundamental" entity the LIBOR rates you have to deal with the joint action of the simple forward LIBOR rates and so with the \textbf{terminal correlation} between rates of different portions of the yield curve. 
%	\item<4-> This issue will be extensively studied in the context of the Libor Market Models further on down the course.
%	%Can you provide an example ?
%\end{itemize}
%\end{frame}

\subsection{Bond Options}
\begin{frame}{An Option to Exchange Fixed with Float}
\begin{itemize}
	\item<1-> We have seen that a Swap can be viewed as an exchange of bonds (fixed for float).
	\item<2-> Hence a Swaption can be regarded as an \textcolor{red}{option to exchange fixed for floating} bonds (or vice versa).
	\item<3-> In case it would be possible to get a simple expression for a \textcolor{red}{Coupon Bond Option} we could use to price Swaptions.
	\item <4-> It would be even simpler if we could express a Coupon Bond Option as a portfolio of Zero Coupon Bond Options.
	\item<5-> Luckily we can do that thanks to a recipe known as \textcolor{red}{Jamshidian's decomposition}.
\end{itemize}
\end{frame}

\begin{frame}{Jamshidian's Decomposition}
\begin{block}{Theorem}
	Consider a sequence of functions $f_i$, a random variable $W$ and a constant $K\ge0$. If each $f_i$ is monotone (decreasing), that is $\cfrac{\partial f_i}{\partial W} < 0;\;\forall i$, then 
	\begin{equation*}
		\left(K - \sum_i f_i(W)\right)^+ = 	\sum_i \left(K_i - f_i(W)\right)^+
	\end{equation*} 
	In financial terms it means that the payoff of an option on a portfolio of assets can be expressed in terms of a portfolio of options on each asset.
\end{block}
\end{frame}

\begin{frame}{Jamshidian's Decomposition Proof}
\begin{itemize}
	\item<1-> Since each $f_i$ is monotone also $\sum_i f_i$ is decreasing. Hence there is a unique solution $\hat{w}$ to 
	\begin{equation*}
		\sum_i f_i(\hat{w}) = K   
	\end{equation*}
	\item<2-> Each $f_i$ is decreasing so
	\begin{columns}
		\column{0.7\linewidth}
		\uncover<2->{\begin{equation*}
				\left(K - \sum_i f_i(W)\right)^+ = \left(\sum_i f_i(\hat{w}) -  \sum_i f_i(W)\right)^+ = 
		\end{equation*}}
		\uncover<3->{\begin{equation*}
				= \left(\sum_i (f_i(\hat{w}) -  f_i(W))\right)^+= \sum_i (f_i(\hat{w}) - f_i(W))\mathbbm{1}_{W\ge \hat{w}}
		\end{equation*}}
		\uncover<4->{\begin{equation*}
				= \sum_i \left(K_i - f_i(W)\right)^+\quad\qedsymbol
			\end{equation*}
		}
		\column{0.35\linewidth}
		\uncover<1->{
			\includegraphics[width=1\linewidth]{images/jamshidian_trick}}
	\end{columns}
\end{itemize}
\end{frame}
%Jamshidian's Condition: You correctly identify that fi​ must be monotone. In the context of interest rates, this means the model must be a one-factor model (where the whole curve moves in one direction relative to the short rate r). If you have a two-factor model, Jamshidian’s decomposition generally fails because ZCB prices for different tenors might move in opposite directions.


\begin{frame}{Back to Coupon Bond Option}
\begin{itemize}
	\item<1-> Consider a coupon bond which pays the following cash flows $\mathcal{C}=\{c_1,\dots,c_n\}$ at dates $\mathcal{T}=\{T_1,\ldots,T_n\}$.
	\item<2-> Let $t\leq T_1$, the bond price is given by
	\begin{equation*}
		\textbf{CB}(t,\mathcal{C},\mathcal{T})=\sum_{i=1}^n c_i \Pi(t, T_i, r(t))
	\end{equation*}
	\item<3-> Suppose we would like to calculate the price of a put option with strike $K$ on this coupon bond. The payoff reads
	\begin{equation*}
		\textbf{CBP}=\left[K-\textbf{CB}(t,\mathcal{C},\mathcal{T})\right]^+ = \left[K-\sum_{i=1}^n c_i \Pi(t, T_i, r(t))\right]^+
	\end{equation*}
\end{itemize}
\end{frame}

\begin{frame}{Coupon Bond Option}
\begin{itemize}
	\item<1-> Now apply the Jamshidian's decomposition to previous payoff.
	\item<2-> First need to find the interest rate value $r^*$ such that $\sum_{i=1}^n c_i \Pi(t, T_i, r^*) = K$.
	\item<3-> Assuming the interest rate model satisfies the required condition (which is true for Short Rate models for example)
	\begin{equation*}
		\frac{\partial \Pi(t,T_i,r(t))}{\partial r}<0,\;\forall 0<t<s
	\end{equation*}
	we can write the payoff as
	\begin{equation}
		\textbf{CBP}(t,T_i,\Pi,r^*) = \sum_{i=1}^n c_i [\Pi(t, T_i, r^*)-\Pi(t, T_i, r(t))]^+
		\label{eq:bond_option_payoff}
	\end{equation}
\end{itemize}
\end{frame}

\begin{frame}{Coupon Bond Option}
\begin{itemize}
	\item \cref{eq:bond_option_payoff} tells us that we can price a coupon bond option as a portfolio of options on ZCBs.
	\item The strike of these option is calculated as the value of a ZCB given a \emph{particular} value of the short rate, determined with a root finding procedure.
	\item In formulas the CBO with maturity $T$ and strike $K$ reads
	\begin{equation}
		\boxed{\textbf{CBP}(t,\mathcal{T},\mathcal{C},K) = \sum_{i=1}^n c_i \textbf{ZBP}(t,T_i,\Pi,r^*)}
	\end{equation}
\end{itemize}
\end{frame}

%The Affine Formula: In the slide "Swaption Pricing under Affine Models," you use τk​ inside Ar​ and Br​. Usually, these functions are defined as A(Tα​,Tk​) and B(Tα​,Tk​). Using τk​ might be confused with the year fraction. I recommend keeping the Tα​,Tk​ notation for clarity.

%The Expected Value: You use Et​. Ensure the measure is clear (typically the Tα​-forward measure or the risk-neutral measure with the specific numeraire).


\begin{frame}{Swaption Pricing under Affine Models}
	\begin{itemize}
		\item One of the most striking properties of \emph{Affine Models} (Vasicek, Hull-White) is that the ZCB price is an exponential function of the spot rate:
		\begin{equation*}
			P(T_\alpha, T_k) = A(T_\alpha, T_k) e^{-B(T_\alpha, T_k) r(T_\alpha)}
		\end{equation*} 
		\item A payer swaption (PSw) can be viewed as a Put option on a coupon bond with strike $K_{bond} = 1$:
		\begin{equation*}
			\textbf{PSw} = N P(t, T_\alpha) \mathbb{E}^{Q^{T_\alpha}}_t \left[ \left( 1 - \sum_{k=\alpha+1}^\beta c_k P(T_\alpha, T_k) \right)^+ \right]
		\end{equation*}
		\item By Jamshidian's Decomposition, we find $r^*$ such that $\sum c_k P(T_\alpha, T_k, r^*) = 1$.
	\end{itemize}
\end{frame}

\begin{frame}{Final Result: Swaption as a Portfolio of ZCB Options}
	\begin{itemize}
		\item Since each ZCB price is monotone in $r$, the swaption price is the sum of European Put options on the individual Zero-Coupon Bonds:
		\begin{equation}
			\boxed{\textbf{PSw}(t) = N \sum_{k=\alpha+1}^\beta c_k \textbf{ZBP}(t, T_\alpha, T_k, X_k)}
		\end{equation}
		where $X_k = P(T_\alpha, T_k, r^*)$ is the strike of the $k$-th ZCB option.
		\item \textbf{Conclusion:} In one-factor affine models, swaptions are not "new" products; they are simply a \textcolor{red}{static portfolio of ZCB options}.
	\end{itemize}
\end{frame}

"Jamshidian's decomposition is the reason why the Hull-White model is so fast. To calibrate the model to a Swaption, we don't need a Monte Carlo simulation; we just need a sum of Black-Scholes-like formulas for ZCB options."


%\begin{frame}{Swaption Pricing under Affine Models}
%\begin{itemize}
%	\item When interest rates are modeled using \textcolor{red}{Affine Short Rate Models} (Vasicek, Hull-White,\ldots) swpation pricing can be performed semi-analytically.
%	\item \cref{eq:swaption_payoff_std} can be rewritten, expressing the underlying swap in terms of bonds, as
%	\begin{equation*}
%		\textbf{PSw}=NP(t_0, T_\alpha)\mathbb{E}_t\left[\max\left(1-\sum_{k=\alpha+1}^\beta c_kP(T_\alpha,T_k), 0\right)\right]
%	\end{equation*}
%	with $c_k=K\tau_k$ for $k=\alpha+1,\ldots,\beta-1$ and $c_\beta=(1+K\tau_\beta)$.
%	\item One of the most striking properties of \emph{Affine Models} is that they relate ZCB price to a spot rate modeled according to 
%	\begin{equation*}
%		P(t,T) = A_r(t,T)e^{-B_r(t,T)r}
%	\end{equation*} 
%\end{itemize}
%\end{frame}
%
%\begin{frame}{Swaption Pricing under Affine Models}
%\begin{itemize}
%	\item Hence
%	\begin{equation*}
%		\textbf{PSw}=NP(t_0, T_\alpha)\mathbb{E}_t\left[\max\left(1-\sum_{k=\alpha+1}^\beta c_k A_r(\tau_k)e^{B_r(\tau_k)r(T_\alpha)}, 0\right)\right]
%	\end{equation*}
%	\item Using the Jamshidian decomposition the $\max$ operator can be swapped with the sum
%	\begin{equation*}
%		\textbf{PSw}=NP(t_0, T_\alpha)\sum_{k=\alpha+1}^\beta c_k \mathbb{E}_t\left[\max\left(\bar{K_k} - A_r(\tau_k)e^{B_r(\tau_k)r(T_\alpha)}, 0\right)\right]
%	\end{equation*}
%	whit $\bar{K_k} := A_r(\tau_k)e^{B_r(\tau_k)r^{*}}$, where the parameter $r^{*}$ is determined by solving
%	\begin{equation*}
%		\sum_{k=\alpha+1}^\beta c_k \left(A_r(\tau_k)e^{B_r(\tau_k)r^{*}}\right) = 1
%	\end{equation*}
%\end{itemize}
%\end{frame}
%
%\begin{frame}{Swaption Pricing under Affine Models}
%\begin{itemize}
%	\item As noted above for the bond options, each element of the sum in the formula above represents a European put option on a zero-coupon bond, which in the \emph{affine models} has a closed-form solution.
%	\item So a payer swaption price is thus given by
%	\begin{equation}
%		\boxed{\textbf{PSw}(t,T,N) = N\sum_{k=1}^n c_k \textbf{ZBP}(t,T_k,K_k)}
%	\end{equation}
%	while the receiver swaption price reads
%	\begin{equation}
%		\boxed{\textbf{RSw}(t,T,N) = N\sum_{k=1}^n c_k \textbf{ZBC}(t,T_k,K_k)}
%	\end{equation}
%\end{itemize}	
%\end{frame}




%\begin{frame}{Adapting to Swaptions}
%\begin{itemize}
%	\item When interest rates are modeled using \textcolor{red}{Affine Short Rate Models} it is rather simple to arrive to the swaption pricing formula.
%	\item \emph{Affine Models} indeed relates ZCB price to a spot rate model according to 
%	\begin{equation*}
%		P(t,T) = A(t,T)e^{-B(t,T)r}
%	\end{equation*} 
%	\item Hence the value $r^*$ can be determined as a solution of 
%	\begin{equation*}
%		\sum_{i=1}^n A(t,t_i)e^{-B(t,t_i)r^*}
%	\end{equation*}		
%	%%		\item Denote as usual with $\tau_i$ the year fraction between $t_{i-1}$ and $t_i$, fix $c_i=X\tau_i$ and $c_n = 1+X\tau_i$.
%	%%		\item Let the swap notional be equal to N.
%	%%		\item Thus for the price of a payer swaption we have to calculate the following payoff
%	%%		\begin{equation*}
%		%%			\left[1-CB(t,\mathcal{C},T)\right]^+
%		%%		\end{equation*}
%	%%		\item We can calculate this payoff via the procedure outlined before.
%\end{itemize}
%\end{frame}
%
%\begin{frame}{Swaption Pricing via Affine Models}
%\begin{itemize}
%	\item Consider an option on a swap which pays a fixed rate $X$ and receives LIBOR.  
%	\item Modeling the short rate with an affine model we can define $r^*$ at time $T$, such that
%	\begin{equation*}
%		\sum_{i=1}^n c_i A(t,T_i)e^{-B(t,T_i)r^*} = 1
%	\end{equation*}
%	where $c_i$ is the coupon value.
%\end{frame}




%To implement Jamshidian’s decomposition, the "engine" is finding the critical interest rate r∗. This is a standard part of a Quant's toolkit for calibrating Short Rate models.
%Numerical Procedure to find r∗
%
%Since the function G(r)=∑k=α+1β​ck​P(Tα​,Tk​,r)−1 is strictly monotonic (decreasing) and continuous, we can solve G(r∗)=0 using the Newton-Raphson method.
%
%Step-by-Step Algorithm:
%
%Initial Guess: Set r0​ equal to the current forward swap rate Sα,β​(t).
%
%Iterative Step: Update the estimate using the formula:
%rn+1​=rn​−G′(rn​)G(rn​)​
%
%Derivative Calculation: In an affine model where P(Tα​,Tk​,r)=Ak​e−Bk​r, the derivative is easy to compute analytically:
%G′(r)=k=α+1∑β​−ck​B(Tα​,Tk​)P(Tα​,Tk​,r)
%
%Convergence: Stop when ∣G(rn​)∣<ϵ (typically 10−8).
%
%Final Review of the "Jamshidian" Section
%
%I have checked for duplications and logical gaps in your slides:
%
%Slide "Jamshidian's Decomposition": The theorem is clean. Note that \mathbbm1W≥w^​ is the same as \mathbbm1∑fi​(W)≤K​ due to the monotonic decreasing property.
%
%Slide "Back to Coupon Bond Option": Your definition of ci​ is correct. Just ensure cn​ includes the final notional repayment (100%).
%
%Slide "Swaption Pricing under Affine Models": You have three slides with this title.
%
%Merge Recommendation: Combine the second and third ones. The second shows the "max" inside the expectation, and the third shows the final ZBP sum. This prevents the presentation from feeling repetitive.


\begin{frame}{Numerical Implementation: Finding $r^*$}
	\begin{itemize}
		\item To use Jamshidian's decomposition, we must solve for $r^*$ in:
		\begin{equation*}
			f(r) = \sum_{k=\alpha+1}^\beta c_k A(T_\alpha, T_k) e^{-B(T_\alpha, T_k) r} - 1 = 0
		\end{equation*}
		\item Since $f'(r) < 0$, the function is strictly decreasing, ensuring a unique solution.
		\item \textbf{Newton-Raphson Iteration:}
		\begin{equation*}
			r_{new} = r_{old} - \frac{f(r_{old})}{f'(r_{old})}
		\end{equation*}
		\item Because $f(r)$ is nearly linear in the region of interest, convergence is typically achieved in 3--5 iterations.
	\end{itemize}
\end{frame}


\subsection{Bermudan Swaption}
\begin{frame}{Bermudan Swaptions}
	\begin{itemize}
		\item<1-> It is a swaption in which the optionality can be exercised at a \textcolor{red}{predetermined set of dates} (not only one).
		\item<2-> It is useful for hedging callable bonds (especially if step-up, i.e. with the coupon increasing with time).
		\item<3-> A \textcolor{red}{Bermudan Swaption} gives the holder the right but not the obligation to enter in an interest rate swap contract at different dates (usually the swap reset dates) with some days of notification to the counter-party.
		\item<4-> The interest rate swap the holder can enter into, is the same existing contract, so if the holder does not exercise at the first date in the call schedule, the option for the following periods is just written on shorter swaps.
	\end{itemize}
\end{frame}

\begin{frame}{Bermudan Swaption Example}
	\begin{itemize}
		\item<1-> As an example consider the following: receiver Bermudan Swaption written on a 3 years swap with the first call date $2y$ from now (we suppose semi-annual payments).
		\item<2-> If at the end of the second year she will not exercise, six months later she will have to decide if entering or not in the same remaining swap which now has become a $2y6m$ swap.
		\item<3-> If again she will not exercise the last possibility will involve the decision of whether or not to enter on the $2y6m-3y$~$FRA$.
	\end{itemize}
\end{frame}		

%\begin{frame}{Bermudan Swaption Payoff}
%\begin{itemize}
%	\item At the maturity $T$, the payoff of a Bermudan swaption is given by
%	\begin{equation*}
	%%		\textbf{BS}(T) = \max(0, V_{\text{swap}}(T))
	%%	\end{equation*}
%%	where $V_{swap}(T)$ is the value of the underlying swap in $T$.
%%	\item At any exercise date $T_i$, the payoff of the Bermudan swaption is given by
%%	\begin{equation*}
	%%	\textbf{BS}(T_i) = \max(K(T_i), V_{\text{swap}}(T_i))
	%%	\end{equation*}
%%	where $V_{\text{swap}}(T_i)$ is the exercise value of the swap and $K(T_i)$ is the intrinsic value, i.e. the holder of the option receives $\max(K(T_i), 0)$ if the option is exercised at time $T_i$.
%%	CONTROLLARE ULTIMO DISCORSO
%%\end{itemize}
%%\end{frame}

\begin{frame}{Bermudan Swaption Pricing}
	\begin{itemize}
		\item<1-> Some interest rate instruments can be priced just looking at the term structure of interest rates (FRA and Swaps)
		\begin{itemize}
			\item the only problem is: which is the right term structure ? (this is a lesson from the 2008 crisis).
		\end{itemize}
		\item<2-> Some other instruments cannot be priced only with the yield curve: we need the future (risk-neutral) evolution of the rates.
		%\item Non-linearities come in !
		%\item Swaptions are non-linear products and may require to model the correlation between forward LIBOR rates.
		\item<3-> Given the complexity of Bermudan swaption valuation (which belongs to the latter class), there is no closed form solution.		
		\item<3-> Typically tree techniques or the Longstaff-Schwartz method are used.
	\end{itemize}
\end{frame}

\begin{frame}{Longstaff-Schwartz Algorithm}
	\begin{itemize}
		\item Consider a Bermudan put option with strike $K$ and $n$-years maturity. Each year you can choose whether to exercise or not.
		%Let's implement a MC which actually simulates, besides the evolution of the market, what an investor holding this option would do.
		\item The evolution of the asset value and of the money market account gives
		\begin{equation}
			\begin{gathered}
				S(t_1=1y) = S(t_0)e^{(r-\frac{1}{2}\sigma^2)(t_1-t_0)+\sigma\sqrt{t_1-t_0}\mathcal{N}(0,1)} \\
				B(t_1=1y)=B(t_0)e^{r(t_1-t_0)}
			\end{gathered}
		\end{equation}
		\item At $t_1$ the investor can:
		\begin{itemize}
			\item \textbf{exercise:} in this case we knows exactly the payoff;
			\item \textbf{does not exercise:} the contract has become a European Put Option with shorter maturity.
		\end{itemize}
		\item \emph{At this point how does the investor know if it is convenient to exercise?}
		\item The decision will depend on each choice convenience:
		\begin{equation*}
			\max[K-S_{t_1}, \textbf{Put}(t_1,T;S_{t_1},K)]
		\end{equation*}
		where $\textbf{Put}(t_1,T;S_{t_1},K)$ is the price of the Put (i.e. the \textbf{continuation value}, the value of holding the option instead of early exercising it).
	\end{itemize}
\end{frame}

\begin{frame}{Longstaff-Schwartz Algorithm}
	\begin{itemize}
		\item So the premium of the option is the average of this discounted payoffs calculated in each Monte Carlo simulation
		\begin{equation*}
			\cfrac{1}{N}\sum_i\max[K-S_{t_1}, \textbf{Put}(t_1,T;S_{t_1},K)]
		\end{equation*}
		\item In this case we could price this product because we have an analytical pricing formula for the Put (Black formula), but what if we didn't ?
		\item \textbf{Brute force solution:} for each realization of $S_{t_i}$ we run another Monte Carlo to price the alternative continuation value.
		\item We need to implement nested MC which is usually a very bad idea. Its execution time grows as $N^2$, which becomes prohibitive when you deal with multiple exercise dates !
	\end{itemize}    
\end{frame}

\begin{frame}{Longstaff-Schwartz Algorithm}
	Longstaff and Schwarz were looking for a smarter solution analyzing the relationship between the continuation value and the realizations of $S$.
	%We can plot the discounted payoff at maturity, $P_i$ vs $S_i(t_1)$.
	\begin{columns}
		\column{0.5\linewidth}
		\includegraphics[width=0.9\linewidth]{images/longstaff_simulation}
		
		\small{They found the Put analytical price (green line) was interpolating the cloud of MC points, so \emph{the price at time $t$ can be computed as the barycentre of the discounted payoff cloud.}}
		\column{0.5\linewidth}    
		\includegraphics[width=0.9\linewidth]{images/longstaff_scatter_1}
	\end{columns}
	
	\textbf{Estimating the future value of an option $(V)$ can be seen as the problem of finding the curve that best fit the cloud of discounted payoffs at the date of interest.}
\end{frame}

\begin{frame}{Longstaff-Schwartz Algorithm}
	
	\begin{itemize}
		\item So they found an \emph{empirical pricing formula} for the unknown contract to be used in the Monte Carlo Simulation
		\begin{equation}
			V(t_i,T,S_{t_i},K)=c_0+c_1 S(t_i) + c_2 S^2(t_i) + c_3 S^3(t_i) + c_4 S^4(t_i) + c_5 S^5(t_i)
		\end{equation}
		\begin{center}
			\includegraphics[width=0.35\linewidth]{images/longstaff_scatter_2}
		\end{center}
		The formula is obviously fast, the cost of the algorithm being just the best fit. 
		\item Clearly we could have used any form for the curve (not only a polynomial), but the small improvement does not justify the increased complexity.
	\end{itemize}
\end{frame}


%%\begin{frame}{Bermudan Swaption Pricing: Outline}
%%	\begin{itemize}
	%%	\item<1-> Consider a tenor structure $\mathcal{T}=\{T_i\}^\beta_{i=\alpha}$ and a Bermudan receiver swaption with time $t$ value $\textbf{RBS}(t,K)$.
	%%	\item<2-> Assuming no prior exercise, at any time point $T_i$ the swaption holder has the right to receive the exercise value $V_e$ of the swaption, i.e., present value of the underlying swap:
	%%	\begin{equation}
		%%	V_e(T_i)=(K-S_{i,\beta}(T_i))+\sum^\beta_{k=i+1} P(T_i,T_k)\tau_k
		%%	\label{eq:exercise_value}
		%%	\end{equation}
	%%	\item<3-> The exercise value has to be compared to the so-called continuation value, $V_c$, of holding the option beyond $T_i$:
	%%	\begin{equation}
		%%	V_c(T_i)=\mathbb{E}[\textbf{RBS}(T_{i+1},K)|S_{i,\beta}(T_i)]
		%%	\label{eq:continuation_value}
		%%	\end{equation}
	%%\end{itemize}
	%%\end{frame}
	%%
	%%\begin{frame}{Bermudan Swaption Pricing: Outline}
	%%	\begin{itemize}
		%%		\item<1-> The value of the Bermudan swaption can now be given in terms of \cref{eq:exercise_value} and \cref{eq:continuation_value}, recursively.
		%%		\item<2-> The evaluation of proceeds backward in time: at $T_{\beta-1}$
		%%		the value of the Bermudan is known, i.e. the swaption payoff
		%%		\begin{equation*}
			%%			\textbf{RBS}(T_{\beta-1},K)=P(T_{\beta-1},T_\beta)\tau_\beta(K-S_{\beta-1,\beta}(T_{\beta-1}))^+
			%%		\end{equation*}
		%%		\item <3->This allows to update the continuation value at $T_{\beta-2}$ with \cref{eq:continuation_value} and compare it to the exercise value
		%%		\begin{equation*}
			%%			\textbf{RBS}(T_j,K)=\max(V_e(T_j),V_c(T_j)),\quad\text{for }j=\beta-2,\beta-3,\ldots,n
			%%		\end{equation*}
		%%	\end{itemize}
	%%\end{frame}
	%%
	%%\begin{frame}{Bermudan Swaption Pricing: Outline}
	%%	\begin{itemize}
		%%		\item<1-> This procedure of comparing “backwardly-cumulated” continuation value with exercise value and deciding upon a swaption exercise is repeated until the initial valuation date is reached, at which point the algorithm yields a price estimate for the Bermudan swaption. 
		%%		\item<2-> The calculation of the continuation value is clearly model-dependent and the choice of modeling framework itself often determines the scope of available numerical techniques.
		%%\end{itemize}
		%%\end{frame}
		
		
		\subsection{Callable Bonds}
		\begin{frame}{Callable Coupon Bond}
			\begin{itemize}
				\item<1-> A \textcolor{red}{callable bond} is a bond in which, on the call date(s) (there can be more than one), the \textbf{issuer} has the right, but not the obligation, to buy back (redeem) the bonds from the bond holders at a defined call price.
				\item<2-> We have seen that a swap can be regarded as an exchange of bonds. It is easy to guess that a replica for the callable bond price can be obtained by simply adding a (swap-)option to the swap used to price the underlying bond.
				\item<3-> If there are multiple callability dates is clear that we need a \textbf{Bermudan} swaption.
				\item<3-> With a receiver bermudan swaption with the same contractual conventions of the bond (i.e. the strike of the swaption is equal to the coupon of the bond) we can offset it. %; which represents the economic equivalent of calling the bond at par.
				%\item So $\max(CCBP(T,S,K,\tau)-100, 0)$ can be represented as $\max(K-S(T_j,\beta)(T), 0)$.
				%\item Intuition: long on the bond, short on the rates.
			\end{itemize}
		\end{frame}
		
		\begin{frame}{Callable vs Non Callable Coupon Bonds}
			\begin{itemize}
				\item<1-> \emph{Ceteris paribus} a non callable coupon bond has an higher price than a callable one because the callability option adds value to the issuer
				\begin{equation*}
					\text{price of callable bond} = \text{price of straight bond} – \text{price of call option}.
				\end{equation*}
				\item<2-> If interest rates decline, the issuer of a callable bond can issue new debt, at lower interest rate than the original one, and use the the proceeds from this second issue to pay off the earlier callable bond by exercising the call feature.
				\item<2-> As a result, \textcolor{red}{the company has refinanced its debt by paying off the higher-yielding callable bonds with the newly-issued debt at a lower interest rate}.		
				%	\item At inception both must be worth 100 (apart from other costs and fees which we will neglect).
				%	\item A typical coupon bond, once credit risk is isolated and remunerated, will pay the average market rates prevailing at the time of the issue. These are related to the swap rate prevailing at that moment (remember that the swap rate is sort of average of forward rates).
			\end{itemize}
		\end{frame}
		
		%\begin{frame}{Callable vs Non Callable Coupon Bonds}
		%\begin{itemize}
		%	\item Suppose credit risk is zero: in this ideal case the coupon bond will pay the corresponding swap rate prevailing on the market.
		%	If we price a $5y$ bond, at inception, the following must hold
		%	\begin{equation*}
			%		CBP(0,5,K,\tau)=100-NPV_{\text5y-swap(0)}=100
			%	\end{equation*}
		%	\item Which implies $NPV_{\text{5y-swap(0)}}=0$, hence $K=S_{\text{5y-swap(0)}}$. This means that credit consideration apart, a bank must pay the market prevailing rate when it issues a bond. %And this should not surprise anyone.
		%\end{itemize}
		%\end{frame}
		
		%\begin{frame}{Callable vs Non Callable Coupon Bonds}
		%\begin{itemize}
		%	\item Consider now the same bond with a callability option after two years each six months.
		%	\item Let us denote with $RBS(t,6m,T_{1c},T_\beta,K,N)$ the Bermudan receiver swaption with first call date $T_{1c}$ and subsequent ones every six months. The last payment date is equal to the swap maturity.
		%	\item In this case the call dates vector is $[2y,2y6m,3y,3y6m,4y,4y6m]$.
		%	\item Suppose $RBS(0,6m,T_{2y},T_{5y},K_1,N)>0$
		%	\begin{equation*}
			%		CBP(0,5,K,\tau)=100-(NPV_{\text{5y-swap(0)}}+NPV_{RBS})=100
			%	\end{equation*}
		%\end{itemize}
		%\end{frame}
		
		\begin{frame}{Risk Analysis of Callable Bonds}
			\begin{itemize}
				\item A callable bond benefits the issuer, and so investors of these bonds are compensated with a more attractive interest rate than on otherwise similar non-callable bonds.
				%\item Paying down debt early by exercising callable bonds saves a company interest expense and prevents the company from being put in financial difficulties in the long term if economic or financial conditions worsen. 
				\item<1-> The investor might not make out as well as the company when rates decrease and the bond is called. Not only she loses the remaining interest payments but unlikely she will be able to match the original coupon.
				\textcolor{red}{This situation is known as reinvestment risk}. 
				
				%For example, let's say a 6\% coupon bond is issued and is due to mature in five years. An investor purchases 10000 worth and receives coupon payments of 6\% x 10,000 or 600 annually. Three years after issuance, the interest rates fall to 4\%, and the issuer calls the bond. The bondholder must turn in the bond to get back the principal, and no further interest is paid.
				%In this scenario, not only does the bondholder lose the remaining interest payments but it would be unlikely they will be able to match the original 6\% coupon. 
				\item<2-> As a result, a callable bond may not be appropriate for investors seeking stable income and predictable returns.
			\end{itemize}
		\end{frame}
		
		\begin{frame}{Break-even Rate}
			\begin{itemize}
				\item<1-> In general, we can assume that \textbf{it is optimal for the issuer to minimize the value of the callable bond}.
				\item<2-> She will exercise the option \emph{if the price of the callable bond exceeds the exercise price ($X$) at the notice date ($t_n$)}. 
				\item<3-> Otherwise, she will give up the option right and the callable bond price will be equal to that of the non-option bond.
				\item<4-> Denote by $r_b$ the \emph{break-even interest rate} which represents the rate value such that the issuer is indifferent between exercising the option or not.
				\begin{equation}
					X\cdot D(t_n, t_c) - P(r,t_n)=0    
				\end{equation}
				where $D(t_n, t_c)$ is the discount factor between notice and exercise dates ($t_c$) and $P(r,t_n)$ the price of the callable bond an instant before the notice date.
				\item<4-> We know the value of the callable bond $P(r,t_n)$ since it is equal to the value of the non-option bond. 
			\end{itemize}
			%We know the value of the callable bond $P(r,t_n)$ since it is equal to the value of the non-option bond. The solution of equation is the cross point between the two curves.    
		\end{frame}
		
		\begin{frame}{Break-even Rate}
			\textbf{The solution is the cross point between the two curves.}
			\begin{center}
				\includegraphics[width=0.5\linewidth]{images/callable_bond}
			\end{center}
		\end{frame}
		
\end{document}
