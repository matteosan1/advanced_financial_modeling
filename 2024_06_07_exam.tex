\documentclass[12pt,a4paper]{exam}

\usepackage[utf8]{inputenc}
\usepackage[T1]{fontenc}
\usepackage{amsmath}
\usepackage{amsfonts}
%\usepackage{amssymb}
\usepackage{graphicx}
\usepackage{geometry}
\usepackage{cancel}
\usepackage{enumitem}

\geometry{a4paper, margin=2cm}

\newcommand{\ito}{It$\hat{o}$}
\newcommand{\expect}[1]{\mathbb{E}^\mathbb{#1}}
\newcommand{\expectt}[2]{\mathbb{E}_{#2}^\mathbb{#1}}

\usepackage{cprotect}

\usepackage{xcolor}
\definecolor{maroon}{cmyk}{0, 0.87, 0.68, 0.32}
\definecolor{halfgray}{gray}{0.55}
\definecolor{ipython-frame}{RGB}{207, 207, 207}
\definecolor{ipython-bg}{RGB}{247, 247, 247}
\definecolor{ipython-red}{RGB}{186, 33, 33}
\definecolor{ipython-green}{RGB}{0, 128, 0}
\definecolor{ipython-cyan}{RGB}{64, 128, 128}
\definecolor{ipython-purple}{RGB}{170, 34, 255}
\usepackage{listings}
\lstdefinelanguage{iPython}{
	morekeywords={access,and,del,except,exec,in,is,lambda,not,or,raise},
	morekeywords=[2]{for,print,abs,all,any,basestring,bin,bool,bytearray,callable,chr,classmethod,cmp,compile,complex,delattr,dict,dir,divmod,enumerate,eval,execfile,file,filter,float,format,frozenset,getattr,globals,hasattr,hash,help,hex,id,input,int,isinstance,issubclass,iter,len,list,locals,long,map,max,memoryview,min,next,object,oct,open,ord,pow,property,range,reduce,reload,repr,reversed,round,set,setattr,slice,sorted,staticmethod,str,sum,super,tuple,type,unichr,unicode,vars,xrange,zip,apply,buffer,coerce,intern,elif,else,if,continue,break,while,class,def,return,try,except,import,finally,try,except,from,global,pass, True, False},
	sensitive=true,
	morecomment=[l]\#,%
	morestring=[b]',%
	morestring=[b]",%
	moredelim=**[is][\color{black}]{@@}{@@},
	%%
	%morestring=[s]{'''}{'''},% used for documentation text (mulitiline strings)
	%morestring=[s]{"""}{"""},% added by Philipp Matthias Hahn
	%%
	%morestring=[s]{r'}{'},% `raw' strings
	%morestring=[s]{r"}{"},%
	%morestring=[s]{r'''}{'''},%
	%morestring=[s]{r"""}{"""},%
	%morestring=[s]{u'}{'},% unicode strings
	%morestring=[s]{u"}{"},%
	%morestring=[s]{u'''}{'''},%
	%morestring=[s]{u"""}{"""}%
	%
	% {replace}{replacement}{lenght of replace}
	% *{-}{-}{1} will not replace in comments and so on
	%literate=
	%{\%}{{{\color{ipython-purple}+}}}1,
	%{�}{{\'a}}1 {�}{{\'e}}1 {�}{{\'i}}1 {�}{{\'o}}1 {�}{{\'u}}1,
	%{�}{{\'A}}1 {�}{{\'E}}1 {�}{{\'I}}1 {�}{{\'O}}1 {�}{{\'U}}1
	%{�}{{\`a}}1 {�}{{\`e}}1 {�}{{\`i}}1 {�}{{\`o}}1 {�}{{\`u}}1
	%{�}{{\`A}}1 {�}{{\'E}}1 {�}{{\`I}}1 {�}{{\`O}}1 {�}{{\`U}}1
	%{�}{{\"a}}1 {�}{{\"e}}1 {�}{{\"i}}1 {�}{{\"o}}1 {�}{{\"u}}1
	%{�}{{\"A}}1 {�}{{\"E}}1 {�}{{\"I}}1 {�}{{\"O}}1 {�}{{\"U}}1
	%{�}{{\^a}}1 {�}{{\^e}}1 {�}{{\^i}}1 {�}{{\^o}}1 {�}{{\^u}}1
	%{�}{{\^A}}1 {�}{{\^E}}1 {�}{{\^I}}1 {�}{{\^O}}1 {�}{{\^U}}1
	%{�}{{\oe}}1 {�}{{\OE}}1 {�}{{\ae}}1 {�}{{\AE}}1 {�}{{\ss}}1
	%{�}{{\c c}}1 {�}{{\c C}}1 {�}{{\o}}1 {�}{{\r a}}1 {�}{{\r A}}1
	%{�}{{\EUR}}1 {�}{{\pounds}}1
	%
	%{^}{{{\color{ipython_purple}\^{}}}}1
	%{=}{{{\color{ipython_purple}=}}}1
	%%
	%*{-}{{{\color{ipython_purple}-}}}1
	%{*}{{{\color{ipython_purple}$^\ast$}}}1
	%{/}{{{\color{ipython_purple}/}}}1%%
	%{+=}{{{+=}}}1
	%{-=}{{{-=}}}1
	%{*=}{{{$^\ast$=}}}1
	%{/=}{{{/=}}}1,
	%
	identifierstyle=\color{black}\footnotesize\ttfamily,
	commentstyle=\color{ipython-cyan}\footnotesize\itshape\ttfamily,
	stringstyle=\color{ipython-red}\footnotesize\ttfamily,
	keepspaces=true,
	showspaces=false,
	showstringspaces=false,
	rulecolor=\color{ipython-frame},
	frame=single,
	frameround={t}{t}{t}{t},
	%framexleftmargin=6mm,
	%numbers=left,
	%numberstyle=\color{ipython-cyan},
	backgroundcolor=\color{ipython-bg},
	%   extendedchars=true,
	basicstyle=\footnotesize\ttfamily,
	keywordstyle=[2]\color{ipython-green}\bfseries\footnotesize\ttfamily, 
	keywordstyle=\color{ipython-purple}\bfseries\footnotesize\ttfamily
}

\lstdefinelanguage{iOutput} {
	sensitive=true,
	identifierstyle=\color{black}\small\ttfamily,
	stringstyle=\color{ipython-red}\small\ttfamily,
	keepspaces=true,
	showspaces=false,
	showstringspaces=false,
	rulecolor=\color{ipython-frame},
	%frame=single,
	%frameround={t}{t}{t}{t},
	%backgroundcolor=\color{ipython-bg},
	basicstyle=\small\ttfamily,
}

\lstnewenvironment{ipython}[1][]{\lstset{language=iPython,mathescape=true,escapeinside={*@}{@*}}%
}{%
}

\lstnewenvironment{ioutput}[1][]{\lstset{language=iOutput,mathescape=true,escapeinside={*@}{@*}}%
}{%
}

\qformat{\thequestion. \textbf{\thequestiontitle}\hfill}
\title{Advanced Financial Modeling Course 24/25\\ Exam}
\author{Prof. Andrea Carapelli, Prof. Matteo Sani}
\date{$7^{\mathrm{th}}$ June 2024}

%\printanswers
\noprintanswers
\begin{document}
\maketitle
%\addpoints{exam}
%\begin{center}
%\fbox{\fbox{\parbox{5.5in}{\centering
%Answer the questions in the spaces provided. If you run out of room for an answer, continue on the page back.}}}
%\end{center}

\begin{center}
\vspace{5mm}
\makebox[0.75\textwidth]{Student's name:\enspace\hrulefill}
\end{center}

\section*{Questions}
\vspace{.5cm}
\begin{questions}

%%%%%%%%%%%%%%%%%%%%%%%%%%%%%%%%%%
\titledquestion{Interest Rate Derivatives} 
Assume that \emph{Horizon Funds Co.} has agreed to pay a 6M EURIBOR and receive a fixed interest rate $K=4.0\%$ p.a. (with interest payable every six months) from a face value $N=100$M. The swap is 1.25 years to expire. The interest rates for 3, 9 and 15 months are 3.75\%, 3.63\% and 3.30\% respectively (interest rates are continouosly compounded). The 6M EURIBOR is currently $L=3.9\%$.

Calculate the value of this swap from \emph{Horizon Funds Co.}.
%\makeemptybox{10cm}

\begin{solution}
Since the company is receiving fixed the value of the contract, in a mono-curve framework, is:
\begin{equation*}
NPV = NPV_{\text{fixed}} - NPV_{\text{float}} = \sum_i Ke^{-rt_i} + N e^{-rT} - (L + N) e^{-r_1 t_1}
\end{equation*}
Hence
\begin{equation*}
NPV_{\text{fixed}} = \sum_i Ke^{-rt_i} + N e^{-rT} = 101.81 M
\end{equation*}
and
\begin{equation*}
NPV_{\text{float}} = (L + N) e^{-r_1 t_1} = 102.93 M
\end{equation*}

The contract $NPV$ for \emph{Horizon Funds Co.} results
\begin{equation*}
NPV = NPV_{\text{fixed}} - NPV_{\text{float}} = -1.12 M
\end{equation*}
\end{solution}

%%%%%%%%%%%%%%%%%%%%%%%%%%%%%%%%%%%%%%%%%%%%%%%%%%%

\titledquestion{Martingale Processes}
Let $S$ be a martingale satisfying the stochastic differential equation $dS_t = \sigma S_t dW_t$, starting from $S_0 = 1$,
where $\sigma$ is a strictly positive constant.
\begin{parts}
%\begin{enumerate}[label=(\alph*),font=\itshape]
\part Check that $S_t$ is strictly positive almost surely for all $t \geq 0$;
\part Compute explicitly $X_t := S_t^{-1}$;
\part Let $\mathbb{Q}$ be a new probability measure defined via $d\mathbb{Q} := S_t d\mathbb{P}$. What is the law of $X_t$ under $\mathbb{Q}$ ? 
\part Show finally the Put-Call symmetry (different from the Put-Call parity!!!!):
\begin{equation*}
\mathbb{E}^{\mathbb{P}}(S_T-K)^+ = K\mathbb{E}^{\mathbb{Q}}\left[(K^{-1}-X_T)^+\right]
\end{equation*}
\end{parts}
%\fillwithlines{8cm}

\begin{solution}
\begin{parts}
%\begin{enumerate}[label=(\alph*),font=\itshape]
\part It$\hat{o}$'s lemma implies that $S_t = \exp (-\frac{1}{2}\sigma^2 t + \sigma W_t)$ for any $t \geq 0$. Since the Brownian motion does not
explode to infinity over any finite time horizon, the result follows.
\part Using the previous representation, we immediately have
\begin{equation*}
X_t = S_t^{-1} = \exp\left(\frac{1}{2}\sigma^2 t - \sigma W_t\right)
\end{equation*}
It further satisfies the stochastic differential equation (by It$\hat{o}$’s lemma):
\begin{equation*}
dX_t = -\frac{dS_t}{S_t^2} + \frac{dS^2_t}{S_t^3} = -\sigma X_t dW_t + \sigma^2_t X_t dt
\end{equation*}
\part Since $S_t$ is a true strictly positive martingale, $\mathbb{Q}$ is a well-defined probability measure, equivalent to $\mathbb{P}$.
With the Girsanov theorem a Brownian motion under $\mathbb{Q}$ can be determined, indeed:
\begin{equation*}
dW^{\mathbb{Q}} = -\gamma dt + dW^{\mathbb{P}}\quad\left(\text{with }\gamma = \frac{-f_t}{\sigma_t}=\sigma \right)
\end{equation*}
Hence Therefore the process $B_t$ defined by $B_t := W_t - \sigma t$ is a standard Brownian motion under $\mathbb{Q}$, and so is $W^{\mathbb{Q}} := -B$, and hence 
\begin{equation*}
dX_t = -\sigma X_t(dW_t - \sigma dt) = \sigma X_t dW_t^{\mathbb{Q}}
\end{equation*}
Under $\mathbb{Q}$, the process $X$ is therefore a geometric Brownian motion.
\part Using the change of measure introduced previously, we can write, for all $K>0$,
\begin{equation*}
\mathbb{E}^{\mathbb{P}}(S_T-K)^+ = \mathbb{E}^{\mathbb{P}}\left[S_T\left(1-\frac{K}{S_T}\right)^+\right] = K \mathbb{E}^{\mathbb{Q}}\left[\left(\frac{1}{K}-\frac{1}{S_T}\right)^+\right] = K\mathbb{E}^{\mathbb{Q}}\left[(K^{-1}-X_T)^+\right]
\end{equation*}
\end{parts}
\end{solution}

%%%%%%%%%%%%%%%%%%%%%%%%%%%%%%%%%%%%%%%%%%%%%%%%%%%
\titledquestion{Extended Short Rate Models}
Define the HJM dynamics of $f(t,T)$ under the risk neutral measure $\mathbb{Q}$
\begin{equation*}
    df(t,T) = \left( \sigma_f(t,T) \int_t^T  \sigma_f(t,u) du \right) dt + \sigma_f(t,T)dW_t
\end{equation*}
Where $\sigma_f$ is the instantaneous forward volatility and $W$ is a Brownian motion. Recall the change of numeraire equality:
\begin{equation*}
    \frac{d \mathbb{Q}^T}{d \mathbb{Q}} = \frac{P(t,T)}{P(0,T)} \frac{B(0)}{B(t)}
\end{equation*}
\begin{parts}
	\part Show that  $f(t,T)$ is a martingale under the $ \mathbb{Q}^T$ forward measure, $dB_t$ is the process of the money market account (deterministic solution) and $dP_t$ is a log-Normal process for the zero coupon bond dynamics;
   \part  What is the difference between forward measure $ \mathbb{Q}^T$ and terminal forward measure $ \mathbb{Q}^{T_f}$?
   \part What is the advantage of "extending" short rate models? 
\end{parts}

\begin{solution}
\begin{parts}
    \part The goal is to define another Brownian under the new probability and establish a connection with the previous. Recall the change of numeraire
\begin{equation*}
    \frac{d \mathbb{Q}^T}{d \mathbb{Q}} = \frac{P(t,T)}{P(0,T)} \frac{B(0)}{B(t)}
\end{equation*}
We know the dynamics of $B(t)$, then apply Ito to $ln P(t,T)$ to get the needed dynamics
\begin{align*}
    \frac{P(t,T)}{P(0,T)} &= e^{\int_0^t (r_u - \frac{1}{2}\sigma_P^2(u,T))du + \int_0^t \sigma_P(u,T)dW_u} \\
    \frac{B(0)}{B(t)} &= e^{-\int_0^t r_u du}
\end{align*}
now substitute to get 
\begin{equation}\label{eq:RN_derivative_HJM}
     \frac{d \mathbb{Q}^T}{d \mathbb{Q}} = e^{- \frac{1}{2} \int_0^t \sigma_P^2(u,T)du + \int_0^t \sigma_P(u,T)dW_u}
\end{equation}
Recall that by the \textbf{Girsanov} theorem, if we have a Brownian motion $W_t$ under $\mathbb{Q}$, and we introduce a new process $y(t) = \int_0^t y_u du$ then $W^T = W_t - \int_0^t y_u du$ is a Brownian motion under $\mathbb{Q}^T$ defined via the R-N derivative
\begin{equation}\label{eq:RN_derivative}
    \frac{d \mathbb{Q}^T}{d \mathbb{Q}} = e^{-\frac{1}{2} \int_0^t y_u^2 du + \int_0^t y_u dW_u}
\end{equation}
in differential form
\begin{equation*}
    dW_t^T = dW_t - y_t dt
\end{equation*}
If we compare the RN derivative [\ref{eq:RN_derivative}] with the equation [\ref{eq:RN_derivative_HJM}] we see that we established a connection as  $y_u = \sigma_P(u,T)$ and we can write the new Brownian as
\begin{equation*}
    dW_t^T = dW_t - \sigma_P(t,T)dt = dW_t + \int_t^T \sigma_f(t,u)du dt
\end{equation*}
now we can replace the Brownian in the Risk Neutral HJM and get the dynamics under the new Forward probability measure
\begin{equation*}
        df(t,T) = \left( \sigma_f(t,T) \int_t^T  \sigma_f(t,u) du \right) dt + \sigma_f(t,T) \left( dW_t^T - \int_t^T \sigma_f(t,u)du dt \right)
\end{equation*}
the drift term simplifies as $f(t,T)$ is a martingale under he $ \mathbb{Q}^T$ forward measure
\begin{equation*}\label{eq:FWD_HJM}
        df(t,T) =  \sigma_f(t,T) dW_t^T 
\end{equation*}
    \part  Typically, you can use the \textbf{longest maturity $P(t,T_f)$} where $T_f > T$ where $T_f$ is the numeraire of the \textbf{Terminal forward measure}. From the Girsanov theorem we get
\begin{equation*}
    dW_t^{T_f} = dW_t - \sigma_P^2(t,T_f)dt = dW_t + \int_t^{T_f} \sigma_f(t,u)du dt
\end{equation*}
if we substitute in the Risk Neutral equation [\ref{eq:RN_HJM}] as we did before we get
\begin{equation*}\label{eq:TERM_HJM}
    df(t,T) = - \left( \sigma_f(t,T) \int_T^{T_f}  \sigma_f(t,u) du \right) dt + \sigma_f(t,T)dW^T_t
\end{equation*}
we can see now the difference between probability measures:
\begin{itemize}
    \item $\mathbb{Q} \to $ the drift integral will be defined with the remaining maturity;
    \item $\mathbb{Q}^T \to $ drift will be zero;
    \item $\mathbb{Q}^{T_f} \to $ the drift integral will be defined between $T$ (maturity of the modelled forward) and $T_f$ (maturity of the terminal numeraire)
\end{itemize}
No matter which is the probability measure, volatility drives the dynamics of $f(t,T)$.
 \part Extended models provide perfect fit to the term structure in $t_0$ (curve is an input of the model) and better fit to volatility structures with respect to standard affine models.
\end{parts}
\end{solution}

%%%%%%%%%%%%%%%%%%%%%%%%%%%%%%%%%%%%%%%%%%%%%
\titledquestion{xVA}
%\begin{enumerate}[label=(\alph*),font=\itshape]
\begin{parts}
    \part Define briefly Counterparty Credit Risk and xVA. Show that, at valuation time $t_0$, with $\tau > t_0$ (no default happened yet), the price of a derivative under counterparty risk is:
    $$ V_D(t) = V(t) - \mathbb{E}^Q \left[ 1_{\tau\leq T} \cdot \text{LGD}(\tau) \cdot D(t,\tau) \cdot \text{V}(\tau) \right] $$
    % \item Suppose you want to estimate future exposure for a linear interest rate derivative (assume that the only risk factor is interest rate) using the Hull and White model.
    % The integrated process reads:
    % \begin{equation*}
    %     r(t) = r(s)e^{-\kappa(t-s)} + \alpha (t) - \alpha(s) e^{-\kappa(t-s)} + \sigma \int_{s}^{t} e^{-\kappa(t-u)}dW(u),
    % \end{equation*}
    % where $\kappa$ and $\sigma$ are positive constants and:
    % \begin{equation*}
    %     \alpha (t) = f^M (0,t) + \frac{\sigma ^2}{2\kappa^2}(1- e^{-\kappa t})^2,
    % \end{equation*}
    % Compute conditional expectation under the risk neutral measure $\mathbb{E}^\mathbb{Q}[r(t) \vert r(s)]$ and  variance of the process ($r(t)$ conditional on $r(s)$ with $t>s$). Which is the statistical distribution of the process? 
    \part Explain the workflow for CVA and DVA calculation using Monte Carlo simulation. 
\end{parts}

\begin{solution}
\begin{parts}
    \part If the default of a counterparty happens after the final payment of derivative $T$, the value at time $t$ is simply $$1_{\tau > T}V(t,T)$$.
 If the default occurs before the maturity time $\tau < T$:
\begin{enumerate}
    \item We receive/pay all the payments until the default time: $1_{\tau \leq T}V(t, \tau)$;
    \item Depending on the counterparty, we may be able to recover some of the future payments, assuming the recovery fraction to be $R$ the value yields: $1_{\tau \leq T}R \max(V(\tau;T), 0)$;
    \item On the other hand, if we owe the money to the counterparty that has defaulted we cannot keep the money but we need to pay it completely back: $1_{\tau \leq T} \min(V(\tau;T), 0)$.
\end{enumerate}
Thus, when including all the components, a price of a \textit{risky} derivative is given by:
\begin{align*}
V_D(t_0, T) = \mathbb{E}^Q& \big[ 1_{\tau > T}V(t_0,T) +
1_{\tau \leq T}V(t_0, \tau) \\
&+ D(t_0, \tau) \, 1_{\tau \leq T}R \max(V(\tau,T), 0) \\
&+ D(t_0, \tau) \, 1_{\tau \leq T} \min(V(\tau, T), 0) \,|\, \mathcal{F}_t \big]
\end{align*}
Since \( x = \max(x, 0) + \min(x, 0) \), the simplified equation reads:
\begin{align*}
V_D(t_0, T) = \mathbb{E}^Q& \big[ 1_{\tau > T}V(t_0,T) +
1_{\tau \leq T}V(t_0, \tau) \\
&+ D(t_0, \tau) \, 1_{\tau \leq T} V(\tau;T)\\
&+ D(t_0, \tau) \, 1_{\tau \leq T}(R-1) \max(V(\tau;T), 0) \,|\, \mathcal{F}_t \big]
\end{align*}

We immediately note that the first three terms in the expression above yield:
\begin{align*}
    &\mathbb{E}^Q \big[ 1_{\tau > T}V(t_0,T) +
1_{\tau \leq T}V(t_0, \tau) + D(t_0, \tau)1_{\tau \leq T} V(\tau,T) \big] \\
    &= \mathbb{E}^Q \big[ 1_{\tau > T}V(t_0,T) +
1_{\tau \leq T}V(t_0, T) \big] \\
    &= V(t_0).
\end{align*}
The value of the risky derivative $V_D(t)$ is:
\begin{align*}
    V_D(t_0) &=  V(t_0) + \mathbb{E}^Q \left[1_{ \tau \leq T} \, (\text{R}(\tau) -1) D(t, \tau) \,  V(\tau)^+ \,|\, \mathcal{F}_t \right] \\
    &= V(t_0) - \mathbb{E}^Q \left[1_{ \tau \leq T} \, \text{LGD}(\tau) D(t, \tau) \,  V(\tau)^+ \,|\, \mathcal{F}_t \right]  \\
    &= V(t_0) - \text{uCVA}(t_0).
\end{align*}
\part
\begin{itemize}
    \item part1, definition of the contract and risk factor involved for pricing
    \item part2, simulation of the risk factors necessary for p1, as we need future paths. Need to define model, time grid discretization.
    \item part3, price the derivative across all paths and timesteps
    \item  part4, compute exposures profiles.
    \item part5, bootstrap credit and funding curve.
    \item part6, aggregate exposures and compute xVA.
\end{itemize}
%     \item[b)] Starting from the uCVA equation:
%         \begin{align*}
%             \text{uCVA}(t) = 1_{\tau > t} \lim_{n \to \infty} \sum_{i=1}^n \mathbb{E}^Q \left[  \left( e^{-\int_{t}^{t_{i-1}} \lambda(s)  ds} - e^{-\int_{t}^{t_i} \lambda(s) ds} \right)  \text{LGD}(t_{i-1}) \cdot D(t, t_{i-1}) \cdot \text{PV}^+(t_{i-1}) \right]. 
%     \end{align*}
%     Assuming : 
%     \begin{itemize}
%         \item finite number of timesteps N,
%         \item constant loss given default,
%         \item independence between default rates and interest rates,
%         \item  deterministic hazard rates in $t_0$.
%     \end{itemize} 
%     \begin{align*}
%         \text{uCVA}_{sw}(t_0) &=  \text{LGD} \sum_{i=1}^N   
%         \left( e^{-\int_{t}^{t_{i-1}} \lambda(s)  ds} - e^{-\int_{t}^{t_i} \lambda(s) ds} \right)
%         \mathbb{E}^Q \left[   D(t, t_{i-1})  \text{PV}^+(t_{i-1}) \right] 
%     \end{align*}
%     with $ \mathbb{E}^Q \left[   D(t, t_{i-1})  \text{PV}^+(t_{i-1}) \right] $ being the swaption part.
\end{parts}
\end{solution}

\end{questions}
\end{document}
