\documentclass{beamer}
\usetheme{afm}

\title{Arbitrage-Free Pricing Theory}
\subtitle{Let's refresh some useful concept}
\course{Advanced Financial Modelling}
\author{\href{mailto:matteo.sani@unisi.it}{Matteo Sani}}

\begin{document}
	\begin{frame}[plain]
		\maketitle
	\end{frame}        
	
\begin{frame}{Few Definitions}
		It may be helpful to explain (and recall) some of the more technical terms we are going to use.\newline
		
		\textbf{Sample space}: all possible future states or outcomes ($\Omega$) of a random process.\newline
		
		\textbf{(Probability) Measure} ($\mathbb{P}, \mathbb{Q}\ldots$): is a mapping which associates a probability to each element in the sample space. Two measures are \textbf{equivalent} if they agree "on what is possible". Note the word \emph{possible}: the two measures can have different probabilities for the same event, but must have the same \emph{null-set} $\{x\in {\mathbb{P}}\mid p (x)=0\}$. 
	\end{frame}

\begin{frame}{Few Definitions}
		\textbf{Contingent claim}: is a contract whose future payoff depends on the value of another “underlying” asset, or more generally, that is dependent on the realization of some uncertain future event $(S, X\ldots)$
		
		\begin{columns}
			\column{0.58\textwidth}
			\textbf{Filtrations}: are totally ordered collections of subsets that are used to model the information that is available at a given point in time ($\mathcal{F}_t$). 
			\column{0.35\textwidth}
				\includegraphics[width=0.8\linewidth]{filtration}
		\end{columns}
		
		\textbf{Martingale}: is a stochastic process for which, at a particular time, the conditional expectation of the next value in the sequence is equal to the present value, regardless of all prior values. It can be imagined as a \emph{drift-less} process
		\begin{equation*}
			dX = \cancel{\mu dt} + \sigma dW
		\end{equation*}
	\end{frame}

	\begin{frame}{Risk Neutral Pricing Foundations}
	Harrison and Pliska proved and formalized the following results:
	\begin{itemize}
		\item \textbf{The market is free of arbitrage if (and only if) there exists an \textcolor{red}{equivalent martingale measure} (EMM) (i.e. a risk-neutral measure).}
	\end{itemize}
	\vfill
	\end{frame}

	\begin{frame}{Risk Neutral Pricing Foundations}
	Harrison and Pliska proved and formalized the following results:
	\begin{itemize}
		\item The market is free of arbitrage if (and only if) there exists an \textcolor{red}{equivalent martingale measure} (EMM) (i.e. a risk-neutral measure).
		\item \textbf{The market is complete if and only if the martingale measure is unique.}
	\end{itemize}
	\vfill
\end{frame}

%\begin{frame}{Summary of Basic Definitions}
%	\begin{itemize}
%		\item The market is complete if and only if the martingale measure is unique.
%		\item In a complete and arbitrage-free market the price of any derivative is uniquely given, either by the value of the associated replicating strategy, or by the expectation of the discounted payoff under the risk-neutral measure
%		\begin{equation}
%			\Pi_t = \mathbb{E}^{\mathcal{Q}^0}[D(t,T)V_A|\mathcal{F}_t]
%			\label{eq:risk_neutral_pricing}
%		\end{equation}
%	\end{itemize}
%\end{frame}


\begin{frame}{Risk Neutral Pricing Foundations}
\begin{itemize}
	\item Arbitrage opportunities rarely exist in practice. If and when they do, gains are extremely small, and are typically short-lived and difficult to spot. \textcolor{red}{Arbitrage exclusion in the mathematical model is close enough to reality}.
	\item A martingale corresponds to the common notion that "a price, changes randomly" so we cannot know if it will go up or down. That is why this mathematical concept is brought into finance.
	\item An \textcolor{red}{equivalent martingale measure} $\mathbb{Q}$ is a probability measure on the space $\Omega$ such that
	\begin{enumerate}
		\item $\mathbb{Q}$ is equivalent to $\mathbb{P}$;
		\item for any asset $A$ and for each time $t$, $0\le t\le T$ there exists a price $\pi_t$
		\begin{equation*}
			\pi_t = \expect{Q^0}[D(t,T)V_A|\mathcal{F}_t]
		\end{equation*}
		which is a $\mathbb{Q}$-martingale.
	\end{enumerate}
\end{itemize}
\end{frame}

%\begin{frame}{Martingale}
%	\begin{block}{Definition}
%		A \textcolor{red}{$\mathcal{F}_t$-martingale} is a (integrable and adapted) stochastic process which models a fair game with the following remarkable feature
%		\begin{equation}
%			\mathbb{E}[X_t|\mathcal{F}_s] = X_s
%		\end{equation}
%		so the best prediction for the future value $X_t$, given the knowledge $\mathcal{F}_s$ at time $s$ is the value at time $s$ itself, $X_s$.
%	\end{block}
%	%	\begin{block}{Properties}
%		\begin{itemize}
%			\item If $X_t$ is a stochastic process with diffusion coefficient $\sigma_t$, such that %which satisfies $\mathbb{E}\left[\left(\int_0^T\sigma^2_s ds\right)^{\frac{1}{2}}\right]<\infty$, and SDE 
%			$dX_t=\mu_t dt+\sigma_t dW_t$, then 
%			\begin{equation*}
%				X\text{ is a martingale } \iff X\text{ is drift-less } (\mu_t=0)
%			\end{equation*}
%		\end{itemize}	
%	\end{frame}

\begin{frame}{Equivalent Martingale Measure}
	\begin{block}{Definition}
		An \textcolor{red}{equivalent martingale measure} $\mathbb{Q}$ is a probability measure on the space $\Omega$ such that
		\begin{enumerate}
			\item $\mathbb{Q}$ is equivalent to $\mathbb{P}$;
			\item for any asset $A$ and for each time $t$, $0\le t\le T$ there exists a price $\pi_t$
			\begin{equation*}
				\pi_t = \expect{Q^0}[D(t,T)V_A|\mathcal{F}_t]
			\end{equation*}
			\item the "discounted asset price" is a $\mathbb{Q}$-martingale
			\begin{equation*}
				\pi_u = \expect{Q^0}[D(0,t)V_A(t)|\mathcal{F}_u], \quad\text{with }(t>u)
			\end{equation*}
		\end{enumerate}
	\end{block}
\end{frame}

\begin{frame}{Risk-Neutral Measure}
	\begin{itemize}
	\item<1->The \textcolor{red}{risk-neutral measure} is agreed upon by the market as a whole just as a consequence of no arbitrage assumption.
	In other words it is nothing more than an \emph{implied probability distribution}.
	\item<2-> Implied from observable prices of tradable instruments, and used to determine \textcolor{red}{objective fair prices} for an asset or financial instrument. Probabilities are assessed with the risk taken out of the equation, so it doesn’t play a factor in the anticipated outcome.
	\item<3-> By contrast, if you tried to estimate the anticipated value of a stock based on how likely it is to go up or down, considering unique factors or market conditions that influence that specific asset, you would be including risk into the equation and, thus, would be looking at \textcolor{red}{real or physical probability}.
	\end{itemize}
\end{frame}

\begin{frame}{Risk Neutral Pricing Foundations}
	Harrison and Pliska proved and formalized the following results:
	\begin{itemize}
		\item The market is free of arbitrage if (and only if) there exists an \textcolor{red}{equivalent martingale measure} (EMM) (i.e. a risk-neutral measure).
		\item the market is complete if and only if the martingale measure is unique;
		\item \textbf{In a complete and arbitrage-free market the price of any derivative is uniquely given, either by the value of the associated replicating strategy, or by the expectation of the discounted payoff under the risk-neutral measure}
		\begin{equation}
			\Pi_t = \expect{Q^0}[D(t,T)V_A|\mathcal{F}_t]
			\label{eq:risk_neutral_pricing}
		\end{equation}
	\end{itemize}
\end{frame}

\begin{frame}{What is the Monte Carlo Technique ?}
\begin{itemize}
	\item From stock prices to interest rates, in the financial world countless factors can influence outcomes, making predictions challenging. 
	\item \textbf{The Monte Carlo simulation technique} offers a powerful tool to navigate this uncertainty and gain valuable 'insights.
	\item It is a probability-based simulation method which exploits random sampling to analyse uncertain outcomes.
	\item Instead of relying on single-point estimates, \emph{it generates multiple scenarios based on probability distributions}. 
	\item This allows us to assess the range of potential outcomes, not just a single average, providing a more comprehensive understanding of the problem in hand.
\end{itemize}
\end{frame}

\begin{frame}{How Does it Work ?}
\begin{enumerate}
	\item Identify the uncertain variables (e.g. stock prices, default probabilities\ldots);
	\item assign probability distributions to each, reflecting their potential movements;
	\item randomly sample from these distributions, simulate different scenarios;
	\item repeat this process hundreds, (or even thousands) of times: you will get a diverse landscape of potential outcomes, ready for analysis.
\end{enumerate}
\vspace{0.25cm}
\begin{columns}
	\column{0.4\linewidth}
	\hfill
	\includegraphics[width=0.8\linewidth]{central_limit_theorem}
	\column{0.6\linewidth}
	The \emph{Central Limit Theorem} tells us:
	\begin{enumerate}
		\item the best estimate of a quantity given by MC experiments is the \textbf{mean} of the simulation results;
		\item with a larger sample size, your sample mean is \textbf{more likely to be close to the real population mean}. In other words, your estimate is more precise.
	\end{enumerate}
\end{columns}
\begin{tikzpicture}[remember picture,overlay]
\node[xshift=6.cm,yshift=-4.cm] (image) at (current page.center) {\includegraphics[width=80px]{python}};
\end{tikzpicture}
\end{frame}

\end{document}
