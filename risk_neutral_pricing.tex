\documentclass{beamer}
\usetheme{afm}

\title{Arbitrage-Free Pricing Theory}
%\subtitle{Basic Definitions and a Bit of Stochastic Calculus}
\course{Advanced Financial Modelling}
\author{\href{mailto:matteo.sani@unisi.it}{Matteo Sani}}

\begin{document}
	\begin{frame}[plain]
		\maketitle
	\end{frame}        
	
\begin{frame}{Few Definitions}
		It may be helpful to explain (and recall) some of the more technical terms we are going to use.\newline
		
		\textbf{Sample space}: all possible future states or outcomes ($\Omega$) of a random process.\newline
		
		\textbf{(Probability) Measure} ($\mathcal{P}, \mathcal{Q}\ldots$): is a mapping which associates a probability to each element in the sample space. Two measures are \textbf{equivalent} if they agree "on what is possible". Note the word \emph{possible}: the two measures can have different probabilities for the same event, but must have the same \emph{null-set} $\{x\in {\mathcal {P}}\mid p (x)=0\}$. 
	\end{frame}

\begin{frame}{Few Definitions}
		\textbf{Contingent claim}: is a derivative whose future payoff depends on the value of another “underlying” asset, or more generally, that is dependent on the realization of some uncertain future event $(S, X\ldots)$.\newline
		
		\textbf{Filtrations}: are totally ordered collections of subsets that are used to model the information that is available at a given point in time ($\mathcal{F}_t$). \newline
		
		\textbf{Martingale}: is a stochastic process for which, at a particular time, the conditional expectation of the next value in the sequence is equal to the present value, regardless of all prior values. It can be imagined as a drift-less process.
	\end{frame}

	\begin{frame}{Risk Neutral Pricing Foundations}
	Harrison and Pliska proved and formalized the following results:
	\begin{itemize}
		\item \textbf{The market is free of arbitrage if (and only if) there exists an \textcolor{red}{equivalent martingale measure} (EMM) (i.e. a risk-neutral measure).}
	\end{itemize}
	\vfill
	\end{frame}

	\begin{frame}{Risk Neutral Pricing Foundations}
	Harrison and Pliska proved and formalized the following results:
	\begin{itemize}
		\item The market is free of arbitrage if (and only if) there exists an \textcolor{red}{equivalent martingale measure} (EMM) (i.e. a risk-neutral measure).
		\item \textbf{The market is complete if and only if the martingale measure is unique.}
	\end{itemize}
	\vfill
\end{frame}

%\begin{frame}{Summary of Basic Definitions}
%	\begin{itemize}
%		\item The market is complete if and only if the martingale measure is unique.
%		\item In a complete and arbitrage-free market the price of any derivative is uniquely given, either by the value of the associated replicating strategy, or by the expectation of the discounted payoff under the risk-neutral measure
%		\begin{equation}
%			\Pi_t = \mathbb{E}^{\mathcal{Q}^0}[D(t,T)V_A|\mathcal{F}_t]
%			\label{eq:risk_neutral_pricing}
%		\end{equation}
%	\end{itemize}
%\end{frame}


\begin{frame}{Risk Neutral Pricing Foundations}
\begin{itemize}
	\item Arbitrage opportunities rarely exist in practice. If and when they do, gains are extremely small, and are typically short-lived and difficult to spot. \textcolor{red}{Arbitrage exclusion in the mathematical model is close enough to reality}.
	\item A martingale corresponds to the common notion that "a price, changes randomly" so we cannot know if it will go up or down. That is why this mathematical concept is brought into finance.
	\item An \textcolor{red}{equivalent martingale measure} $\mathcal{Q}$ is a probability measure on the space $\Omega$ such that
	\begin{enumerate}
		\item $\mathcal{Q}$ is equivalent to $\mathcal{P}$;
		\item for any asset $A$ and for each time $t$, $0\le t\le T$ there exists a price $\pi_t$
		\begin{equation*}
			\pi_t = \mathbb{E}^{\mathcal{Q}^0}[D(t,T)V_A|\mathcal{F}_t]
		\end{equation*}
		which is a $\mathcal{Q}$-martingale.
	\end{enumerate}
\end{itemize}
\end{frame}

%\begin{frame}{Martingale}
%	\begin{block}{Definition}
%		A \textcolor{red}{$\mathcal{F}_t$-martingale} is a (integrable and adapted) stochastic process which models a fair game with the following remarkable feature
%		\begin{equation}
%			\mathbb{E}[X_t|\mathcal{F}_s] = X_s
%		\end{equation}
%		so the best prediction for the future value $X_t$, given the knowledge $\mathcal{F}_s$ at time $s$ is the value at time $s$ itself, $X_s$.
%	\end{block}
%	%	\begin{block}{Properties}
%		\begin{itemize}
%			\item If $X_t$ is a stochastic process with diffusion coefficient $\sigma_t$, such that %which satisfies $\mathbb{E}\left[\left(\int_0^T\sigma^2_s ds\right)^{\frac{1}{2}}\right]<\infty$, and SDE 
%			$dX_t=\mu_t dt+\sigma_t dW_t$, then 
%			\begin{equation*}
%				X\text{ is a martingale } \iff X\text{ is drift-less } (\mu_t=0)
%			\end{equation*}
%		\end{itemize}	
%	\end{frame}

\begin{frame}{Risk-Neutral Measure}
	\begin{itemize}
	\item<1->The \textcolor{red}{risk-neutral measure} is agreed upon by the market as a whole just as a consequence of no arbitrage assumption.
	In other words it is nothing more than an \emph{implied probability distribution}.
	\item<2-> Implied from observable prices of tradable instruments, and used to determine \textcolor{red}{objective fair prices} for an asset or financial instrument. Probabilities are assessed with the risk taken out of the equation, so it doesn’t play a factor in the anticipated outcome.
	\item<3-> By contrast, if you tried to estimate the anticipated value of a stock based on how likely it is to go up or down, considering unique factors or market conditions that influence that specific asset, you would be including risk into the equation and, thus, would be looking at \textcolor{red}{real or physical probability}.
	\end{itemize}
\end{frame}

\begin{frame}{Risk Neutral Pricing Foundations}
	Harrison and Pliska proved and formalized the following results:
	\begin{itemize}
		\item The market is free of arbitrage if (and only if) there exists an \textcolor{red}{equivalent martingale measure} (EMM) (i.e. a risk-neutral measure).
		\item the market is complete if and only if the martingale measure is unique;
		\item \textbf{In a complete and arbitrage-free market the price of any derivative is uniquely given, either by the value of the associated replicating strategy, or by the expectation of the discounted payoff under the risk-neutral measure}
		\begin{equation}
			\Pi_t = \mathbb{E}^{\mathcal{Q}^0}[D(t,T)V_A|\mathcal{F}_t]
			\label{eq:risk_neutral_pricing}
		\end{equation}
	\end{itemize}
\end{frame}

\begin{frame}{Risk Neutral Pricing Foundations}
	Harrison and Pliska proved and formalized the following results:
	\begin{itemize}
		\item The market is free of arbitrage if (and only if) there exists an \textcolor{red}{equivalent martingale measure} (EMM) (i.e. a risk-neutral measure).
		\item the market is complete if and only if the martingale measure is unique;
		\item \textbf{In a complete and arbitrage-free market the price of any derivative is uniquely given, either by the value of the associated replicating strategy, or by the expectation of the discounted payoff under the risk-neutral measure}
		\begin{equation*}
			\Pi_t = \textcolor{red}{\mathbb{E}^{\mathcal{Q}^0}[}D(t,T)V_A|\mathcal{F}_t\color{red}{]}
		\end{equation*}
		\item \small{\textcolor{red}{expectation $\implies$ Monte Carlo Simulation}}	
	\end{itemize}
\end{frame}

\begin{frame}{Risk Neutral Pricing Foundations}
	Harrison and Pliska proved and formalized the following results:
	\begin{itemize}
		\item The market is free of arbitrage if (and only if) there exists an \textcolor{red}{equivalent martingale measure} (EMM) (i.e. a risk-neutral measure).
		\item the market is complete if and only if the martingale measure is unique;
		\item \textbf{In a complete and arbitrage-free market the price of any derivative is uniquely given, either by the value of the associated replicating strategy, or by the expectation of the discounted payoff under the risk-neutral measure}
		\begin{equation*}
			\Pi_t = \textcolor{red}{\mathbb{E}^{\mathcal{Q}^0}[}D(t,T)\textcolor{blue}{V_A}|\mathcal{F}_t\color{red}{]}
		\end{equation*}
		\item \small{\textcolor{red}{expectation $\implies$ Monte Carlo Simulation}}
		\item \small{\textcolor{blue}{random variable, stochastic process $\implies$ dynamics evolution}}
	\end{itemize}
\end{frame}

\begin{frame}{Risk Neutral Pricing Foundations}
	Harrison and Pliska proved and formalized the following results:
	\begin{itemize}
		\item The market is free of arbitrage if (and only if) there exists an \textcolor{red}{equivalent martingale measure} (EMM) (i.e. a risk-neutral measure).
		\item the market is complete if and only if the martingale measure is unique;
		\item \textbf{In a complete and arbitrage-free market the price of any derivative is uniquely given, either by the value of the associated replicating strategy, or by the expectation of the discounted payoff under the risk-neutral measure}
		\begin{equation*}
			\Pi_t = 	\textcolor{red}{\mathbb{E}^{\mathcal{Q}^0}[}D(t,T)\textcolor{blue}{V_A}|\mathcal{F}_t\color{red}{]}
		\end{equation*}
		\item \small{\textcolor{red}{expectation $\implies$ Monte Carlo Simulation}}
		\item \small{\textcolor{blue}{random variable, stochastic process $\implies$ dynamics evolution}}
	\end{itemize}
	\begin{tikzpicture}[remember picture,overlay]
		\node[xshift=6.5cm,yshift=-3.2cm] (image) at (current page.center) {\includegraphics[width=20px]{python_logo}};
		\node[align = center, yshift=1.45cm, below=of image] {\tiny{\href{shorturl.at/knyGT}{shorturl.at/knyGT}}};
	\end{tikzpicture}
\end{frame}

%\subsection{Martingales}
%\begin{frame}{Filtration}
%	\begin{block}{Definition}
%		With the symbol $\mathcal{F}^X_t$ it is indicated a \textbf{filtration}. It represents the information generated by $X$ on the interval $[0, t]$, i.e. what has happened to $X$ over the interval. 
%	\end{block}
%	\begin{itemize}	
%		\item If the value of a stochastic variable $X$ ca be completely determined given observations of its trajectories $\{X(s); 0\leq s \leq t\}$, then we can write $X\in\mathcal{F}_t^X$ and $X$ is said to be $\mathcal{F}_t^X$\emph{-measurable}.
%		\item If $Y$ is a stochastic process such that $Y(t)\in\mathcal{F}_t^X$ for all $t$ then we say that $Y$ is \emph{adapted} to the filtration $\mathcal{F}_t^X$. 
%	\end{itemize}
%\end{frame}
%
%\begin{frame}{Conditional Expectation}
%	\begin{block}{Definition}
%		Given the information (filtration) $\mathcal{F}_t$, for any stochastic variable $X$ consider
%		\begin{equation*}
%			\mathbb{E}[X|\mathcal{F}_t]
%		\end{equation*}
%		which represents the \textbf{conditional expectation} of $X$.
%		By definition it also holds that $\mathbb{E}[\mathbb{1}_{\mathcal{F}}X] = \mathbb{E}[\mathbb{1}_{\mathcal{F}}\mathbb{E}[X|\mathcal{F}]]$.
%	\end{block}
%	\begin{itemize}
%		\item Given $X$ and $Y$ stochastic variables with $Y$ $\mathcal{F}_t$-measurable:
%		\begin{equation*}
%			\mathbb{E}[Y\cdot X|\mathcal{F}_t] =  Y\cdot\mathbb[X|\mathcal{F}_t]
%		\end{equation*}
%		indeed if $Y\in\mathcal{F}_t$ we know exactly its value, so in the expectation it can be treated as a constant and taken outside.
%		\item If $X$ is a stochastic variable and $s<t$ (\emph{law of iterated expectations}):
%		\begin{equation*}
%			\mathbb{E}[\mathbb{E}[X|\mathcal{F}_t]|\mathcal{F}_s] = \mathbb{E}[X|\mathcal{F}_s]
%		\end{equation*}
%	\end{itemize}
%\end{frame}

	

%\begin{frame}{Risk Neutral Measure}
%	\begin{itemize}
%		\item<1-> Prices of assets depend crucially on their risk as investors typically demand more profit for bearing more risk.
%		\item<2-> Therefore, today's price of a claim on a risky amount realized tomorrow will generally differ from its expected value.
%		\item<2-> Most commonly, investors are risk-averse and today's price is below the expectation, remunerating those who bear the risk.
%		\item<4-> Consequently to price assets, the calculated expected values need to be adjusted for an investor's risk preferences
%		\item<5-> Unfortunately, these adjustments would vary between investors and an individual's risk preference is very difficult to quantify.
%		\item<6-> \textcolor{red}{It turns out that, under few assumptions, there is an alternative way to do this calculation.}
%	\end{itemize}
%\end{frame}

%\subsection{Fundamental Theorems of Arbitrage Pricing}

%\begin{frame}{Hedging}
%	\begin{itemize}
%		\item A portfolio $\mathbf{\theta}$ in the assets $A$ is a \textcolor{red}{replicating portfolio} for a contingent claim $X$ if
%		\begin{equation}
%			V_t = \sum_{j=1}^K \theta_j S_t^j(\omega_i)\quad\forall i=1,2,\ldots,N
%		\end{equation}
%		\item If all claims (assets) can be replicated the market is said to be \textbf{complete}.
%		\item From a financial point of view, there is no difference between holding the claim and holding the portfolio, no matter what happens on the stock market, the value of the claim and of the stock will be the same.
%	\end{itemize}
%\end{frame}

	
%\begin{frame}{Real World Measure $\mathcal{P}$}
%	\begin{itemize}
%	\item<1-> When we model derivative prices, we take as given some "true" probability measure $\mathcal{P}$, which assigns probabilities to different states of the world. 
%	\item<2-> These states in turn affect the path of security prices. 
%	\item<3-> And these states plus the corresponding probabilities are supposed to reflect the subjective beliefs of traders or investors about what will happen in the future.
%	\item<4-> Unfortunately, under $\mathcal{P}$ it is usually quite complicated to price derivatives, and the probabilities themselves cannot easily be derived.
%	\item<5-> This makes it hard to work out the price processes and it is necessary to use simulation techniques.
%	\end{itemize}
%\end{frame}
%	
%	\begin{frame}{Changing Measure}
%		\begin{itemize}
%			\item<1-> Then it makes sense to look for a new set of probabilities (measure) which could simplify the pricing process giving the correct result at the same time.
%			%(Actually, no-arbitrage constructions or the Feynman-Kac formula will give you an explicit PDE whose solution is $C_t$, which will not generally be analytical solvable.)
%			\item<2-> That's why \textcolor{red}{changes of probability measure are important in mathematical finance}: allow to express derivative prices in closed-form.
%			\item<3-> At the beginning of the course we have seen how by simply assuming the absence of arbitrage it is possible to define a \textcolor{red}{new measure} under which the price of a derivative is given by the discounted expectation of its payoff.
%			\item<4-> This result has been formalized by \emph{Harrison} and \emph{Pliska} in 1981. 
%		\end{itemize}
%	\end{frame}
	

%\texttt{
%\begin{frame}{Portfolio}
%	\begin{block}{Definition}
%		A \textcolor{red}{portfolio} is a vector $\mathbf{\theta}\in \mathbb{R}^K$ whose $j$ components represent the number of shares of asset $A_j$ (asset $A_0$ is risk-free). It's value is
%		\begin{equation}
%			V_t(\mathbf{\theta}, \omega)=\sum_{j=1}^K\theta_jS^j_t(\omega)
%		\end{equation} 
%		where $S_t^j$ is the value of $j$-th asset, and $\omega$ a market situation. A portfolio is said to be \textcolor{red}{self-financing} if its value changes only due to variations of the asset prices.
%	\end{block}
%\end{frame}
%
%\begin{frame}{Arbitrage}
%	\begin{block}{Definition}
%		An \textcolor{red}{arbitrage} is a self financing portfolio $\mathbf{\theta}$ such that
%		\begin{equation}
%			\begin{cases}
%				%V_0(\mathbf{\theta}, \omega)\le 0 \text{ and } \mathcal{P}\{V_t(\mathbf{\theta}, \omega) > 0\} > 0
%				V_0 = 0 \\
%				P(V_{t}\geq 0)=1\text{ and }P(V_{t}\neq 0)>0,\,0<t\leq T
%			\end{cases}
%		\end{equation}
%		where $V_t$ denotes the portfolio value at time $t$ and $T$ is the time the portfolio ceases to be available on the market. 
%		\emph{This means that the value of the portfolio is never negative, and guaranteed to be positive at least once over its lifetime.}
%	\end{block}
%\end{frame}
%
%\begin{frame}{One Period Model}
%	Consider a bank-account $B(t)=e^{rt}$ ($r$ denote the risk-free rate) and assume today's stock price to be $S_0$. In one period of time from now, the price could be 
%	\begin{equation*}
%		\begin{cases}
%			S_0\cdot u = S_u \quad\text{with a certain probability $p_u$} \\
%			S_0\cdot d = S_d \quad\text{with a certain probability $p_d$}\\ 
%		\end{cases}, \text{with }(u > d)
%	\end{equation*}
%	\pause
%	If we want our simple model to \emph{avoid arbitrage opportunities}, we must impose conditions on $u$ and $d$. 
%	\pause
%	
%	In case $e^r > u$, I could short the stock in $t_0$ and invest the proceeds $S_0$ into the bank account: in both future states in $t_1$, I could buy the stock back for less than my proceeds 
%	\begin{equation*}
%		S_0e^r > S_u > S_d
%	\end{equation*} Similarly for $e^r < d$\ldots
%	\pause
%	\begin{equation*}
%		\boxed{d\le e^r \le u \implies \text{no arbitrage}}
%	\end{equation*}
%\end{frame}
%
%
%\begin{frame}{Risk-Neutral Measure}
%	\begin{equation*}
%		\begin{aligned}
%			S_0 &= \frac{S_0(u-d)e^r}{(u-d)e^r} = \frac{S_0(u-d)e^r + (S_0ud - S_0ud)}{(u-d)e^r}=\\
%			&= \frac{1}{e^r}\left(\frac{S_0ue^r - S_0ud}{u-d} + \frac{-S_0de^r + S_0ud}{u-d}\right)=\\
%			&= \frac{1}{e^r}\left(S_0u\frac{e^r - d}{u-d} + S_0d\frac{u - e^r}{u-d}\right)
%		\end{aligned}
%	\end{equation*}
%	\pause
%	The no arbitrage condition implies the following bounds
%	\begin{equation*}
%		\boxed{0\le\frac{e^r -d}{u-d}\le 1,\quad 0\le\frac{u - e^r}{u-d}\le 1}
%	\end{equation*}
%	\pause
%	also
%	\begin{equation}
%		\boxed{\frac{e^r -d}{u-d} + \frac{u - e^r}{u-d} = 1}
%		\label{eq:risk_neutral_probabilities}
%	\end{equation}
%\end{frame}
%
%\begin{frame}{Risk-Neutral Measure}
%	So we can interpret $p_u=\cfrac{e^r -d}{u-d}$ and $p_d=\cfrac{u - e^r}{u-d}$ as a \textcolor{red}{(risk-neutral) measure} ($\mathcal{Q}$).\vspace{0.3cm}
%	
%	\pause
%	\begin{block}{Definition}
%		A \textcolor{red}{probability measure} is a real-valued function that assigns probabilities to a set of events in a sample space that satisfies measure properties such as countable additivity, and assigning value 1 to the entire space.
%	\end{block}	
%	\pause
%	Rewriting previous expression of $S_0$ in terms of the newly defined probabilities
%	\begin{equation}
%		S_0 = \frac{S_up_u + S_dp_d}{e^r} = e^{-r}\mathbb{E}^\mathcal{Q}[S_1]
%		\label{eq:risk_neutral_price}
%	\end{equation}
%	
%	So the stock price is the discounted stock expectation \emph{under the chosen probability measure} at $t_1$.
%\end{frame}
%
%\subsection{Fundamental Theorems of Arbitrage Pricing}
%\begin{frame}{Fundamental Theorems of Arbitrage Pricing}
%	\begin{block}{Theorem I}
%		There exists a \emph{risk-neutral measure} if and only \textcolor{red}{if arbitrages do not exist}.
%	\end{block}
%	\pause
%	Proof:
%	\begin{itemize}
%		\item \textbf{no-arbitrage$\rightarrow$risk-neutral measure}: requiring the model to be arbitrage-free sets conditions on $u, d$ and $e^r$ such that $p_u$ and $p_d$ define a probability measure (risk-neutral measure).
%		\item \textbf{risk-neutral measure$\rightarrow$no-arbitrage}: consider an arbitrary portfolio $\theta$ and check that given the assumptions must be arbitrage-free
%		\begin{equation*}
%			V_0 = xB_0 + yS_0 = 0
%		\end{equation*}
%		This yields $x = -yS_0$.
%	\end{itemize}
%\end{frame}
%
%\begin{frame}{Fundamental Theorems of Arbitrage Pricing}
%	Proof continued:
%	\begin{itemize}
%		\item At $t=1$ we have 
%		\begin{equation*}
%			V_1 = xB_1 + yS_1 = -yS_0e^r + yS_0Z = yS_0(Z - e^r)\quad\text{with $Z$=\{u, d\}}
%		\end{equation*}
%		To make $\theta$ an arbitrage opportunity must be $V_1\geq 0$.
%		
%		Imagine our portfolio has $y > 0$ then there is arbitrage if and only if $u - e^r \geq 0$ and $d - e^r \geq 0$
%		which can not happen according to the assumptions. 
%		
%		Thus, if $y > 0$ \emph{the portfolio is not an arbitrage one}. 
%		\item The case $y < 0$ is treated in the same way. 
%	\end{itemize}
%	\pause
%	\vspace{0.5cm}
%	But what does it mean "risk-neutral measure" ?
%\end{frame}
%
%%\begin{frame}{Few Definitions}
%%	TOGLIERE
%%	\begin{block}{Definition}
%	%		A \textcolor{red}{numeraire} is any positive non-dividend-paying asset. It is a reference asset chosen to normalize all other asset prices to it. Having a numeraire allows for the comparison of the value of goods against one another.
%	%	\end{block}
%%    METTERE NELLE SLIDE DOPO
%%    \begin{block}{Definition}
%	%		A \textcolor{red}{probability measure} is a real-valued function that assignes probabilities to a set of events in a sample space that satisfies measure properties such as countable additivity, and assigning value 1 to the entire space.
%	%	\end{block}	
%%\end{frame}
%
%\begin{frame}{Risk-Neutral Measure}
%	%We have never talked about the probabilities of the stock going up or down; every investor might have her view of the world with different probabilities assigned to the stock. 
%	
%	The \textcolor{red}{risk-neutral measure} is agreed upon by the market as a whole just as a consequence of no arbitrage assumption.
%	In other words it is nothing more than an \emph{implied probability distribution}.
%	\pause
%	
%	Implied from observable prices of tradable instruments, and used to determine \textcolor{red}{objective fair prices} for an asset or financial instrument. Probabilities are assessed with the risk taken out of the equation, so it doesn’t play a factor in the anticipated outcome.
%	\pause
%	
%	By contrast, if you tried to estimate the anticipated value of a stock based on how likely it is to go up or down, considering unique factors or market conditions that influence that specific asset, you would be including risk into the equation and, thus, would be looking at \textcolor{red}{real or physical probability}.
%\end{frame}
%
%\begin{frame}{Contingent Claim}
%	\begin{block}{Definition}
%		A \emph{contingent claim} (financial derivative) is any stochastic variable $X$ of the form $X=\Phi(Z)$ , where $Z$ is a stochastic process driving the stock price process above. 
%		The function $\Phi$ is called the \emph{contract function}.
%	\end{block}
%	As an example consider an European call option on the stock $S$ with strike $K$, (assume $S_0d < K < S_0u$)
%	\begin{equation*}
%		X = \begin{cases}
%			\Phi(u) = S_0 u - K\\
%			\Phi(d) = 0
%		\end{cases}
%	\end{equation*}
%	
%	We aim at determine the "fair" price $\Pi(t; X)$, if it exists, for a given contingent claim $X$.
%\end{frame}
%
%\begin{frame}{Hedging}
%	\begin{itemize}
%		\item A portfolio $\mathbf{\theta}$ in the assets $A$ is a \textcolor{red}{replicating portfolio} for a contingent claim $X$ if
%		\begin{equation}
%			V_t = \sum_{j=1}^K \theta_j S_t^j(\omega_i)\quad\forall i=1,2,\ldots,N
%		\end{equation}
%		\item If all claims (assets) can be replicated the market is said to be \textbf{complete}.
%		\item From a financial point of view, there is no difference between holding the claim and holding the portfolio, no matter what happens on the stock market, the value of the claim and of the stock will be the same.
%	\end{itemize}
%\end{frame}
%
%\begin{frame}{Fundamental Theorems of Arbitrage Pricing}
%	\begin{block}{Theorem II}
%		%Suppose exists a replicating portfolio for a claim $X$ then  
%		Let $\mathcal{M}$ be an arbitrage-free market with a risk-less asset. If for every derivative there is a replicating portfolio %in the assets $A_j$ 
%		then the market $\mathcal{M}$ is complete. Conversely, if the market $\mathcal{M}$ is complete, and if the unique risk-neutral measure $\mathcal{Q}$ gives positive probability to every market scenario $\omega$, then for every derivative security there is a replicating portfolio.% in the assets $A_j$.
%	\end{block}
%	Proof:
%	\begin{itemize}
%		\item \textbf{complete market$\rightarrow$no-arbitrage}: consider the replicating portfolio $\theta$ of a contingent claim $X$. If $\Pi(0; X) < V_0^\theta$ then at time $t=0$ we buy the claim, sell the portfolio and put the proceeds $V_0^\theta-\Pi_0$ in the bank so the net position is 0. At $t=1$ we will receive the claim payoff $X$ and will have to pay $V_1^\theta =X$  to the holder of the portfolio. This cancels but we still have $e^r(V_0^\theta-\Pi_0)>0$ in the bank. Thus we have an arbitrage. Similary $\Pi(0; X) > V_0^\theta$.
%	\end{itemize}
%\end{frame}
%
%\begin{frame}{Fundamental Theorems of Arbitrage Pricing}
%	Proof continues:
%	\begin{itemize}
%		\item \textbf{no-arbitrage$\rightarrow$complete market}: fix and arbitrary claim $X$ with contract function $\Phi$ and a portfolio such that
%		\begin{equation*}
%			\begin{cases}
%				V_1 = x e^r + yS_0u = \Phi(u)\\
%				V_1 = x e^r + yS_0d = \Phi(d)
%			\end{cases}
%		\end{equation*}
%		
%		Since our model assume $d<u$ there exists a \textbf{unique} solution
%		\begin{equation}
%			\begin{aligned}
%				x &= e^{-r}\cfrac{u\Phi(d)-d\Phi(d)}{u-d} \\
%				y &= \cfrac{1}{S_0}\cfrac{\Phi(u)-\Phi(d)}{u-d}
%			\end{aligned}
%			\label{eq:replica_portfolio_value}
%		\end{equation}
%	\end{itemize}
%\end{frame}
%
%\begin{frame}{Risk-Neutral Measure and Pricing}
%	\begin{block}{Proposition}
%		Under the conditions which makes our model arbitrage-free, there exists a martingale (risk-neutral) measure and the free arbitrage price of a contingent claim $X$ is given by 
%		\begin{equation*}
%			\Pi(0; X) = e^{-r}\mathbb{E}^{\mathcal{Q}}[X]
%		\end{equation*} 
%		The martingale measure $\mathcal{Q}$ is uniquely identified by \cref{eq:risk_neutral_probabilities}.
%	\end{block}
%	%Assume there exists a \textcolor{red}{risk-neutral measure} $\mathcal{Q}^0$ %on the set $\Omega$ of possible market scenarios 
%	%and let $A$ be an asset. Then, for each time $t$, $0\le t\le T$ there exists a unique price $\pi_t$ associated with $A$
%	%\begin{equation}
%	%	\pi_t = \mathbb{E}^{\mathcal{Q}^0}[D(t,T)V_A|\mathcal{F}_t]
%	%	\label{eq:risk_neutral_pricing}
%	%\end{equation}
%	
%	Indeed, the market is complete, so from the replicating portfolio we know that $\Pi(0;X) = V_0$ and using 	\cref{eq:risk_neutral_probabilities} and \cref{eq:replica_portfolio_value} we can explicitly calculate the price as
%	\begin{equation*}
%		\Pi(0;X) = e^{-r}\left[\cfrac{e^r-d}{u-d}\Phi(u)+\cfrac{u-e^r}{u-d}\Phi(d)\right] = e^{-r}\left[q_u\Phi(u)+q_d\Phi(d)\right]
%	\end{equation*}
%	the expected value, under the martingale measure.
%	Such a price is given by the expectation of the discounted payoff under the measure $\mathcal{Q}^0$.
%	
%	%Note that $\mathcal{F}_t$ is called \textcolor{red}{filtration} and represents our knowledge of the system up to time $t$, i.e. the expectation is indeed \emph{conditioned} to what happened until time $t$.
%	%The benefit of this risk-neutral pricing approach is that once the risk-neutral probabilities are calculated, they can be used to price every asset based on its expected payoff. These theoretical risk-neutral probabilities differ from actual real-world probabilities, which are sometimes also referred to as physical probabilities. If real-world probabilities were used, the expected values of each security would need to be adjusted for its individual risk profile.
%	
%	%You might think of this approach as a structured method of guessing what the fair and proper price for a financial asset should be by tracking price trends for other similar assets and then estimating the average to arrive at your best guess. 
%	
%	%For this approach, you would try to level out the extreme fluctuations at either end of the spectrum, creating a balance that creates a stable, level price point. You would essentially be minimizing the possible unusual high market outcomes while increasing the possible lows.
%\end{frame}
%
%%\begin{frame}{Risk-Neutral Measure and Pricing}
%%	\begin{itemize}
%	%		\item Later in this course, in the context of the change of measure, we are going to formalize the previous slide statement.
%	%		\item In summary, we will show a generalization of the original ideas of Black and Scholes, showing that, under complete markets with no arbitrage, it is possible to use for pricing purposes (only) stochastic models that do not factor in the risk premium.
%	%		\item \textbf{Example:} imagine an asset such that $S_0=100$ and that $S_u=120$ and $S_d=80$. If the risk-free rate is 5\% the risk-neutral probability is 
%	%		\begin{equation*}
%		%			q = \frac{e^r - d}{u-d} \approx 63\%
%		%		\end{equation*}
%	%	\end{itemize}
%%	\begin{tikzpicture}[remember picture,overlay]
%	%	\node[xshift=5cm,yshift=-3.7cm] (image) at (current page.center) {\includegraphics[width=20px]{python_logo}};
%	%	\node[align = center, yshift=1.45cm, below=of image] {\tiny{\href{shorturl.at/htCFJ}{shorturl.at/htCFJ}}};
%	%	\end{tikzpicture}
%%\end{frame}
%
%
%
%%\begin{frame}{Hedging}
%%	\begin{itemize}
%	%		\item A portfolio $\mathbf{\theta}$ in the assets $A$ is a \textcolor{red}{replicating portfolio} for the asset $B$ if
%	%		\begin{equation}
%		%			S_t^{B}(\omega_i) = \sum_{j=1}^K \theta_j S_t^j(\omega_i)\quad\forall i=1,2,\ldots,N
%		%		\end{equation}
%	%		\item In particular it can be demonstrated if the market is \emph{arbitrage-free} then the relation holds for all $t$.
%	%		\item The importance of replicating portfolios is that they enable financial institutions that sell
%	%		asset $B$ (e.g. a call options) to \textcolor{red}{hedge}: for each sold share of asset $B$, buy $\theta_j$ shares
%	%		of asset $A_j$ and hold them to time $t + 1$. Then at time $t + 1$, 
%	%		\begin{equation*}
%		%			\text{net gain }= \text{ net loss } = 0
%		%		\end{equation*}
%	%	\end{itemize}
%%\end{frame}
%%
%%\begin{frame}{Fundamental Theorems of Arbitrage Pricing}
%%	\begin{itemize}
%	%		\item In some circumstances, an arbitrage-free market may admit more than one risk-neutral measure, i.e. \textcolor{red}{incomplete markets}.
%	%		\item By contrast, a \textcolor{red}{complete market} is one that has a unique risk-neutral measure.
%	%	\end{itemize}
%%	\pause
%%	\begin{block}{Theorem II}
%	%		Let $\mathcal{M}$ be an arbitrage-free market with a risk-less asset. If for every derivative security there is a replicating portfolio in the assets $A_j$ then the market $\mathcal{M}$ is complete. Conversely, if the market $\mathcal{M}$ is complete, and if the unique risk-neutral measure $\mathcal{Q}$ gives positive probability to every market scenario $\omega$, then for every
%	%		derivative security there is a replicating portfolio in the assets $A_j$.
%	%	\end{block}
%%\end{frame}
%}


\end{document}
