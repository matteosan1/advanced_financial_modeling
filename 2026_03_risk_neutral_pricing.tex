\documentclass{beamer}
\usetheme{afm}

\title{Change of Measure and Its Applications}
\course{Advanced Financial Modeling}
\author{\href{mailto:matteo.sani@unisi.it}{Matteo Sani}}

\begin{document}
	\begin{frame}[plain]
		\maketitle
	\end{frame}
	
	\section{Change of Measure}
	\subsection{Numeraires}
	\begin{frame}{Numeraires}
		\begin{itemize}
			\item<1-> Since the main issue in pricing theory is to compute
			\begin{equation}
				\Pi_t = \E^{\mathbb{Q}^X}\left[\frac{1}{X_T}V_A|\mathcal{F}_t\right]
				\label{eq:risk_neutral_pricing}
			\end{equation}
			ideally we would like to define a probability measure $\mathbb{Q}^X$ under which the discounted derivative process is a martingale and such that the expectation of the payoff is analytically tractable (i.e. easy to compute).
			\item<1-> We determine such measure $\mathbb{Q}^X$ with a \textcolor{maincolor}{change of numeraire}.
			\item<3-> A \textcolor{maincolor}{numeraire} is any strictly positive stochastic process $N_t$ that is taken as unit of reference when pricing an asset $S_t$
			\begin{equation*}
				\tilde{S_t}:=\frac{S_t}{N_t}, \quad t \ge 0
			\end{equation*}
		\end{itemize}
	\end{frame}
	
	\begin{frame}{How to Choose a Numerair}
		\begin{itemize}
			\item We may compute asset values w.r.t. USD, EUR or JPY. Others might prefer commodities: 1~oz of gold could be a numeraire.
			\item In any case, once we choose a numeraire, all the other asset values are determined.
			\item Practical reasons induce to prefer one numeraire to others but theory doesn't suggest there is a best one. \textbf{We want to exploit this in our financial models.}
			\begin{columns}
				\column{0.5\linewidth}
				\includegraphics[width=0.9\linewidth]{images/usd_eur}
				\column{0.5\linewidth}
				\includegraphics[width=0.9\linewidth]{images/usd_chy}
			\end{columns}
		\end{itemize}
	\end{frame}
	
	\begin{frame}{Deterministic or Stochastic Numeraires}
		\begin{itemize}
			\item<1-> \textcolor{maincolor}{Deterministic numeraires} are easy to handle as they imply just an algebraic transformation (i.e. do not involve any risk),
			\begin{itemize}
				\item the exchange rate of the Italian Lira to the Euro was locked at EUR 1 = ITL 1936.27 on 31 December 1998.
			\end{itemize}
			\item<2-> When the numeraire is a \textcolor{maincolor}{stochastic process}, and we want to move to it, the pricing of a claim has to be changed in order to take into account the new risk (the intrinsic randomness of the new numeraire).
			\item<3-> In particular a change of numeraire implies also a change in the underlying measure (probability distribution). Indeed, starting with the bank account numeraire
			\begin{equation*}
				\begin{cases}
					\cfrac{S}{B_t} = \cfrac{\sum_i S_i p_i}{e^{rt}}\\
					\cfrac{U}{B_t} = \cfrac{\sum_i U_i p_i}{e^{rt}}
				\end{cases}\implies
				\frac{S}{U} = \frac{\sum_i S_i p_i}{\sum_j U_j p_j} = \sum_i S_i \frac{p_i}{\sum_j U_j p_j} = \sum_i S_i \pi_i
			\end{equation*}
		\end{itemize}
	\end{frame}
	
	\begin{frame}{Stochastic Numeraire Examples (I)}
		\begin{itemize}
			\item<1-> \textbf{Money market account.} Given $r_t$, a possibly random and time dependent risk-free interest rate process, let
			\begin{equation*}
				B_t := \exp\left(\int_0^t r_s ds\right)
			\end{equation*}
			In this case 
			\begin{equation*}
				\tilde{S_t}:=\frac{S_t}{B_t}=e^{-\int_0^t r_s ds}S_t, \quad t \ge 0
			\end{equation*}
			represents the discounted price of the asset at time 0.
			\item<2-> \textbf{Currency exchange rate.} In this case $N_t := R_t$ denotes e.g. the EUR/SGD exchange rate. Let
			\begin{equation*}
				\tilde{S_t}:=\frac{S_t}{R_t}, \quad t \ge 0
			\end{equation*}
			denotes the price of a local (SG) asset quoted in units of the foreign currency (EUR) (notice the difference with previous ITL/EUR example above).
		\end{itemize}
	\end{frame}
	
	\begin{frame}{Stochastic Numeraire Examples (II)}
		\begin{itemize}
			\item<1-> \textbf{Forward numeraire.} The price $P(t,T)$ of a bond paying $P(T,T)=1$ at maturity $T$. In this case
			\begin{equation*}
				N_t := P(t,T)=\E\left[e^{\int_t^T r_s ds}\right]
			\end{equation*}
			%Notice that the process $t\rightarrow e^{\int_0^t r_s ds} P(t,T)=\E\left[e^{\int_0^T r_s ds}\right], \quad 0 \le t \le T$ is a martingale.
			\item<2-> \textbf{Annuity numeraire.} Processes of the form
			\begin{equation*}
				N_t = P(t, T_0, T_n) := \sum_{k=1}^{n}(T_k - T_{k-1})P(t, T_k), \quad 0 \le t \le T
			\end{equation*}
			where $P(t,T_1),P(t,T_2),\ldots,P(t,T_n)$ are bond prices with maturities $T_1 < T_2 < \ldots < T_n$.
		\end{itemize}
	\end{frame}
	
	\subsection{Radon-Nikodym Derivative}
	\begin{frame}{Radon-Nikodym Derivative}
		We need to understand how to pass from a numeraire to another, hence by a measure to another, in an arbitrage free setting (until now we have worked with the bank account $B$ numeraire).
		\begin{block}{Definition}
			When two measures are equivalent it is possible to express the first in terms of the second through the \textcolor{maincolor}{Radon-Nikodym derivative}. Indeed there exists a strictly positive \textcolor{maincolor}{martingale process $\zeta_t$} such that
			\begin{equation*}
				\mathbb{Q}^* =\int_{A} \zeta_t(\omega)d\mathbb{Q}(\omega) := \frac{d\mathbb{Q}^*}{d\mathbb{Q}} = \zeta_t
				\label{eq:radon_nikodym_der}
			\end{equation*}				
		\end{block}
	\end{frame}
	
	\begin{frame}{Intuition from Expected Value}
		\begin{itemize}
			\item To get a sense about Radon-Nikodym derivative, consider a function $F(x)$ with density function $p(x)$ under a measure $\mathbb{P}$.
			\item Suppose there exists a function $q(x)$, which the properties of a density function, then we can write
			\begin{equation*}
				\expect{P}[F(x)]=\int F(x)p(x)dx = \int F(x)p(x)\frac{q(x)}{q(x)}dx
			\end{equation*}
			\item If we define $\psi(x)=F(x)\frac{p(x)}{q(x)}$ the expected value can be written as 
			\begin{equation*}
				\expect{P}[F(x)] =\int\psi(x)q(x)dx=\expect{Q}[\psi(x)]=\expect{Q}\left[F(x)\frac{p(x)}{q(x)}\right] 
				=\expect{Q}\left[F(x)\zeta\right]
			\end{equation*}
			\item The process $\zeta$ tracks the \emph{likelihood ratio} between two measures and transforms the expectations dynamically. What is $\E[\zeta]$ ? Intuitively, we have that
			\begin{equation*}
				\E^\mathbb{Q}[\zeta]=\E^\mathbb{Q}\left[\frac{d\mathbb{P}}{d\mathbb{Q}}\right]=\int \frac{p(x)}{\cancel{q(x)}} \cancel{q(x)} dx =\int p(x)dx = 1
			\end{equation*}
		\end{itemize}
		
	\end{frame}
	
	\begin{frame}{Conditional Expectations}
		\begin{itemize}
			\item In case of a conditioned expectation
			\begin{equation}
				\expect{P}[X|\mathcal{F}_t] = \frac{\expect{Q}\left[X\cfrac{d\mathbb{P}}{d\mathbb{Q}}\bigg|\mathcal{F}_t\right]}{\expect{Q}[\zeta_t|\mathcal{F}_t]}
				\label{eq:conditioned_expectation}
			\end{equation}
			\item To prove this result, for $A \in \mathcal{F}_t$:
		\begin{align*}
			\int_A \frac{\mathbb{E}^\mathbb{Q}[X\zeta_T|\mathcal{F}_t]}{\zeta_t} d\mathbb{P} &= \int_A \frac{\mathbb{E}^\mathbb{Q}[X\zeta_T|\mathcal{F}_t]}{\zeta_t} \zeta_T d\mathbb{Q} = \quad \text{by tower property}  \\ 
			&= \int_A \mathbb{E}^\mathbb{Q} \left[ \frac{\mathbb{E}^\mathbb{Q}[X\zeta_T|\mathcal{F}_t]}{\zeta_t} \zeta_T \bigg| \mathcal{F}_t \right] d\mathbb{Q} = \int_A \frac{\mathbb{E}^\mathbb{Q}[X\zeta_T|\mathcal{F}_t]}{\zeta_t} \mathbb{E}^\mathbb{Q}[\zeta_T|\mathcal{F}_t] d\mathbb{Q} \\
			&= \int_A \mathbb{E}^\mathbb{Q}[X\zeta_T|\mathcal{F}_t] d\mathbb{Q} = \int_A X\zeta_T d\mathbb{Q} = \int_A X d\mathbb{P}
		\end{align*}
		\end{itemize}
	\end{frame}
	
	\begin{frame}[fragile]{Radon-Nikodym Derivative Example}
		\begin{itemize}
			\item Imagine two probability densities $\mathbb{P} = \mathcal{N}(\mu_1, \sigma_1)$ and $\mathbb{Q} = \mathcal{N}(\mu_2, \sigma_2)$.
			\item Let's numerically compute the Radon-Nikodym derivative that moves you from the $\mathbb{P}$ to the $\mathbb{Q}$ world.
		\end{itemize}
		\begin{columns}
			\column{0.5\linewidth}
			\begin{ipython}
mu_p, sigma_p = 0.05, 0.2
mu_q, sigma_q = 0.02, 0.2

P = norm(mu_p, sigma_p)
Q = norm(mu_q, sigma_q)

def dQ_dP(x):
    return Q.pdf(x) / P.pdf(x)

def f(x, strike=0.03):
    return np.maximum(x - strike, 0)

val_Q, _ = quad(lambda x: f(x) * Q.pdf(x), -5, 5)
val_P_adj, _ = quad(lambda x: f(x)*dQ_dP(x)*P.pdf(x), -5, 5)
E_RND, _ = quad(lambda x: dQ_dP(x)* P.pdf(x), -5, 5)
\end{ipython}
\begin{ioutput}
Expectation under Q (P scaled): 0.074888 (0.074888)
Expectation RDN: 1.0
\end{ioutput}
			\column{0.5\linewidth}
			\begin{center}
				\includegraphics[width=1\linewidth]{images/radon_nikodym_plot}
			\end{center}
		\end{columns}
	\end{frame}
	
	\subsection{Change of Numeraire}
	\begin{frame}{Change of Numeraire}
		\begin{block}{Theorem}
			Assume exists a numeraire $N_t$ and the associated measure $\mathbb{Q}^N$, \textit{equivalent} to $\mathbb{P}$, such that the price of every traded asset $S_t$ relative to $N$ is a \textit{martingale} under $\mathbb{Q}^N$.
			\begin{enumerate}
			\item Let $U$ be another arbitrary numeraire, \textbf{there exists a measure $\mathbb{Q}^U$, also equivalent to $\mathbb{P}$, such that the price of every traded asset $S_t$, normalized to $U$, is a martingale under $\mathbb{Q}^U$}
			\begin{equation*}
				\frac{S_t}{U_t} = \expect{U}\left[\frac{S_T}{U_T}\bigg|\mathcal{F}_t\right],\quad 0\le t \le T
			\end{equation*}
			\item The Radon-Nikodym derivative defining the measure $\mathbb{Q}^U$ is given by
			\begin{equation}
				\frac{d\mathbb{Q}^U}{d\mathbb{Q}^N} = \frac{U_T N_t}{U_t N_T}
				\label{eq:radon_nikodym_der2}
			\end{equation}
			\end{enumerate}
		\end{block}
	\end{frame}	
	
	\begin{frame}{Change of Numeraire (Proof part 2)}
		\begin{itemize}
			\item Let's prove first the second part.
			\item By definition of $\mathbb{Q}^N$, for every asset price $S_t$ holds
			\begin{equation*}
				\begin{cases} 
					\expectt{N}{t}\left[\cfrac{S_T}{N_T}\right] = \cfrac{S_t}{N_t} \\
					\expectt{U}{t}\left[\cfrac{U_t S_T}{N_t U_T}\right] = \cfrac{U_t}{N_t}\expectt{U}{t}\left[\cfrac{S_T}{U_T}\right] = \cfrac{\cancel{U_t} S_t}{N_t \cancel{U_t}} = \cfrac{S_t}{N_t}
				\end{cases}
			\end{equation*}
			\item Since both equations result in $S_t/N_t$ they can be equated
			\begin{equation*}
				\expectt{N}{t}\left[\cfrac{S_T}{N_T}\right] = \expectt{U}{t}\left[\cfrac{U_t S_T}{N_t U_T}\right]
			\end{equation*}
		\end{itemize}
		\vfill
		\begin{tcolorbox}[colback=gray!5, colframe=maincolor, arc=0mm, boxrule=0.5pt]
		\footnotesize
		From time to time to ease the notation I will use $\expect{N}\left[\cdots \middle| \mathcal{F}_t\right] = \expectt{N}{t}\left[\cdots\right]$
		\end{tcolorbox}
	\end{frame}
	
	\begin{frame}{Change of Numeraire (Proof part 2)}
		\begin{itemize}
			\item By definition of Radon-Nikodym derivative and from the result of the previous slide, it holds
			\begin{equation*}
				\begin{cases}
					\expectt{N}{t}\left[\cfrac{S_T}{N_T}\right] =\expectt{U}{t}\left[\cfrac{S_T}{N_T} \cfrac{d\mathbb{Q}^N}{d\mathbb{Q}^U}\right] \\
					\expectt{N}{t}\left[\cfrac{S_T}{N_T}\right] =\expectt{U}{t}\left[\cfrac{U_t S_T}{N_t U_T}\right]
				\end{cases}\implies
				\cfrac{\cancel{S_T}}{N_T} \cfrac{d\mathbb{Q}^N}{d\mathbb{Q}^U} = \cfrac{U_t \cancel{S_T}}{N_t U_T}
			\end{equation*}
			where we have used the fact that the expectation arguments under $U$ must equal to get \cref{eq:radon_nikodym_der2}. 
		\end{itemize}
		\myendproof
	\end{frame}
	

	
	\begin{frame}{Change of Numeraire (Proof part 1)}
		\begin{itemize}
			\item The conditional expectation under the new measure $\mathbb{Q}^U$ is:
			\begin{equation*}
				\mathbb{E}^U \left[ \frac{S_T}{U_T} \bigg| \mathcal{F}_t \right] = \frac{\mathbb{E}^N \left[ \frac{d\mathbb{Q}^U}{d\mathbb{Q}^N} \frac{S_T}{U_T} \bigg| \mathcal{F}_t \right]}{\mathbb{E}^N \left[ \frac{d\mathbb{Q}^U}{d\mathbb{Q}^N} \bigg| \mathcal{F}_t \right]}
			\end{equation*}
			\item Substituting the Radon-Nikodym derivative $\frac{d\mathbb{Q}^U}{d\mathbb{Q}^N} = \frac{U_T / U_0}{N_T / N_0}$:
			\begin{align*}
				\text{Numerator:} \quad & \mathbb{E}^N \left[ \frac{U_T N_0}{U_0 N_T} \frac{S_T}{U_T} \bigg| \mathcal{F}_t \right] = \frac{N_0}{U_0} \mathbb{E}^N \left[ \frac{S_T}{N_T} \bigg| \mathcal{F}_t \right] = \frac{N_0 S_t}{U_0 N_t} \\
				\text{Denominator:} \quad & \mathbb{E}^N \left[ \frac{U_T N_0}{U_0 N_T} \bigg| \mathcal{F}_t \right] = \frac{N_0}{U_0} \mathbb{E}^N \left[ \frac{U_T}{N_T} \bigg| \mathcal{F}_t \right] = \frac{N_0 U_t}{U_0 N_t}
			\end{align*}
			\item The ratio yields $\frac{S_t}{U_t}$, proving the martingale property.
		\end{itemize}
		\myendproof
	\end{frame}
	
	\begin{frame}{Change of Numeraire Remarks}
		This powerful theorem, we have just proved allows to
		\begin{enumerate}
			\item give a simple rule to write (the otherwise difficult to derive) Radon-Nikodym derivative;
			\item the arbitrage-free price of a claim $V$ at time $t$ is unique. The choice of numeraire only changes the \textit{representation} of the calculation:
			\begin{equation*}
				V_t = B_t \mathbb{E}^{\mathbb{Q}^B} \left[ \frac{V_T}{B_T} \bigg| \mathcal{F}_t \right] = N_t \mathbb{E}^{\mathbb{Q}^N} \left[ \frac{V_T}{N_T} \bigg| \mathcal{F}_t \right]
			\end{equation*}
			\item find a characterization of our process by means of which we can work-out more easily the fundamental pricing formula. In particular under the "standard" risk-neutral measure $\mathbb{Q}^B$, assets grow at rate $r$ which makes integration harder.
			By switching to measure $\mathbb{Q}^N$:
			 \begin{itemize}
			 	\item "normalize" the world so that $N$ is the benchmark.
			 	\item the relative price $S/N$ becomes a \textcolor{red}{driftless process} (Martingale).
			 	\item the expectation $\mathbb{E}^{\mathbb{Q}^N}$ becomes a simple calculation of the payoff's "average" value in terms of the new benchmark.
			 \end{itemize}
		\end{enumerate}
	\end{frame}
	

%	\begin{frame}{Asset Price divided by Numeraire}
%		\begin{itemize}
%			\item<1-> Let $B$ be the money market account and $\mathbb{Q}^B$ the corresponding risk-neutral measure. Also let $N$ be another numeraire (note that for what just said $N/B$ is a $\mathbb{Q}^B$-martingale). 
%			\item<2-> From the Change of Numeraire Theorem we can define a new measure by mean of
%			\begin{equation*}
%				\frac{d\mathbb{Q}^N}{d\mathbb{Q}^B} = \frac{N_TB_0}{B_TN_0}
%			\end{equation*}
%			\item<3-> Then, for any asset $S$ such that $S/B$ is a $\mathbb{Q}^B$-martingale
%			\begin{equation*}
%				\expect{N}\left[\frac{S_T}{N_T}\bigg|\mathcal{F}_t\right] = \cfrac{\expect{B}\left[\cfrac{S_T}{N_T}\cfrac{N_TB_0}{B_TN_0}\bigg|\mathcal{F}_t\right]}{\expect{B}\left[\cfrac{N_TB_0}{B_TN_0}\bigg|\mathcal{F}_t\right]}
%				=\cfrac{\expect{B}\left[\cfrac{S_T}{B_T}\bigg|\mathcal{F}_t\right]}
%				{\expect{B}\left[\cfrac{N_T}{B_T}\bigg|\mathcal{F}_t\right]}
%				=\frac{S_tB_t}{N_tB_t}=\frac{S_t}{N_t}
%			\end{equation*}
%			\myendproof
%			\item<4-> \textcolor{maincolor}{So $S/N$ is a $\mathbb{Q}^N$-martingale.}
%		\end{itemize}
%	\end{frame}
	
	\subsection{Applications}
	\begin{frame}{Examples}
		\footnotesize{\tiny {\tiny }}{
			\begin{table}[bt]
				\renewcommand*{\arraystretch}{1.4}
				\begin{tabular}{|l|l|} \hline
					\begin{tabular}{@{}l@{}}
						Any asset divided by the bank account
						$B_t$\\(recall $dB_t = r_t B_t dt$)
						\boxed{\cfrac{S_t}{B_t} = e^{-\int_0^t r_s ds}S_t}
					\end{tabular}
					& \begin{tabular}{l}
						It is a martingale under the\\
						measure $\mathbb{Q}^B$ associated to \\
						the bank account numeraire,\\
						i.e. the risk neutral measure.
					\end{tabular} \\ \hline
					\begin{tabular}{@{}l@{}}
						The forward rate\\
						\boxed{F(t; T_1, T_2) = \frac{1}{T_2-T_1}\left(\frac{P(t,T_1) - P(t,T_2)}{P(t,T_2)}\right)}\\
						can be interpreted as a portfolio of two ZCBs\\
						divided by another ZCB.		
					\end{tabular}
					& \begin{tabular}{l}
						Under the measure $\mathbb{Q}^{T^2}$\\ 
						associated to the numeraire\\ 
						$P(\cdot,T_2)$ it is a martingale.\end{tabular}\\ \hline  
					\begin{tabular}{@{}l@{}}
						The swap rate
						\boxed{S_{\alpha,\beta}(t) = \frac{P(t,T_\alpha)-P(t,T_\beta)}{\sum_{i=\alpha+1}^{\beta}\tau_i P(t,T_i)}}
						\\can be interpreted as a portfolio of two ZCBs\\
						divided by a portfolio of ZCBs.		
					\end{tabular}
					& \begin{tabular}{l}
						It is a martingale under the\\
						measure associated to the\\
						annuity numeraire.
					\end{tabular} \\ \hline
				\end{tabular}
		\end{table}}
	\end{frame}
	
%	\begin{frame}{A Useful Separation}
%		\begin{itemize}
%			\item<1-> Until now we have used $B(t)$, the money market account, as numeraire. But as we have seen it is natural to look for the most convenient one, that is, the one which minimizes the mathematical difficulties according to the problem at hand.
%			\item<2-> Given a contingent claim whose payoff at time $T$ is $\chi$, we have the following formula for its price $\Pi$
%			\begin{equation*}
%				\Pi_\chi(t,T)=\expect{B}\left[e^{-\int_t^T r_s ds}\chi\bigg|\mathcal{F}_t \right]=\expect{B}\left[e^{\int_0^t r_s ds}e^{-\int_0^T r_s ds}\chi\bigg|\mathcal{F}_t \right]=B_t\expect{B}\left[B^{-1}_T\chi|\mathcal{F}_t\right]
%			\end{equation*}
%			\item<3-> If $\chi$ and the short rate process were independent under $\mathbb{Q}^B$ (recall $\E[XY]=\E[X]\E[Y]$) then we could write
%			\begin{equation*}
%				\Pi_\chi(t,T)=\expect{B}\left[e^{-\int_t^T r_s ds}\bigg|\mathcal{F}_t\right]\expect{B}\left[\chi|\mathcal{F}_t\right] = P(t,T)\expect{B}\left[\chi|\mathcal{F}_t\right]
%			\end{equation*}
%		\end{itemize}
%	\end{frame}
	
	\begin{frame}{Simplifying the Pricing Formula}
		\begin{itemize}
			\item  Until now we have used $B(t)$, the money market account, as numeraire. In other situations, a better numeraire ist he \textcolor{maincolor}{forward measure} $\mathbb{Q}^T$ (also called the $T$-measure) is defined as the martingale measure associated to the numeraire process $P(\cdot,T)$.
			\item It is easy to see that using \cref{eq:radon_nikodym_der2}, the Radon-Nykodim derivative is given in this case by
			\begin{equation}
				\zeta_t = \frac{d\mathbb{Q}^T}{d\mathbb{Q}^B} = \frac{P(t,T)\cancel{B(0)}}{B_t P(0,T)} ,\quad\left(\zeta_T=\frac{\cancel{P(T,T)}B(0)}{B(T)P(0,T)}=\frac{1}{B(T)P(0,T)}\right)
				\label{eq:radon_nikodym_t_forward}
			\end{equation}
			\item Applying the change of numeraire to the pricing formula, we get
			\begin{equation*}
				\begin{aligned}
					\Pi_\chi(t,T) & = B_t\expect{B}\left[B^{-1}_T\chi|\mathcal{F}_t\right] 
					= B_t\expect{B}\left[P(0,T)\zeta_T\chi|\mathcal{F}_t\right]\quad\text{(using } B_T^{-1} = \zeta_TP(0,T))\\
					& = B_tP(0,T)\expect{B}\left[\zeta_T|\mathcal{F}_t\right]\expect{T}\left[\chi|\mathcal{F}_t\right]\quad\text{(by \cref{eq:conditioned_expectation}, cond. expect.)}\\
					& = B_tP(0,T)\zeta_t\expect{T}\left[\chi|\mathcal{F}_t\right] =  \cancel{B_tP(0,T)}\frac{P(t,T)}{\cancel{B_tP(0,T)}}\expect{T}\left[\chi|\mathcal{F}_t\right] = P(t,T)\expect{T}\left[\chi|\mathcal{F}_t\right]
				\end{aligned}
			\end{equation*}
			\item It achieves the desired separation although under a new measure.			
		\end{itemize}
	\end{frame}
	
	\begin{frame}{Identity between $\mathbb{Q}^B$ and $\mathbb{Q}^T$}
		By construction of the martingale measure $\mathbb{Q}^B$, the following relationship holds
		\begin{equation*}
			\begin{gathered}
				\frac{P(t,T)}{B_t}=\expect{B}\left[\frac{P(T,T)}{B_T}\right]\\[0.3cm]
				P(t,T)=\expect{B}\left[\frac{P(T,T)}{B_T}B_t\right] = \expect{B}\left[\frac{B_t}{B_T}\right]
			\end{gathered}
		\end{equation*}
		\pause
		Plugging the result into the Radon-Nikodym derivative gives
		\begin{equation*}
			\frac{d\mathbb{Q}^T}{d\mathbb{Q}^B} = \frac{B_t}{B_T}\frac{1}{P(t,T)} =\frac{B_t/B_T}{\expect{B}[B_t/B_T]}
		\end{equation*}
		\myendproof
		\pause
		\begin{block}{Proposition}
			If interest rates are deterministic (i.e. the Radon-Nikodym derivative is 1), then the measures $\mathbb{Q}^B$ and $\mathbb{Q}^T$ are identical.
		\end{block}
	\end{frame}

	\begin{frame}{The Forward Rate Under $\mathbb{Q}^T$}
		\begin{block}{Proposition}
			Consider the forward numeraire $P(t,T)$ and denote with $\mathbb{Q}^T$ its associated measure.
			The forward rate spanning the interval $[S,T]$ is the $\mathbb{Q}^T$-expectation of the future spot rate at time $S$ for the maturity $T$
			\begin{equation}
				F(t;S,T) =\expect{T}[L(S,T)|\mathcal{F}_t]\quad \text{with }t<S<T
			\end{equation}
		\end{block}
		\vfill
		$F(t;S,T) = (P(t,S)-P(t,T))/(\tau\cdot P(t,T))$ is the value of a tradable portfolio (long $1/\tau$ bonds maturing at $S$, short $1/\tau$ bonds maturing at $T$) divided by the numeraire $P(t,T)$. It must be a $\mathbb{Q}^T$-martingale
		\begin{equation*}
			F(t;S,T) = \expect{T}[F(S;S,T)|\mathcal{F}_t] = \expect{T}\left[\frac{1}{\tau}\left(\frac{1-P(S,T)}{P(S,T)}\right)\bigg|\mathcal{F}_t\right] = \expect{T}[L(S,T)|\mathcal{F}_t]
		\end{equation*}
		\myendproof
	\end{frame}
	
	\begin{frame}{The Forward Rate Under $\mathbb{Q}^T$ (I)}
		A similar result can be derived for the corresponding instantaneous quantities
		$f(t,T)=\expect{T}[r_T|\mathcal{F}_t]$. Indeed in $\mathbb{Q}^B$ measure
		\begin{equation*}
			\frac{P(t,T)}{B_t}=\expect{B}\left[\frac{P(T,T)}{B_T}\bigg|\mathcal{F}_t\right] \implies
			P(t,T)=\expect{B}\left[\frac{B_tP(T,T)}{B_T}\bigg|\mathcal{F}_t\right]=\expect{B}\left[e^{-\int_t^Tr_u du}\big|\mathcal{F}_t\right]
		\end{equation*}
		Differentiating with respect to $T$ ($\frac{d}{dx}\int_c^x f(t)dt=f(x)$)
		\begin{equation*}
			\frac{\partial P(t,T)}{\partial T}=
			\expect{B}\left[-r(T)e^{-\int_t^Tr_u du}\big|\mathcal{F}_t\right]
		\end{equation*}
		Changing numeraire to $P(\cdot,T)$and using reciprocal of \cref{eq:radon_nikodym_t_forward} ($\zeta^{-1}=\frac{B_t/B_T}{P(t,T)/P(T,T)}$)
		\begin{equation*}
			\frac{\partial P(t,T)}{\partial T}=
			\expect{T}\left[-r(T)\cancel{e^{-\int_t^Tr_u du}}\frac{P(t,T)}{\cancel{e^{-\int_t^Tr_u du}}}\bigg|\mathcal{F}_t\right]=
			P(t,T)\expect{T}\left[-r(T)|\mathcal{F}_t\right]
		\end{equation*}
	\end{frame}
	
	\begin{frame}{The Forward Rate Under $\mathbb{Q}^T$ (II)}
		Hence
		\begin{equation*}
			\begin{aligned}
				f(t,T)&=\frac{1}{P(t,T)}\frac{\partial P(t,T)}{\partial T}=
				-\frac{\partial \ln P(t,T)}{\partial T}
				= \expect{T}\left[r(T)|\mathcal{F}_t\right]=	\expect{T}\left[f(T,T)|\mathcal{F}_t\right]
			\end{aligned}
		\end{equation*}
		Which demonstrates the initial statement and also shows that \textcolor{maincolor}{the instantaneous forward rate is a martingale under the $T$-forward measure}. 
		\myendproof
		
		Note the negative sign which represents the "decay" of the bond
		price as maturity increase.
		
		\begin{tcolorbox}[colback=gray!5, colframe=maincolor, arc=0mm, boxrule=0.5pt]
			\footnotesize
			$f(t, T) = \expect{T}[r(T)|\mathcal{F}_t]$ has a nice connection with the \textcolor{maincolor}{expectation hypothesis of the term structure of interest rates}.
			Its basic idea is that the long-term rate is determined purely by current and future expected short-term rates.
			\begin{itemize}
				\item \href{https://pages.stern.nyu.edu/~sternfin/asangvin/ExpHyp.pdf}{\emph{The Expectation Hypothesis, A. Sangvinatsos}}
				\item \href{https://www.jstor.org/stable/2327547}{\emph{A Re-Examination of Traditional Hypotheses about the Term Structure of Interest Rates}, J.C. Cox, J.E. Ingersoll, and S.A. Ross}
			\end{itemize}
		\end{tcolorbox}	
	\end{frame}
		
%	\begin{homework}
%		\begin{frame}{\textcolor{white}{Homework}}
%			\begin{itemize}
%				\item[white] If the pure expectations hypothesis holds, why might the yield curve be flat, upward sloping, or downward sloping? The government implements a credible “tight” monetary policy by raising short-term (e.g. 3-month) interest rates. How could this affect the yield curve?
%			\end{itemize}
%	%If the pure expectations hypothesis holds, why might the yield curve be flat, upward sloping, or downward sloping? The government implements a credible “tight” monetary policy by raising short-term (e.g. 3-month) interest rates. How could this affect the yield curve?
%	%
%	%
%	%
%	%Why the Yield Curve Can Be Flat, Upward Sloping, or Downward Sloping Under the Pure Expectations Hypothesis
%	%
%	%The pure expectations hypothesis (PEH) suggests that the shape of the yield curve is solely determined by market expectations of future short-term interest rates. Here's how different expectations lead to different yield curve shapes:
%	%
%	%Flat Yield Curve: A flat yield curve implies that the market expects short-term interest rates to remain relatively constant in the future. If investors anticipate that short-term rates will neither rise nor fall significantly, long-term rates (which are essentially averages of expected future short-term rates) will be similar to current short-term rates, resulting in a flat curve.
%	%
%	%Upward Sloping Yield Curve: An upward sloping yield curve (also known as a "normal" yield curve) indicates that the market expects short-term interest rates to rise in the future. As long-term rates reflect the average of expected future short-term rates, they will be higher than current short-term rates if future rates are anticipated to increase.   
%	%
%	%Downward Sloping Yield Curve: A downward sloping yield curve (also known as an "inverted" yield curve) suggests that the market expects short-term interest rates to decline in the future. In this case, long-term rates will be lower than current short-term rates because they incorporate the expectation of falling future short-term rates.   
%	%
%	%How a "Tight" Monetary Policy Affects the Yield Curve
%	%
%	%When a government implements a credible "tight" monetary policy by raising short-term interest rates, it can have a significant impact on the yield curve. Here's how:
%	%
%	%Initial Impact: The immediate effect of raising short-term rates is to push the very front end of the yield curve upward. This is because the policy directly targets short-term rates.
%	%
%	%Expectations of Future Rates: The key question is how this policy action affects market expectations of future short-term rates. If the market believes that the central bank's commitment to tight monetary policy is credible and that the rate hikes are likely to be sustained for some time to combat inflation, then expectations of future short-term rates may also rise. This would lead to an upward shift in the entire yield curve.
%	%
%	%Potential for Inversion: However, if the market believes that the rate hikes will eventually lead to an economic slowdown or recession, expectations of future short-term rates may actually decline. In this scenario, the yield curve could become flat or even inverted. This is because long-term rates would reflect the expectation of lower future short-term rates, even though current short-term rates are high due to the policy action.
%	%
%	%Credibility is Key
%	%
%	%The credibility of the government's monetary policy is crucial in determining how the yield curve responds. If the market trusts that the central bank will stick to its tight policy, the yield curve is more likely to shift upward. However, if the market doubts the central bank's resolve or believes that the policy will be short-lived, the yield curve may flatten or invert.
%	%
%	%Other Factors
%	%
%	%It's important to remember that the yield curve is influenced by various factors beyond just expectations of future short-term rates. These include:   
%	%
%	%Inflation expectations: If the tight monetary policy is successful in curbing inflation, this could lower long-term rates.
%	%Economic growth prospects: If the market believes the rate hikes will lead to a recession, this could put downward pressure on long-term rates.
%	%Supply and demand for bonds: The relative supply and demand for bonds of different maturities can also affect the yield curve.   
%	%In summary:
%	%
%	%The yield curve's shape reflects market expectations of future short-term interest rates.
%	%
%	%A credible tight monetary policy can initially raise short-term rates and potentially lead to an upward shift in the yield curve if the market believes the policy will be sustained. However, if the market anticipates an economic slowdown, the yield curve may flatten or invert. The credibility of the policy and other economic factors play a significant role in determining the ultimate impact on the yield curve.   
%		\end{frame}
%	\end{homework}

	
	\subsection{Girsanov Theorem}
	\begin{frame}{Which Dynamics ?}
		\begin{itemize}
			\item<1-> We're left with one important question:
			\textcolor{maincolor}{what does the path of an asset price $S_t$ look like under a new measure $\mathbb{Q}$ ?} (need to know in order to really compute its expectation under $\mathbb{Q}$)
			\item<2-> \emph{Girsanov's theorem} answers to this question since it tells us, when we change from $\mathbb{P}$ to some other measure $\mathbb{Q}$, how the dynamics of a process ($W_t$) changes under $\mathbb{Q}$.
			\item<3-> Will see that it evolves as the sum of a Brownian motion under $\mathbb{Q}$ and a drift related to the Radon-Nikodym derivative characterizing $\mathbb{Q}$.
			\item<4-> In many cases we therefore want to choose the Radon-Nikodym derivative so that the drift of $W_t$ w.r.t. $\mathbb{Q}$ exactly cancels out the drift of $S_t$, leaving us with a pure diffusion process (martingale). 
		\end{itemize}
	\end{frame}
	
\begin{frame}{Girsanov Theorem}
	\begin{block}{Theorem}
		Consider the SDE 
		\begin{equation*}
			dX_t = f_t dt + \sigma_t dW_t
		\end{equation*}
		under $\mathbb{P}$. 
		
		Let be given a new drift $f^*_t$ and assume $\gamma_t=\frac{f_t^*-f_t}{\sigma_t}$ such that $\E\left[\exp\left(\frac{1}{2}\int_0^t\gamma_t^2dt\right)\right]<\infty$.
		Define the measure 
		\begin{equation}
			\frac{d\mathbb{P}^*}{d\mathbb{P}}=\exp\left(-\frac{1}{2}\int_0^t \gamma_s^2 ds + \int_0^t \gamma_s dW_s \right)
		\end{equation}
		Then $\mathbb{P}^*$ is equivalent to $\mathbb{P}$. 
		The Radon-Nikodym derivative process is an \textcolor{maincolor}{exponential martingale}.
	\end{block}
\end{frame}

\begin{frame}{Girsanov Theorem}
	\begin{block}{Theorem continued}
		Also the process
		\begin{equation}
			dW^*_t = dW_t -\gamma_s dt
		\end{equation} 
		is a Brownian motion under $\mathbb{P}^*$, and 
		\begin{equation*}
			dX_t = f^*_t dt + \sigma_t dW^*_t
		\end{equation*}
		The condition $\E\left[\exp\left(\frac{1}{2}\int_0^t\gamma_t^2dt\right)\right]<\infty$ is a sufficient but non-necessary, and it is know as the \textcolor{maincolor}{Novikov condition}.
		This ensures the RN derivative is a true martingale and not just a local martingale. If it's not a true martingale, $\E[\frac{d\mathbb{Q}}{d\mathbb{P}}]\ne 1$, and $\mathbb{Q}$ is not a probability measure.
	\end{block}
\end{frame}
		
%	\begin{frame}{A Trivial Example}
%		\begin{itemize}
%			\item Consider the stochastic differential equation
%			\begin{equation*}
%				dX_t = b(X_t, t) dt + a(X_t, t) dW_t
%			\end{equation*}
%			\item Let's assume that the drift and diffusion coefficients are such that there exists a unique solution to the equation which is $X$.
%			\item We want to find a probability measure $\mathbb{P}^*$ such that the drift of $X$ is $\tilde{b}(X_t,t)$ instead of $b(X_t,t)$.
%		\end{itemize}
%	\end{frame}
%	
%	\begin{frame}{A Trivial Example}
%		\begin{equation*}
%			\begin{aligned}
%				dX_t &= \tilde{b}(X_t,t) dt+b(X_t,t) dt -\tilde{b}(X_t,t) dt + a(X_t,t) dW_t = \\
%				&=\tilde{b_t} dt + (b_t -\tilde{b_t})dt + a_t dW_t =\\
%				&=\tilde{b_t}dt+ a_t\overbrace{\left(\frac{b_t-\tilde{b_t}}{a_t}\right)}^{-\gamma_t}dt + a_t dW_t = \\
%				&= \tilde{b_t}dt+a_t dW_t - a_t\gamma_t dt\\
%				&=\tilde{b_t}dt+a_t d\tilde{W_t}
%			\end{aligned}
%		\end{equation*}
%		where $d\tilde{W_t}=dW_t-\gamma_t dt$.
%		
%		    Typo: You have bt​~​dt+(bt​−bt​~​)dt+at​dWt​. On the next line, you define −γt​=at​bt​−b~t​​.
%		
%		Clarity: Explicitly state: "We are 'borrowing' drift from the dt term and 'hiding' it inside the dWt​ term to create dW~t​."
%		
%	\end{frame}
%	
%	\begin{frame}{A Trivial Example}
%		\begin{itemize}
%			\item<1-> If the Novikov condition is satisfied then we can apply the Girsanov theorem and we have that
%			\begin{equation}
%				\mathbb{P}^* = \expect{P}\left[\exp\left(-\frac{1}{2}\int_0^t \gamma_s^2 ds + \int_0^t \gamma_s dW_s \right)\right]
%			\end{equation}
%			and that $\tilde{W}$ is a Brownian motion on $\mathbb{P}^*$.
%			\item<2-> In practice, don't need to determine the new measure $\mathbb{P}^*$.
%			\item<2-> It is enough to know it exists, such that we can work with the resulting SDE of the process of interest under the new measure.	
%		\end{itemize}
%	\end{frame}
	
	\begin{frame}{Monte Carlo Strikes Again}
		Consider an Ito process with drift $\mu=0.8$ and diffusion coefficient $\sigma=0.4$, ($Y_0=0$).
		\begin{equation*}
			dY = \mu dt + \sigma dW
		\end{equation*}
		\begin{enumerate}
			\item Determine the appropriate transformation, according to the Girsanov theorem, which makes the original process drift-less.
			\item Verify that applying the Radon-Nikodym to the a simulated path of the process the drift indeed disappear.
		\end{enumerate}
	\end{frame}
	
	\begin{frame}[fragile]{A Trivial Example}
		First simulate $N=10000$ realizations of a Brownian Motion ($W$) and a of an Ito process ($Y$).
		\begin{codebox}
			\begin{ipython}[linewidth=0.6\linewidth]
T = 1
M = 252
dt = T/M
N = 10000
mu = 0.8
sigma = 0.4

epsilon = np.random.normal(size=(M, N))
dW = np.sqrt(dt) * epsilon
W = np.cumsum(np.vstack([np.zeros(N), dW]), axis=0)
dY = mu * dt + sigma * dW
Y = np.cumsum(np.vstack([np.zeros(N), dY]), axis=0)
			\end{ipython}
		\end{codebox}
	\end{frame}
	
	\begin{frame}{A Trivial Example}
		\begin{center}
			\includegraphics[width=0.9\linewidth]{images/ito_for_girsanov}
		\end{center}
	\end{frame}
	
	\begin{frame}{A Trivial Example}
		According to the Girsanov Theorem the Radon-Nikodym derivative in this case becomes $\gamma_t = \cfrac{-\mu}{\sigma}$.
		
		We have seen that the RN probability distribution tended to assign higher weights to negative values, thus reducing the resulting drift of the process.
		Using its definition above it is possible to verify the Girsanov theorem, in particular:
		\begin{enumerate}
			\item the original and the transformed process have the same evolution ($\sigma$ is unchanged in the transformation);
			\item $\E^{\mathbb{Q}}[Y] = 0$ so the new process is drift-less.
		\end{enumerate}  
	\end{frame}
	
	\begin{frame}[fragile]{A Trivial Example}
		\begin{columns}
			\column{0.5\linewidth}
			Define the $RN$ process according to the Girsanov theorem and compute $\E$ and standard deviation for the process $Y$ in the original measure and in the new one (using the Radon-Ninkodym derivative).
			\begin{ipython}	
gamma = -mu/sigma

W_T = np.random.normal(0, np.sqrt(T), N)
Y_T = mu * T + sigma * W_T

zeta_T = np.exp(gamma * W_T - 0.5 * (gamma**2) * T)
mean_P = np.mean(Y_T)
std_P = np.std(Y_T)
mean_Q = np.mean(Y_T * zeta_T)
std_Q = np.sqrt(np.mean((Y_T - mean_Q)**2 * zeta_T))
\end{ipython}
\begin{ioutput}
Mean under P: 0.8002 (Expected: 0.8000)
Mean under Q: 0.0035 (Expected: 0.0000)
Volatility P: 0.4003
Volatility Q: 0.3933	
\end{ioutput}			
			\column{0.5\linewidth}
			\includegraphics[width=0.9\linewidth]{images/girsanov}
		\end{columns}
	\end{frame}
	
	
%	\begin{homework}
%		\begin{frame}{\textcolor{white}{Homework}}
%			\begin{itemize}
%				\item[white] Consider a stock price which has the following dynamics under the real-world measure $\mathbb{P}$
%				\begin{equation*}
%					dS_t = \mu S_t dt + \sigma S_t dW_t
%				\end{equation*}
%				Determine what happens to the stock SDE when moving to two different numeraires:
%				\begin{itemize}
%					\item[white] risk-neutral measure (bank account numeraire);
%					\item[white] stock measure (stock numeraire).
%				\end{itemize}
%			\end{itemize}
%		\end{frame}
%	\end{homework}





%	\begin{frame}[fragile]{Physical vs Risk-Neutral Measure}
%		\begin{columns}
%			\column{0.5\linewidth}
%			Imagine an underlying asset $S$ which currently is valued 100. Simulate $N$ possible realization of the random variable $S$ both under the \textbf{physical} probability measure and the in the \textbf{risk-neutral measure}.
%			
%			Consider then a 1 year call option on that asset with strike $K=120$. Determine and compare its value using Monte Carlo simulation under both measures.	
%			\column{0.5\linewidth}
%			\begin{ipython}	
%				import numpy as np
%				
%				mu = 0.05
%				r = 0.01
%				sigma = 0.20
%				gamma = (mu-r)/sigma
%				S0 = 100
%				K = 120
%				T = 1
%				M = 365
%				dt = T/M
%				N = 100000
%			\end{ipython}
%		\end{columns}
%	\end{frame}
%	
%	\begin{frame}[fragile]{Physical vs Risk-Neutral Measure}
%		\begin{codebox}
%			\begin{ipython}[linewidth=0.7\linewidth]
%				dW = np.sqrt(dt)*np.random.normal(size=N*M).reshape(M, N)
%				SP = np.zeros_like(dW)
%				SP[0, :] = S0
%				for t in range(1, M):
%				SP[t, :] = SP[t-1, :] + mu*dt*SP[t-1, :] + sigma*SP[t-1, :]*dW[t-1, :]
%				
%				SP_exp = np.mean(SP, axis=1)
%			\end{ipython}
%		\end{codebox}
%		\begin{center}
%			\includegraphics[width=0.60\linewidth]{images/evolution_under_P}
%		\end{center}
%	\end{frame}
%	
%	\begin{frame}[fragile]{Physical vs Risk-Neutral Measure}
%		\begin{codebox}
%			\begin{ipython}[linewidth=0.7\linewidth]
%				SQ = np.zeros_like(dW)
%				SQ[0, :] = S0
%				for t in range(1, M):
%				SQ[t, :] = SQ[t-1, :] + r*dt*SQ[t-1, :] + sigma*SQ[t-1, :]*dW[t-1, :]
%				
%				SQ_exp = np.mean(SQ, axis=1)
%			\end{ipython}
%		\end{codebox}
%		\begin{center}
%			\includegraphics[width=0.60\linewidth]{images/evolution_under_Q}
%		\end{center}
%	\end{frame}
%	
%	\begin{frame}[fragile]{Physical vs Risk-Neutral Measure}
%		\begin{columns}
%			\column{0.7\linewidth}
%			\begin{ipython}
%				# sanity check with Radon-Nikodym Derivative
%				
%				RN = np.ones_like(dW)
%				for t in range(1, M):
%				RN[t, :] = np.exp(-gamma*dW[t, :] - 0.5*gamma**2*dt)
%				
%				
%				
%				# call price calculation under the two measures
%				
%				CQ = np.exp(-r*T)*np.maximum((SQ[-1, :]-K), 0)
%				print (f"Price under Q: {np.mean(CQ):.2f}")
%				
%				CP_RN = np.exp(-r*T)*np.maximum((SP[-1, :]-K), 0)*np.cumprod(RN, axis=0)[-1, :]
%				print (f"Price under Q (via RN): {np.mean(CP_RN):.2f}")		
%			\end{ipython}
%			\begin{ioutput}
%				
%				Price under Q: 2.31
%				Price under Q (via RN): 2.30
%			\end{ioutput}
%			\column{0.3\linewidth}
%			\includegraphics[width=0.80\linewidth]{images/rdn_physical_risk_neutral}
%			\vspace{1.6cm}
%		\end{columns}	
%	\end{frame}
%	
%	\begin{homework}
%		\begin{frame}{\textcolor{white}{Homework}}
%			\begin{itemize}
%				\item[white] Suppose $W(t)$ is a standard Brownian motion. For each of the following choices of $X_t$, find an equivalent probability measure $\mathbb{Q}$ such that $X_t$ is a Brownian motion in the new measure. Assume $X_0=W_0=0$
%				\begin{equation*}
%					\begin{gathered}
%						dX_t = 2dt + dW_t\\
%						dX_t = 2dt + 6dW_t
%					\end{gathered}
%				\end{equation*}
%				\item[white] Let $X_t$ be the unique solution to the following stochastic differential equation, under $\mathbb{P}$:
%				\begin{equation*}
%					dX_t = X_t(\mu_t dt + \sigma_t dW_t)
%				\end{equation*}
%				where $\mu$ and $\sigma$ are bounded and adapted processes, and $\sigma >0$ almost surely.
%				\begin{itemize}
%					\item[white] Show that $X_t\exp(-\int_0^t \mu_s ds)$ is a martingale.
%					\item[white] Find a probability $\mathbb{Q}$, equivalent to $\mathbb{P}$ under which $X$ is a martingale.
%					\item[white] Find a probability $\tilde{\mathbb{P}}$, equivalent to $\mathbb{P}$, under which the inverse process $X^{-1}$ is a martingale.
%				\end{itemize}
%			\end{itemize}
%		\end{frame}
%	\end{homework}
	
	\begin{frame}{Drift Changes Generalization}
		\begin{block}{Proposition}
			Assume the following dynamics for a $n$-vector diffusion process $X$ under a generic $N$-measure
			\begin{equation*}
				dX_t = \mu_t^N(X_t)dt + \sigma_t(X_t)CdW^N_t
			\end{equation*}
			where $dW^N_t$ is a $n$-dimensional Brownian motion with correlation matrix $C$. Under the $U$-measure, we have
			\begin{align}
				\mu^U_t(X_t) &= \mu^N_t(X_t) - \rho\sigma(X_t)\left(\frac{\sigma^N_t}{N_t}-\frac{\sigma^U_t}{U_t}\right)' \\
				CdW^U_t &= CdW^N_t + \rho\left(\frac{\sigma^N_t}{N_t}-\frac{\sigma^U_t}{U_t}\right)' dt
			\end{align}
$\rho=CC'$ is the correlation matrix of $<dW^N_i,dW^N_j>$ and $\sigma^N_t$ and $\sigma^U_t$ are the (vector) volatilities of numeraires $N$ and $U$. %(one component for each Brownian motion).
		\end{block}
	\end{frame}
	
%	\begin{frame}{Drift Changes (Proof)}
%		%We now provide a formal proof of the above proposition in the special case of \textcolor{maincolor}{$n=1$}, in which \textcolor{maincolor}{$\rho=1$}.
%		
%		Indicate by $\mathbb{Q}^N$ and $\mathbb{Q}^U$ the $N$-measure and $U$-measure. By Girsanov theorem we have the following expression for the Radon-Nikodym derivative
%		\begin{equation*}
%			\zeta_t = \frac{d\mathbb{Q}^N}{d\mathbb{Q}^U} = e^{-\frac{1}{2}\int_0^t\gamma_s^2 ds + \int_0^t\gamma_s dW_s^U}
%		\end{equation*}
%		with 
%		\begin{equation}
%			\gamma_t=\frac{[\mu^N_t(X_t)-\mu_t^U(X_t)]'}{(\sigma_t(X_t)C)'}
%			\label{eq:gamma_t}
%		\end{equation}
%		\pause
%		We also know that $\zeta_t$ is an exponential martingale hence its dynamics is such that 
%		\begin{equation}
%			d\zeta_t=\gamma_t\zeta_tdW_t^U
%			\label{eq:dzeta1}
%		\end{equation}
%	\end{frame}
%	
%	\begin{frame}{Drift Changes (Proof)}
%		By the main theorem on numeraire change \cref{eq:radon_nikodym_der2}, and using the fact that $\zeta_t$ is a $\mathbb{Q}^U$-martingale, 
%		\begin{equation}
%			\zeta_t = \frac{d\mathbb{Q}^N}{d\mathbb{Q}^U} = \frac{U_0N_t}{N_0U_t}
%			\label{eq:zeta_numeraire}
%		\end{equation}
%		thus
%		\begin{equation}
%			d\zeta_t= \frac{U_0}{N_0}d\left(\frac{N_t}{U_t}\right)= \frac{U_0}{N_0}\sigma_t^{N/U}CdW_t^U
%			\label{eq:dzeta2}
%		\end{equation}
%		where $\sigma^{N/U}_t$ is the volatility of the process $N_t/U_t$, which is also a martingale under $\mathbb{Q}^U$.
%	\end{frame}
%	
%	\begin{frame}{Drift Changes (Proof)}
%		Comparing the two results for $d\zeta_t$ (\cref{eq:dzeta1}, \cref{eq:dzeta2} and using \cref{eq:zeta_numeraire}) we get
%		\begin{equation*}
%			\begin{gathered}
%				\gamma_t\zeta_tdW_t^U = \gamma_t\frac{\cancel{U_0}N_t}{\cancel{N_0}U_t}\cancel{dW_t^U}=	\frac{\cancel{U_0}}{\cancel{N_0}}\sigma^{N/U}_tC\cancel{dW_t^U} \implies 
%				\gamma_t = \frac{U_t}{N_t}\sigma^{N/U}_tC
%			\end{gathered}
%		\end{equation*}
%		\pause
%		Using the definition of $\gamma_t$ (\cref{eq:gamma_t}) and remembering that given two matrices $A$ and $B$ it holds ($(AB)' = B'A'$)
%		\begin{equation}
%			\begin{gathered}
%				(\mu_t^N(X_t)-\mu_t^U(X_t))'= \gamma_t (\sigma_t(X_t)C)'=\frac{U_t}{N_t}\sigma^{N/U}_t CC'(\sigma_t(X_t))'\\
%				\mu_t^U(X_t)=\mu_t^N(X_t)-\frac{U_t}{N_t}\sigma_t(X_t)\rho(\sigma^{N/U}_t)'
%			\end{gathered}
%			\label{eq:gamma}
%		\end{equation}
%	\end{frame}
%	
%	\begin{frame}{Intermezzo}
%		\begin{itemize}
%			\item One of the classical formulas of differential calculus is the Leibniz rule $d(x y) = x(dy) + y(dx)$
%			\item For stochastic processes this becomes, applying It$\hat{o}$'s formula to the function $F(X,Y) = XY$
%			\begin{equation*}
%				dF(x_i)=\sum_{i=1}^n \frac{\partial F}{\partial x_i}dx_i
%				+\frac{1}{2}\sum_{i,j=1}^2 \frac{\partial^2 F}{\partial x_i \partial x_j}dx_i dx_j
%			\end{equation*}
%			For $n=2$:
%			\begin{equation*}
%				\begin{gathered}
%					\frac{\partial F}{\partial X}=Y,\frac{\partial F}{\partial Y}=X \\
%					\frac{\partial^2 F}{\partial X^2}=0,\frac{\partial^2 F}{\partial X\partial Y}=\frac{\partial^2 F}{\partial Y\partial X}=1,\frac{\partial^2 F}{\partial Y^2}=0\\
%					\\
%					d(XY) = XdY + YdX + dXdY
%				\end{gathered}
%			\end{equation*}
%		\end{itemize}
%	\end{frame}
%	
%	\begin{frame}{Drift Changes (Proof)}	
%		Now let $N_t$ and $U_t$ have dynamics under $\mathbb{Q}^U$ given by 
%		\begin{equation*}
%			\begin{gathered}
%				dN_t = (\ldots) dt + \sigma_t^NCdW^U_t\\
%				dU_t = (\ldots) dt + \sigma_t^UCdW^U_t 
%			\end{gathered}
%		\end{equation*}
%		\pause
%		From what we have just seen about the product rule in stochastic calculus
%		\begin{equation*}
%			\begin{gathered}
%				d\left(\frac{N_t}{U_t}\right)=\frac{1}{U_t}dN_t+N_td\frac{1}{U_t}+dN_td\frac{1}{U_t} \quad \left(d\frac{1}{U_t}=-\frac{1}{U^2_t}dU_t+\cancel{\frac{1}{U^3_t}dU_tdU_t}\right) \\
%			\end{gathered}
%		\end{equation*}
%		\pause
%		Replacing the dynamics for $N_t$ and $U_t$ (ignoring the terms in $dt$ since we know that $d\cfrac{N_t}{U_t}$ is a martingale)
%		\begin{equation}
%			d\left(\frac{N_t}{U_t}\right) = \frac{dN}{U_t}-\frac{N_tdU}{U^2_t}\cancel{-\frac{dNdU}{U_t^2}}=\frac{\sigma^N_t CdW^U_t}{U_t} - \frac{N_t\sigma^U_t C dW^U_t}{U^2_t}
%			\label{eq:dnu}
%		\end{equation}
%	\end{frame}
%	
%	\begin{frame}{Drift Changes (Proof)}   
%		Taking $d(N_t/U_t)$ definition from \cref{eq:dzeta2} and comparing it with \cref{eq:dnu}
%		\begin{equation}
%			d\left(\frac{N_t}{U_t}\right)=\sigma_t^{N/U}\cancel{C dW^U_t} = \frac{\sigma^N_t \cancel{CdW^U_t}}{U_t} - \frac{N_t\sigma^U_t \cancel{C dW^U_t}}{U^2_t}\implies \sigma_t^{N/U}= \frac{\sigma^N_t}{U_t} - \frac{N_t\sigma^U_t}{U^2_t}
%		\end{equation}
%		Replacing above expression for $\sigma_t^{N/U}$ into \cref{eq:gamma}
%		\begin{equation}
%			\begin{aligned}
%				\mu_t^U(X_t)&=\mu_t^N(X_t)-\frac{\cancel{U_t}}{N_t}\sigma_t(X_t)\rho\left(\frac{\sigma^N_t}{\cancel{U_t}} - \frac{N_t}{U_t^{\cancel{2}}}\sigma^U_t\right)'\\
%				&=\mu_t^N(X_t)-\sigma_t(X_t)\rho\left(\frac{\sigma^N_t}{N_t} - \frac{\sigma^U_t}{U_t}\right)'
%			\end{aligned}
%		\end{equation}
%		which proves the \textcolor{maincolor}{first part} of the statement.
%	\end{frame}
%	
%	\begin{frame}{Drift Changes (Proof)}
%		Expressing $\gamma_t$ coefficient in terms of the numeraires volatilities
%		\begin{equation*}
%			\begin{cases}
%				\gamma_t = \frac{[\mu_t^N(X_t) - \mu_t^U(X_t)]'}{(\sigma_t(X_t)C)'}\\
%				\mu_t^N(X_t) - \mu_t^U(X_t) = \sigma_t(X_t)\rho \left(\frac{\sigma^N_t}{N_t} - \frac{\sigma^U_t}{U_t}\right)'
%			\end{cases}\implies \gamma_t = \frac{\left(\frac{\sigma^N_t}{N_t} - \frac{\sigma^U_t}{U_t}\right)CC'(\sigma_t(X_t))}{C'(\sigma_t(X_t))'}
%		\end{equation*}
%		\begin{equation}
%			\gamma_t = \left(\frac{\sigma^N_t}{N_t} - \frac{\sigma^U_t}{U_t}\right)C = C'\left(\frac{\sigma^N_t}{N_t} - \frac{\sigma^U_t}{U_t}\right)'\quad(\text{$C$ is a symmetric matrix})    
%			\label{eq:gamma_3}
%		\end{equation}
%		\pause
%		Finally from the Girsanov theorem we get the diffusion process under the new numeraire
%		\begin{equation}
%			\begin{gathered}
%				CdW^N_t = CdW^U_t - C\gamma_t dt \\
%				CdW^N_t = CdW^U_t - \rho\left(\frac{\sigma^N_t}{N_t}-\frac{\sigma^U_t}{U_t}\right)' dt
%			\end{gathered}
%			\label{eq:girsanov_shock_extended}
%		\end{equation}
%		\myendproof
%		
%		which proves also the \textcolor{maincolor}{second part} of the proposition.
%		
%		%As an exercise, once you know the Vasicek short rate model, try to determine the new drift when moving from bank account to forward meeasure.%the result is an application of the previous formula with $X = r$, $Q = P^T$ , $\sigma(X_t,t) = \sigma$, $\sigma_B (t) = -A(t, T)\sigma$ and $m(X_t,t) = a(b-rt)$.
%	\end{frame}
	












%	\begin{frame}{Asset/Numeraire by Girsanov}
%		Assuming an asset $S$ and a numeraire $N$ with the following evolutions under the risk-neutral measure
%		\begin{equation}
%			\begin{cases}
%				\cfrac{dS_t}{S_t} = r_tdt + \sigma^S_t dW^B_t\quad\text{(asset)} \\
%				\cfrac{dN_t}{N_t} = r_tdt + \sigma^N_t dW^B_t\quad\text{(numeraire)}
%			\end{cases}
%			\label{eq:S_N_dynamics}
%		\end{equation}
%		
%		by Girsanov Theorem (\cref{eq:girsanov_shock_extended}), under $\mathbb{Q}^N$, we get
%		\begin{equation}
%			dW^N_t = dW_t^B - \sigma_t^N dt
%			\label{eq:girsanov_ex}
%		\end{equation}
%		which is a Brownian motion.
%	\end{frame}
%	
%	\begin{frame}{Asset/Numeraire by Girsanov}
%		Now let's apply It$\hat{o}$'s lemma to $S_t/N_t$
%		\begin{equation*}
%			\begin{aligned}
%				d\left(\frac{S}{N}\right) &= \frac{1}{S}dS - \frac{S}{N^2}dN + \frac{S}{N^3}dN^2-\frac{1}{N^2}dSdN = \frac{S}{N}\left(\frac{dS}{S}-\frac{dN}{N}+\frac{dN^2}{N^2}-\frac{dSdN}{SN} \right) = \\
%				&=\frac{S}{N}\left(rdt+\sigma^S dW^B - rdt - \sigma^N dW^B + (\sigma^N)^2 dt - \sigma^S\sigma^N dt \right) = \quad\textit{(with \cref{eq:S_N_dynamics})}\\
%				&=\frac{S}{N}((\sigma^N)^2 - \sigma^S\sigma^N) dt + \sigma^S dW^B - \sigma^N dW^B = \quad\textit{(with \cref{eq:girsanov_ex})}\\
%				&=\frac{S}{N}((\sigma^N)^2 - \sigma^S\sigma^N) dt + \sigma^S (dW^N + \sigma^N dt) - \sigma^N(dW^N+\sigma^Ndt) = \\
%				&=\frac{S}{N}(\sigma^S - \sigma^N)dW^N 
%			\end{aligned}
%		\end{equation*}
%		which shows that $\cfrac{S}{N}$ is a $\mathbb{Q}^N$-martingale (no drift in dynamics).
%	\end{frame}

	\begin{frame}{The Vasicek Model under $\mathbb{P}$}
		\begin{itemize}
			\item Statistical analysis of historical interest rate data suggests that rates are mean-reverting. Under the physical measure $\mathbb{P}$, we model the short rate $r_t$ as:
			\begin{equation*}
			dr_t = a(b-rt)dt + \sigma W_t^\mathbb{P}
			\end{equation*}
			where $a$ speed of mean reversion, $b$ long-term mean level, $\sigma$ volatility, and $W_t^\mathbb{P}$ Brownian motion under the real-world measure.
			\item If we use this SDE to price a bond via $\E^\mathbb{P}[\exp(-\int r_s ds)]$, we are assuming investors are risk-neutral towards interest rate volatility.
			\item In reality the bond is a risky asset and investors will not buy it unless the price is lower than the mathematical average of discounted cash-flows.
			\item \textbf{Investors demand a premium for bearing interest rate risk}.
		\end{itemize}
	\end{frame}
	
	\begin{frame}{Market Price of Risk}
		\begin{itemize}
			\item The \emph{Market Price of Risk} $\lambda$ represents the extra return (drift) investors demand per unit of risk they take on.
			\item Girsanov’s Theorem allows us to mathematically "subtract" this compensation from the real-world drift, leaving us with a process that can be priced as if the world were risk-neutral
			\begin{equation*}
				dW_t^\mathbb{Q} = dW_t^\mathbb{P} + \lambda dt \implies dW_t^\mathbb{P} = dW_t^\mathbb{Q} - \lambda dt	
			\end{equation*}
			\item Substituting this into our $\mathbb{P}$-dynamics:
			\begin{equation*}
				dr_t = a(b-rt)dt + \sigma(dW_t^\mathbb{Q}-\lambda dt)
			\end{equation*}
			
		\end{itemize}
	\end{frame}
	
	\begin{frame}{The Risk-Neutral Dynamics}
		\begin{itemize}
			\item Grouping the $dt$ terms from the previous slide:
			\begin{equation*}
				dr_t=[a(b-rt)-\sigma\lambda]dt+\sigma dW_t^\mathbb{Q}
			\end{equation*}
			\item To maintain the mean-reverting structure $dr_t=a^*(b^*-rt)dt+\sigma dW_t^\mathbb{Q}$, we distribute the terms:
			\begin{equation*}
				dr_t=[ab-ar_t-\sigma\lambda]dt+\sigma dW_t^\mathbb{Q} =\overbrace{a}^{a^*} \bigg[\overbrace{\left(b-\frac{\sigma\lambda}{a}\right)}^{b^*}-rt\bigg]dt+\sigma dW_t^\mathbb{Q}
			\end{equation*}
			where the speed of reversion remains unchanged, and the long-term mean $b^* = b - \frac{\sigma \lambda}{a}$ is shifted.
			\item If $\lambda < 0$ (usual case for risk-aversion in bonds), then $b^* > b$.
			The market "prices" bonds as if the long-term interest rate is higher than what we see in historical data. 
			This compensates the bond buyer for the risk of rising rates.			
		\end{itemize}
	\end{frame}
	
	\begin{frame}{An Example}
		\begin{columns}
		\column{0.4\linewidth}
		\begin{itemize}		
		\item \textcolor{blue}{The Blue Line ($\mathbb{P}$)}:"Best Guess" based on history: last 20 years of data, rates to return to 8\%.
		\item \textcolor{red}{The Red Line ($\mathbb{Q}$)}: "Pricing Guess" investors are afraid of interest rate, they price bonds as if rates are at 7.4\%.
		\item The vertical distance between the blue and red lines is where the Market Price of Risk lives.
		\end{itemize}
		\column{0.6\linewidth}
		\includegraphics[width=0.9\linewidth]{images/vasicek_under_P}
		\end{columns}
		\vfill
		We aren't changing the math of mean reversion; we are just changing the destination ($b\to b^*$) to ensure our model matches the market prices.
	\end{frame}
	
	\begin{frame}{Why Girsanov is Essential Here}
	\begin{itemize}
		\item Without Girsanov we would simply know that \emph{prices are expectations}, but we wouldn't know which interest rate process to simulate.
		\item With Girsanov we have a specific recipe. We take our historical $b$, subtract the risk adjustment $\frac{\sigma\lambda}{a}$, and we have a ready-to-use SDE for Monte Carlo or the Riccati equations.
		\item The Radon-Nikodym derivative that makes this happen is:
		\begin{equation*}
		\frac{d\mathbb{P}}{d\mathbb{Q}}=\exp\left(-\frac{1}{2}\int_0^T\lambda^2 dt - \int_0^T\lambda dW_t^\mathbb{P}\right)
		\end{equation*}
	\end{itemize}
	
	
\end{frame}









%	\begin{frame}{Quanto Options: Motivation and Setup}
%	\begin{itemize}
%		\item A Quanto is a derivative where the underlying asset is denominated in a \textcolor{blue}{foreign currency}, but the payoff is paid in a \textcolor{red}{domestic currency} at a fixed exchange rate $X_{fix}$.
%		\item The assets:
%		\begin{itemize}
%			\item $S_t^f$: foreign asset price (in foreign currency);
%			\item $X_t$: exchange rate (units of domestic per unit of foreign);
%			\item $B_d(t)$, $B_f(t)$: domestic and foreign money market accounts.
%		\end{itemize}
%		\item The foreign asset $S_t^f$ is a martingale under the foreign risk-neutral measure $\mathbb{Q}^f$ (when divided by $B_f$). However, the domestic investor lives in the $\mathbb{Q}^d$ world.
%		\item To price the asset in the domestic world, we must use the domestic bank account $B_d(t)$ as numeraire, but the asset's value in domestic terms is $S_t^d = X_t S_t^f$.
%	\end{itemize}
%\end{frame}
%
%\begin{frame}{The Quanto Drift Adjustment}
%	\begin{itemize}
%		\item Under the domestic measure $\mathbb{Q}^d$, the process for the foreign asset $S^f$ is:
%		\begin{equation*}
%			\frac{dS^f_t}{S^f_t} = (r_f \textcolor{red}{-\rho\sigma_S \sigma_X} ) dt + \sigma_S dW^{d}_t
%		\end{equation*}
%		\item \textbf{Derivation via Numeraire Change:}
%		\begin{enumerate}
%			\item We move from $\mathbb{Q}^f$ (numeraire $B_f$) to $\mathbb{Q}^d$ (numeraire $B_d$).
%			\item The Radon-Nikodym derivative is $\frac{\mathbb{Q}^d}{\mathbb{Q}^f}=\frac{B_d(T)X(0)}{B_f(T)X(T)}$.
%			\item By Girsanov's Theorem, the drift of $S_f$ is adjusted by the covariance between the asset and the exchange rate.
%		\end{enumerate}
%		\item \textbf{Economic Intuition:} The term $-\rho\sigma_S\sigma_X$ is the \textcolor{red}{Quanto Adjustment}. If the foreign currency strengthens when the stock rises ($\rho > 0$), the domestic investor "loses" the currency gain due to the fixed exchange rate, so the expected return must be adjusted downwards.
%	\end{itemize}
%\end{frame}
%	
%
%	\begin{frame}{The Radon-Nikodym Derivative}
%	
%	The Domestic investor uses $B_d(t)$ as a numeraire. Any asset $A_t$​ denominated in domestic currency must eb a\Qd-martingale when discounted by Bd​. The domestic value of the foreign bank account is Bf​(t)Xt​. Therefore, the Radon-Nikodym derivative ξt​=dQfdQd​​Ft​​ is:
%	ξt​=Bf​(t)Xt​/(Bf​(0)X0​)Bd​(t)/Bd​(0)​=erf​t(Xt​/X0​)erd​t​
%	
%	Applying Itô's Lemma to ξt​=Xt​X0​​e(rd​−rf​)t, we find the dynamics of the density process:
%	dξt​=−σX​ξt​dW2,tf​
%	
%	Crucial Step: This identifies the Girsanov kernel. The "volatility" of the RN derivative is −σX​ attached to the exchange rate's noise.
%\end{frame}
%\begin{frame}{Applying Girsanov's Theorem}
%	
%	Shifting the Wind Girsanov tells us that to get the Brownian motions under Qd, we must subtract the "covariance" between the asset's noise and the density process noise.
%	
%	Under Qd, the Brownian motions become:
%	
%	For W2​ (Exchange Rate): dW2,td​=dW2,tf​−(−σX​)dt=dW2,tf​+σX​dt.
%	
%	For W1​ (Asset): Since W1​ and W2​ are correlated by ρ, the shift is:
%	dW1,td​=dW1,tf​−Cov(dW1f​,ξdξ​)
%	dW1,td​=dW1,tf​−(−ρσX​)dt=dW1,tf​+ρσX​dt
%	
%	The result: dW1,tf​=dW1,td​−ρσX​dt.
%\end{frame}
%\begin{frame}{The Final Quanto SDE}
%	
%	The "Convexity" Adjustment Substitute dW1,tf​ back into the foreign asset SDE:
%	Stf​dStf​​=rf​dt+σS​(dW1,td​−ρσX​dt)
%	Stf​dStf​​=(rf​−ρσS​σX​)dt+σS​dW1,td​
%	
%	Why did the drift change?
%	
%	The domestic investor is exposed to Sf through the product Xt​Stf​.
%	
%	In a domestic risk-neutral world, the product Xt​Stf​ must grow at rate rd​.
%	
%	Because Xt​ and Stf​ are correlated, their product has an extra "covariance" term in its growth (d(XS)=XdS+SdX+dXdS).
%	
%	If we fix the exchange rate (Xfix​), we strip away the currency's volatility but we are left with the covariance residue.
%\end{frame}
%\begin{frame}{Mathematical Example for the Exam}
%	
%	The Forward Quanto Price Calculate the Forward price of a Quanto asset FQ​=EQd[STf​]. Since the drift of Sf under Qd is rq​=rf​−ρσS​σX​, the expectation is:
%	FQ​=S0f​e(rf​−ρσS​σX​)T
%	
%	Comparison:
%	
%	Standard Forward (Foreign): F=S0f​erf​T
%	
%	Quanto Forward: FQ​=F⋅e−ρσS​σX​T
%	
%	Student takeaway: If ρ>0 (Stock and Currency rise together), the Quanto Forward is lower than the standard Forward. You are "penalized" for the fact that you don't benefit from the currency strengthening when your stock does well.
%	Suggestions for "Challenging" additions:
%	
%	The "Reverse" Quanto: Ask students to derive the adjustment for a domestic asset priced in a foreign currency. (The sign of ρ flips).
%	
%	Implied Correlation: Since σS​ and σX​ are observable from options, but ρ is not, how can we use Quanto prices to "extract" the market's view on the correlation between Nikkei 225 and USD/JPY?
%	
%	Would you like me to create a slide on "Siegel's Paradox"? It’s a classic companion to Quanto theory that explains why 1/X doesn't have the same drift as X.
%	
%\end{frame}


%%	Slide 1: Siegel’s Paradox - Il Problema della Simmetria
%%	
%%	La Domanda: Se il tasso di cambio Euro/Dollaro (Xt​) è una martingala sotto una certa misura, il tasso Dollaro/Euro (1/Xt​) può essere anch'esso una martingala?
%%	
%%	La Risposta Matematica: No. A causa della convessità della funzione f(x)=1/x, per la Disuguaglianza di Jensen:
%%	E[XT​1​]>E[XT​]1​
%%	
%%	Significato Finanziario: Se entrambi gli investitori (USA e EU) fossero neutrali al rischio e convinti che il cambio non abbia drift, entrambi si aspetterebbero paradossalmente di guadagnare nel lungo termine detenendo la valuta estera.
%%	Slide 2: Derivazione tramite il Lemma di Itô
%%	
%%	Supponiamo che sotto la misura domestica Qd, il cambio segua una dinamica GBM (senza drift per semplicità di calcolo):
%%	Xt​dXt​​=0⋅dt+σX​dWt​
%%	
%%	Applichiamo il Lemma di Itô a Yt​=1/Xt​:
%%	
%%	f(X)=1/X⟹f′=−1/X2,f′′=2/X3
%%	
%%	dYt​=−Xt2​1​(dXt​)+21​Xt3​2​(dXt​)2
%%	
%%	dYt​=−Xt2​1​(σX​Xt​dWt​)+Xt3​1​(σX2​Xt2​dt)
%%	
%%	Risultato:
%%	Yt​dYt​​=σX2​dt−σX​dWt​
%%	
%%	L'intuizione: Anche se X non ha drift, il suo inverso 1/X ha un drift positivo pari alla sua varianza σX2​. Questo è il "volatility smile" dei cambi in forma pura.
%%	Slide 3: Girsanov e la Coerenza dei Mercati
%%	
%%	Perché non c'è arbitraggio? Potrebbe sembrare che tutti possano guadagnare semplicemente scambiando valuta. Girsanov risolve il paradosso imponendo un cambio di misura obbligatorio quando cambiamo prospettiva.
%%	
%%	Sotto Qd (Investitore Domestico), il cambio X (prezzo del foreign) ha drift rd​−rf​.
%%	
%%	Quando passiamo all'investitore straniero (Qf), non cambiamo solo l'asset, cambiamo la misura di probabilità.
%%	
%%	Il passaggio da Qd a Qf (visto nella slide del Quanto) introduce un termine di deriva che cancella esattamente il surplus di rendimento dovuto alla convessità (σX2​), rendendo i mercati internazionali coerenti.
%%	Slide 4: Applicazione Pratica - Forward FX
%%	
%%	In un mondo risk-neutral, il tasso Forward è definito come:
%%	Ft,T​=EtQd​[XT​]
%%	
%%	Se uno studente provasse a calcolare il forward inverso come 1/Ft,T​, commetterebbe un errore sistematico.
%%	
%%	Ft,T​=Xt​e(rd​−rf​)(T−t)
%%	
%%	EtQf​[1/XT​]=(1/Xt​)e(rf​−rd​)(T−t)
%%	
%%	Il paradosso di Siegel spiega perché, nei mercati reali, le opzioni out-of-the-money sulle valute costano in modo diverso a seconda della valuta di riferimento (il cosiddetto Vanna-Volga pricing).
%%	Riassunto per gli Studenti: Il "Big Picture"
%%	
%%	Girsanov è il "traduttore" tra due mondi (valute).
%%	
%%	Il Quanto Adjustment è il prezzo che paghi per bloccare il traduttore (fissare il cambio).
%%	
%%	Il Paradosso di Siegel è la prova che non puoi tradurre semplicemente invertendo i numeri; devi invertire l'intera distribuzione di probabilità.

\section{Convexity}
\begin{frame}{Convexity Correction}
	\begin{itemize}
		\item<1-> In financial lingo, convexity is a broadly understood and often non-specific term for nonlinear behavior of the price of an instrument as a function of evolving markets.
		\item<2-> The concept of \emph{convexity adjustment} is required for all asset classes when there are payment delays or when the payment dates of the contract do
		not correspond to those used in the numeraire defintion. %Generally, if we have a maturity date $T$ but payment takes place at time $T + τ∗$ , convexity has to be taken into account. 
		\item<3-> From the perspective of financial modeling they arise as the results of valuation done under the \emph{"wrong"} measure.
		\item<4-> The higher the uncertainty in the market (high volatility) the more pronounced the effect of the convexity will become.
	\end{itemize}
\end{frame}

\begin{frame}{The Good\ldots}
	\begin{itemize}
		\item<1-> Let us consider a basic interest rate payoff function, whose payoff depends on the EURIBOR rate
		\begin{equation*}
			V(t_0) = N D(t_0)\expect{Q}\left[\cfrac{F(T_{i-1};T_{i-1},T_i)}{D(T_i)}\right]=
			NP(t_0,T_i)\mathbb{E}^{T_i}[F(T_{i-1};T_{i-1},T_i)]
		\end{equation*}
		\item<2-> Since $F_{i}(T_{i-1}) = F(T_{i-1};T_{i-1}, T_i)$ is a martingale under the $T_i$-forward measure, we get
		\begin{equation*}
			V(t_0)= NP(t_0,T_i)F_i(t_0)
		\end{equation*}
	\end{itemize}
\end{frame}

\begin{frame}{\ldots and the Bad}
	\begin{itemize}
		\item<1-> Imagine now a similar contract where, however, the payments will take place at some earlier time $T_{i-1} < T_i$. \item<2->Its current value is given by
		\begin{equation*}
			V(t_0) = N D(t_0)\expect{Q}\left[\cfrac{F_i(T_{i-1})}{D(T_{i-1})}\right]
		\end{equation*}
		\item<3-> When changing measure (to the $T_{i-1}$-forward measure) we get the following Radon-Nikodym derivative:
		\begin{equation*}
			\cfrac{d\mathbb{Q}^{T_{i-1}}}{d\mathbb{Q}}=\cfrac{P(T_{i-1},T_{i-1})D(t_0)}{P(t_0,T_{i-1})D(T_{i-1})}
		\end{equation*}
		\item<4->So that
		\begin{equation*}
			V(t_0) = N D(t_0)\expect{Q}\left[\cfrac{F_i(T_{i-1})}{D(T_{i-1})}\right] = N D(t_0)\mathbb{E}^{T_{i-1}}\left[\cfrac{P(t_0,T_{i-1})D(T_{i-1})}{P(T_{i-1},T_{i-1})D(t_0)}\cfrac{F_i(T_{i-1})}{D(T_{i-1})}\right] 
		\end{equation*}
	\end{itemize}
\end{frame}

\begin{frame}{The Correction}
	\begin{itemize}
		\item<1-> After few semplifications
		\begin{equation*}
			\begin{aligned}
				V(t_0) = N \cancel{D(t_0)}\mathbb{E}^{T_{i-1}}&\left[\cfrac{P(t_0,T_{i-1})\cancel{D(T_{i-1})}}{P(T_{i-1},T_{i-1})\cancel{D(t_0)}}\cfrac{F_i(T_{i-1})}{\cancel{D(T_{i-1})}}\right] = \\
				&= N P(t_0,T_{i-1})\mathbb{E}^{T_{i-1}}[F_i(T_{i-1})]
			\end{aligned}
		\end{equation*}
		\item<1-> Although the Libor rate is a martingale under the $T_i$-forward
		measure, it is however not a martingale under the $T_{i-1}$-forward measure, i.e.,
		\begin{equation*}
			\mathbb{E}^{T_{i-1}}\left[F_i(T_{i-1})\right]\neq \mathbb{E}^{T_i}\left[F_i(T_{i-1})\right] = F_i(t_0)
		\end{equation*}
		\item<2-> The difference between these two expectations is commonly referred to as a \textbf{convexity}.
	\end{itemize}
\end{frame}

\begin{frame}{Changing Measures}
	\begin{itemize}
		\item<1-> By the change of measure technique, we can simplify the expressions above, to some extent. 
		\item<2-> Indeed moving to the $T_i$-forward measure, we find:
		\begin{equation*}
			\cfrac{d\mathbb{Q}^{T_i}}{d\mathbb{Q}^{T_{i-1}}}=\cfrac{P(T_{i-1},T_i)P(t_0,T_{i-1})}{P(t_0,T_i)P(T_{i-1},T_{i-1})}
		\end{equation*}
		\item<3-> The derivative value is then equal to
		\only<3-3>{\begin{equation*}
				\begin{aligned}
					V(t_0) &= N P(t_0,T_{i-1})\mathbb{E}^{T_{i-1}}[F_i(T_{i-1})] = \\
					&=N P(t_0,T_{i-1})\mathbb{E}^{T_i}\left[F_i(T_{i-1})\cfrac{P(t_0,T_i)P(T_{i-1},T_{i-1})}{P(T_{i-1},T_i)P(t_0,T_{i-1})}\right]=\\
				\end{aligned}
		\end{equation*}}
		\only<4-4>{\begin{equation*}
				\begin{aligned}
					V(t_0) &= N P(t_0,T_{i-1})\mathbb{E}^{T_{i-1}}[F_i(T_{i-1})] = \\
					&=N \cancel{P(t_0,T_{i-1})}\mathbb{E}^{T_i}\Bigg[F_i(T_{i-1})\cfrac{P(t_0,T_i)\overbrace{P(T_{i-1},T_{i-1})}^{=1}}{P(T_{i-1},T_i)\cancel{P(t_0,T_{i-1})}}\Bigg]=\\
				\end{aligned}
		\end{equation*}}
		\only<5->{\begin{equation*}
				\begin{aligned}
					V(t_0) &= N P(t_0,T_{i-1})\mathbb{E}^{T_{i-1}}[F_i(T_{i-1})] = \\
					&=N \mathbb{E}^{T_i}\left[F_i(T_{i-1})\cfrac{P(t_0,T_i)}{P(T_{i-1}, T_i)}\right]\\
				\end{aligned}
		\end{equation*}}
	\end{itemize}
\end{frame}

\begin{frame}{The Correction}
	\begin{itemize}
		\item<1-> By simply adding and subtracting $F_i(T_{i-1})$ can be rewritten as 
		\begin{equation*}
			\begin{aligned}
				V(t_0) &= N \mathbb{E}^{T_i}[F_i(T_{i-1})] + N \mathbb{E}^{T_i}\left[F_i(T_{i-1})\left(\cfrac{P(t_0,T_i)}{P(T_{i-1}, T_i)}-1\right)\right]\\
				&=N\left(F_i(t_0) + \mathbb{E}^{T_i}\left[F_i(T_{i-1})\left(\cfrac{P(t_0,T_i)}{P(T_{i-1}, T_i)}-1\right)\right]\right)
			\end{aligned}
		\end{equation*}
		\item<2-> The convexity correction can finally be expressed as
		\begin{equation}
			\boxed{	
				P(t_0,T_i)\mathbb{E}^{T_i}\left[\cfrac{F_i(T_{i-1})}{P(T_{i-1},T_i)}\right]-F_i(t_0)}
		\end{equation}
	\end{itemize}
\end{frame}

\begin{frame}{Final Remark}
	\begin{itemize}
		\item<1-> From the definition of EURIBOR rate: $P(T_{i-1},T_i)=\cfrac{1}{1+\tau_i F_i(T_{i-1})}$ hence
		\begin{equation*}
			\mathbb{E}^{T_i}\left[\cfrac{F_i(T_{i-1})}{P(T_{i-1},T_i)}\right]=F_i(t_0)+\tau_i\mathbb{E}^{T_i}[F_i^2(T_{i-1})]
		\end{equation*}
		\item<2-> Note that although $F_i(T_{i-1})$ is a martingale under the $T_i$-forward measure, \textbf{its square is not}.
		\item<3-> Indeed applying \ito~lemma to $F_i^2$ with the  dynamics $dF_i(t) = \sigma F_i(t)dW(t)$, give us
		\begin{equation*}
			dF_i^2=\sigma^2F_i^2dt+2\sigma F_i^2dW
		\end{equation*}
		which has a drift term, so it is not a martingale.
	\end{itemize}
\end{frame}

\subsection{In Arrears Swap}
\begin{frame}{In-Arrears Swap}
	\begin{itemize}
		\item<1-> An \emph{in-arrears swap} is an IRS that resets and pays at the same dates $\{T_{\alpha+1},\ldots, T_\beta\}$ and with fixed-leg rate $K$. More precisely, the value of an payer IAS is
		\begin{equation}
			\textbf{IAS}=\expect{Q}\left[\sum_{i=\alpha+1}^{\beta}D(0,T_i)\tau_{i+1}(F_{i+1}(T_i)-K)\right]
		\end{equation}
		\item<2-> To compute the expectation we can proceed as follows
		\begin{equation*}
			\begin{aligned}
				\expect{Q}\Bigg[\sum_{i=\alpha+1}^{\beta}D(0,T_i)\tau_{i+1}&(F_{i+1}(T_i)-K)\Bigg] = \\ 
				&=\expect{Q}\left[\sum_{i=\alpha+1}^{\beta}D(0,T_i)\left(\cfrac{1}{P(T_i,T_{i+1})}-(1+\tau_{i+1}K)\right)\right]=\ldots
			\end{aligned}
		\end{equation*}
	\end{itemize}
\end{frame}

\begin{frame}{In-Arrears Swap}
	\begin{itemize}
		\item<1-> To further simplify the expression it is convenient to use the following result from the \emph{tower property} of the expectation:
		\begin{equation*}
			\begin{aligned}
				&\mathbb{E}_t\left[\cfrac{XD(t,S)}{P(T,S)} \right]=\mathbb{E}_t\left[\mathbb{E}_T\left[\cfrac{XD(t,T)D(T,S)}{P(T,S)} \right]\right]=\mathbb{E}_t\left[\cfrac{XD(t,T)}{P(T,S)}\mathbb{E}_T[D(T,S)]\right]= \\ 
				& =\mathbb{E}_t\left[\cfrac{XD(t,T)}{P(T,S)}P(T,S)\right]=\mathbb{E}_t[XD(t,T)],\quad (t<T<S)
			\end{aligned}
		\end{equation*}
		\item<2-> The previous expression then becomes (with $T=T_i$ and $S=T_{i+1}$):
		\begin{equation*}
			\begin{aligned}
				\ldots=\expect{Q}\Bigg[\sum_{i=\alpha+1}^{\beta}&\left(\cfrac{D(0,T_i)}{P(T_i,T_{i+1})}-D(0,T_i)(1+\tau_{i+1}K)\right)\Bigg] = \\
				&=\expect{Q}\left[\sum_{i=\alpha+1}^{\beta}\left(\cfrac{D(0,T_{i+1})}{P(T_i,T_{i+1})^2}-D(0,T_i)(1+\tau_{i+1}K)\right)\right] = \ldots
			\end{aligned}
		\end{equation*}
	\end{itemize}
\end{frame}

\begin{frame}{In-Arrears Swap}
	\begin{itemize}
		\item<1-> From the linearity of the expectation operator:
		\begin{equation*}
			\begin{aligned}
				\ldots &=\sum_{i=\alpha+1}^{\beta}\left(\expect{Q}\left[\cfrac{D(0,T_{i+1})}{P(T_i,T_{i+1})^2}\right]-\expect{Q}\left[D(0,T_i)(1+\tau_{i+1}K)\right]\right) = \\
				&=\sum_{i=\alpha+1}^{\beta}\left(\expect{Q}\left[\cfrac{D(0,T_{i+1})}{P(T_i,T_{i+1})^2}\right]-P(0,T_i)(1+\tau_{i+1}K)\right) = \ldots\\
			\end{aligned}
		\end{equation*}
		the second one follows from the ZCB definition.
		\item<2-> Before computing the remaining expectation let's make a change of measure from the risk-neutral to $P(\cdot, T_{i+1})$
		\only<2-2>{
			\begin{equation*}
				\ldots=\sum_{i=\alpha+1}^{\beta}\Bigg(\mathbb{E}^{T_{i+1}}\Bigg[\cfrac{D(0,T_{i+1})}{P(T_i,T_{i+1})^2}\cfrac{D(T_{i+1},T_{i+1})P(0,T_{i+1})}{D(0,T_{i+1})P(T_{i+1},T_{i+1})}\Bigg]-P(0,T_i)(1+\tau_{i+1}K)\Bigg)
				%&=\sum_{i=\alpha+1}^{\beta}\left[P(0,T_{i+1})\mathbb{E}^{i+1}\left[\cfrac{1}{P(T_i,T_{i+1})^2}\right]-P(0,T_i)(1+\tau_{i+1}K)\right] = \\
				%&=\sum_{i=\alpha+1}^{\beta}\left[P(0,T_{i+1})\mathbb{E}^{i+1}\left[(1+\tau_{i+1}F_{i+1}(T_i))^2\right]-P(0,T_i)(1+\tau_{i+1}K)\right]
				%\end{aligned}
		\end{equation*}}
		\only<3->{
			\begin{equation*}
				\ldots=\sum_{i=\alpha+1}^{\beta}\Bigg(\mathbb{E}^{T_{i+1}}\Bigg[\cfrac{\cancel{D(0,T_{i+1})}}{P(T_i,T_{i+1})^2}\cfrac{\overbrace{D(T_{i+1},T_{i+1})}^{=1}P(0,T_{i+1})}{\cancel{D(0,T_{i+1})}\underbrace{P(T_{i+1},T_{i+1})}_{=1}}\Bigg]-P(0,T_i)(1+\tau_{i+1}K)\Bigg)
		\end{equation*}}
	\end{itemize}
\end{frame}

\begin{frame}{In-Arrears Swap}
	\begin{itemize}
		\item<1-> Finally
		\begin{equation}
			\textbf{IAS}=\sum_{i=\alpha+1}^{\beta}\Bigg(P(0,T_{i+1})\mathbb{E}^{T_{i+1}}\Bigg[\cfrac{1}{P(T_i,T_{i+1})^2}\Bigg]-P(0,T_i)(1+\tau_{i+1}K)\Bigg)
			\label{eq:exp_ias}
		\end{equation}
		\item<2-> After the last substitution $\frac{1}{P(T_i,T_{i+1})^2}=(1+\tau_{i+1}F_{i+1}(T_i))^2$, it is relatively easy to compute the expected value in Eq.~\ref{eq:exp_ias}, since under $\mathbb{Q}^{T_{i+1}}$, $F_{i+1}$ is a martingale
		\begin{equation*}
			dF_{i+1}(t) = \sigma_{i+1}(t)F_{i+1}(t)dW^{i+1}(t)
		\end{equation*}
		\uncover<3->{so that (using \ito~isometry), 
			\begin{equation*}
				\mathbb{E}^{T_{i+1}}[F_{i+1}^2(T_i)]=F_{i+1}^2(0)\exp\left(\int_0^{T_i}\sigma_{i+1}^2(t)dt\right) = F_{i+1}^2(0)\exp(v_{i+1}^2)
			\end{equation*}
			where the $v$ can be determined from cap market prices.}
	\end{itemize}
\end{frame}

\begin{frame}{In-Arrears Swap}
	\begin{itemize}
		\item<1-> The final expression is
		\begin{equation}
			\begin{gathered}
				\textbf{IAS} = \sum_{i=\alpha+1}^{\beta}(P(0,T_{i+1})\mathbb{E}^{T_{i+1}}[(1+\tau_{i+1}F_{i+1}(T_i))^2]-P(0,T_i)(1+\tau_{i+1}K)) \\
				=\sum_{i=\alpha+1}^{\beta}[
				P(0,T_{i+1}) [1+2\tau_{i+1}F_{i+1}(0) + \tau_{i+1}^2 F_{i+1}^2(0)\exp(v_{i+1}^2)]\\
				-(1+\tau_{i+1}K)P(0,T_i)] \\
			\end{gathered}
			\label{eq:ias_price}
		\end{equation}
		\item<2-> From~\cref{eq:ias_price} it is apparent how the valuation results in a vanilla swap-like part plus the convexity correction due to the arrears feature.
		\item<3-> Also, contrary to the plain-vanilla case, the In-Arrear Swap value depends on the volatility of forward rates through the caplet volatilities $v$. 
		\item<4-> Notice however that correlations between different rates are not involved in this product, as expected by the nature of the contract.
		
		%Remark 13.1.1. If caplet prices are available in the market for each maturity
		%Ti and strike K, we can price an in-arrears swap consistently with %theobserved caplet smile by noting that we can write
	\end{itemize}
\end{frame}


\end{document}
