\documentclass[12pt,a4paper]{exam}

\usepackage[utf8]{inputenc}
\usepackage[T1]{fontenc}
\usepackage{amsmath}
\usepackage{amsfonts}
%\usepackage{amssymb}
\usepackage{graphicx}
\usepackage{geometry}
\usepackage{cancel}
\usepackage{enumitem}

\geometry{a4paper, margin=2cm}

\newcommand{\ito}{It$\hat{o}$}
\newcommand{\expect}[1]{\mathbb{E}^\mathbb{#1}}
\newcommand{\expectt}[2]{\mathbb{E}_{#2}^\mathbb{#1}}

\usepackage{cprotect}

\usepackage{xcolor}
\definecolor{maroon}{cmyk}{0, 0.87, 0.68, 0.32}
\definecolor{halfgray}{gray}{0.55}
\definecolor{ipython-frame}{RGB}{207, 207, 207}
\definecolor{ipython-bg}{RGB}{247, 247, 247}
\definecolor{ipython-red}{RGB}{186, 33, 33}
\definecolor{ipython-green}{RGB}{0, 128, 0}
\definecolor{ipython-cyan}{RGB}{64, 128, 128}
\definecolor{ipython-purple}{RGB}{170, 34, 255}
\usepackage{listings}
\lstdefinelanguage{iPython}{
	morekeywords={access,and,del,except,exec,in,is,lambda,not,or,raise},
	morekeywords=[2]{for,print,abs,all,any,basestring,bin,bool,bytearray,callable,chr,classmethod,cmp,compile,complex,delattr,dict,dir,divmod,enumerate,eval,execfile,file,filter,float,format,frozenset,getattr,globals,hasattr,hash,help,hex,id,input,int,isinstance,issubclass,iter,len,list,locals,long,map,max,memoryview,min,next,object,oct,open,ord,pow,property,range,reduce,reload,repr,reversed,round,set,setattr,slice,sorted,staticmethod,str,sum,super,tuple,type,unichr,unicode,vars,xrange,zip,apply,buffer,coerce,intern,elif,else,if,continue,break,while,class,def,return,try,except,import,finally,try,except,from,global,pass, True, False},
	sensitive=true,
	morecomment=[l]\#,%
	morestring=[b]',%
	morestring=[b]",%
	moredelim=**[is][\color{black}]{@@}{@@},
	%%
	%morestring=[s]{'''}{'''},% used for documentation text (mulitiline strings)
	%morestring=[s]{"""}{"""},% added by Philipp Matthias Hahn
	%%
	%morestring=[s]{r'}{'},% `raw' strings
	%morestring=[s]{r"}{"},%
	%morestring=[s]{r'''}{'''},%
	%morestring=[s]{r"""}{"""},%
	%morestring=[s]{u'}{'},% unicode strings
	%morestring=[s]{u"}{"},%
	%morestring=[s]{u'''}{'''},%
	%morestring=[s]{u"""}{"""}%
	%
	% {replace}{replacement}{lenght of replace}
	% *{-}{-}{1} will not replace in comments and so on
	%literate=
	%{\%}{{{\color{ipython-purple}+}}}1,
	%{�}{{\'a}}1 {�}{{\'e}}1 {�}{{\'i}}1 {�}{{\'o}}1 {�}{{\'u}}1,
	%{�}{{\'A}}1 {�}{{\'E}}1 {�}{{\'I}}1 {�}{{\'O}}1 {�}{{\'U}}1
	%{�}{{\`a}}1 {�}{{\`e}}1 {�}{{\`i}}1 {�}{{\`o}}1 {�}{{\`u}}1
	%{�}{{\`A}}1 {�}{{\'E}}1 {�}{{\`I}}1 {�}{{\`O}}1 {�}{{\`U}}1
	%{�}{{\"a}}1 {�}{{\"e}}1 {�}{{\"i}}1 {�}{{\"o}}1 {�}{{\"u}}1
	%{�}{{\"A}}1 {�}{{\"E}}1 {�}{{\"I}}1 {�}{{\"O}}1 {�}{{\"U}}1
	%{�}{{\^a}}1 {�}{{\^e}}1 {�}{{\^i}}1 {�}{{\^o}}1 {�}{{\^u}}1
	%{�}{{\^A}}1 {�}{{\^E}}1 {�}{{\^I}}1 {�}{{\^O}}1 {�}{{\^U}}1
	%{�}{{\oe}}1 {�}{{\OE}}1 {�}{{\ae}}1 {�}{{\AE}}1 {�}{{\ss}}1
	%{�}{{\c c}}1 {�}{{\c C}}1 {�}{{\o}}1 {�}{{\r a}}1 {�}{{\r A}}1
	%{�}{{\EUR}}1 {�}{{\pounds}}1
	%
	%{^}{{{\color{ipython_purple}\^{}}}}1
	%{=}{{{\color{ipython_purple}=}}}1
	%%
	%*{-}{{{\color{ipython_purple}-}}}1
	%{*}{{{\color{ipython_purple}$^\ast$}}}1
	%{/}{{{\color{ipython_purple}/}}}1%%
	%{+=}{{{+=}}}1
	%{-=}{{{-=}}}1
	%{*=}{{{$^\ast$=}}}1
	%{/=}{{{/=}}}1,
	%
	identifierstyle=\color{black}\footnotesize\ttfamily,
	commentstyle=\color{ipython-cyan}\footnotesize\itshape\ttfamily,
	stringstyle=\color{ipython-red}\footnotesize\ttfamily,
	keepspaces=true,
	showspaces=false,
	showstringspaces=false,
	rulecolor=\color{ipython-frame},
	frame=single,
	frameround={t}{t}{t}{t},
	%framexleftmargin=6mm,
	%numbers=left,
	%numberstyle=\color{ipython-cyan},
	backgroundcolor=\color{ipython-bg},
	%   extendedchars=true,
	basicstyle=\footnotesize\ttfamily,
	keywordstyle=[2]\color{ipython-green}\bfseries\footnotesize\ttfamily, 
	keywordstyle=\color{ipython-purple}\bfseries\footnotesize\ttfamily
}

\lstdefinelanguage{iOutput} {
	sensitive=true,
	identifierstyle=\color{black}\small\ttfamily,
	stringstyle=\color{ipython-red}\small\ttfamily,
	keepspaces=true,
	showspaces=false,
	showstringspaces=false,
	rulecolor=\color{ipython-frame},
	%frame=single,
	%frameround={t}{t}{t}{t},
	%backgroundcolor=\color{ipython-bg},
	basicstyle=\small\ttfamily,
}

\lstnewenvironment{ipython}[1][]{\lstset{language=iPython,mathescape=true,escapeinside={*@}{@*}}%
}{%
}

\lstnewenvironment{ioutput}[1][]{\lstset{language=iOutput,mathescape=true,escapeinside={*@}{@*}}%
}{%
}


\title{Advanced Financial Modeling Course 23/24\\ Exam}
\author{Prof. Andrea Carapelli, Prof. Matteo Sani}
\date{$16^{\mathrm{th}}$ February 2024}

\printanswers
%\noprintanswers
\begin{document}
\maketitle
\addpoints{exam}
\begin{center}
\fbox{\fbox{\parbox{5.5in}{\centering
Answer the questions in the spaces provided. If you run out of room for an answer, continue on the page back.}}}
\end{center}

\begin{center}
\vspace{5mm}
\makebox[0.75\textwidth]{Student's name:\enspace\hrulefill}
\end{center}

\section*{Questions}
\vspace{.5cm}
\begin{questions}

%%%%%%%%%%%%%%%%%%%%%%%%%%%%%%%%%%%%%%%%%%%%%%%%%%%%%
\question Consider the process $Y(t) = 2^{W(t)}$, where $\{W(T):t\geq 0\}$ is a standard Brownian motion. Is this a martingale ?
\fillwithlines{3cm}
\begin{solution}
With $g(t)=2^{W(t)}$, we find:
\begin{equation*}
dg(t) = \ln2\cdot 2^{W(t)}dW(t) +\cfrac{(\ln2)^2}{2}2^{W(t)}dt
\end{equation*}
Note that $g, g_{x}, g_{xx}$ exist and are continouos. 
Due to the appearance of a $dt$-term, the process is not a martingale.
\end{solution}

%%%%%%%%%%%%%%%%%%%%%%%%%%%%%%%%%%%%%%%%%%%%%%%%%%%%%
\question Let $\{W(T):t\geq 0\}$ be a Brownian motion on a probability space and let $\mathcal{F}_t$ be its natural filtration. Consider a stock with price process $\{S(t):0\leq t \leq T\}$, with 
\begin{equation*}
S(t)=S(0)\exp\left\{\int_0^t e^{-u}dW(u) + \int_0^t(1-\frac{1}{2}e^{-2u}du\right\}
\end{equation*}
\begin{enumerate}[label=(\alph*),font=\itshape]
\item Let 
\begin{equation*}
X(t)=\int_0^t e^{-u}dW(u)+\int_0^t(1-\frac{1}{2}e^{-2u}du
\end{equation*}
and determine the distribution of $X(t)$.
\item Prove that $\{S(t):t\geq 0\}$ is an \ito's process.
\item Let $r$ be a constant interest rate. Find the risk-neutral measure $\tilde{\mathbb{P}}$, equivalent to $\mathbb{P}$, such that the discounted price process $\{we^{-rt}S(t): 0\leq t \leq T\}$ is a martingale under $\tilde{\mathbb{P}}$.
\end{enumerate}
\fillwithlines{3cm}
\begin{solution}
\begin{enumerate}[label=(\alph*),font=\itshape]
\item Let $Y(t)=\int_0^t e^{-u}dW(u)$, or the first term of the $X(t)$ process. From the stochastic calculus results we know that $Y(t)$ is normally distributed with $\mathbb{E}[Y(t)]=0$ and 
\begin{equation*}
\text{Var}[Y(t)]=\int_0^t e^{-2u}du = \frac{1}{2}(1-e^{-2t})
\end{equation*} 
Since 
\begin{equation*}
X(t) = Y(t) + \int_0^t (1-\frac{1}{2}e^{-2u})du = Y(t) + t + \frac{1}{4}(e^{-2t}-1)
\end{equation*} 
we see that $X(t)$ is normally distributed, with mean 
\begin{equation*}
\mathbb{E}[X(t)] = t + \frac{1}{4}(e^{-2t}-1)
\end{equation*} 
and variance
\begin{equation*}
\text{Var}[X(t)] = \text{Var}[Y(t)] = \frac{1}{2}(1-e^{-2t})
\end{equation*} 
\item With 
\begin{equation*}
X(t) = \int_0^t e^{-u}dW(u) + \int_0^t (1-\frac{1}{2}e^{-2u})du
\end{equation*} 
we have 
\begin{equation*}
dX(t) = e^{-t}dW(t) + (1-\frac{1}{2}e^{-2t})dt
\end{equation*} 
and $dX(t)dX(t)=e^{-2t} dt$. Note that $S(t)=S(0)e^{X(t)}$, so let $f(x)=S(0)e^x$, then $f_x(x)=f_{xx}(x)=f(x)$. By the \ito formula we get
\begin{equation*}
\begin{aligned}
dS_t &= df(X_t) = S_tdX+\frac{1}{2}S(t)dXdX = \\
&=S_t\left(e^{-t}dW + (1-\frac{1}{2}e^{-2t})dt\right)+\frac{1}{2}S_t e^{-2t}dt = \\
&=S_t dt+S_t e^{-t}dW
\end{aligned}
\end{equation*}
This shows that $S_t$ is an \ito process.
\item Define 
\begin{equation*}
\theta(r)=\cfrac{1-r}{e^{-t}}=e^t (1-r)
\end{equation*}
Consider the random variable $Z$, defined by
\begin{equation*}
\begin{aligned}
Z &= \exp\left(-\int_0^T \theta(u)dW(u) - \frac{1}{2}\int_0^T \theta^2(u)du\right) = \\
&=\exp\left(-\int_0^T e^u (1-r)dW(u) - \frac{1}{2}\int_0^T e^{2u} (1-r)^2 du\right)
\end{aligned}
\end{equation*}
Define the measure $\tilde{\mathbb{P}}$ by $\tilde{\mathbb{P}}(A) =\int_A Z d\mathbb{P}$ and consider the process 
\begin{equation*}
\begin{aligned}
\tilde{W}_t = \int_0^t \theta(u)dW(u) +W(t) = \int_0^t e^{u}(1-r)du+W(t)=(1-r)(e^t -1)+W(t)
\end{aligned}
\end{equation*}
By Girsanov Theorem $\tilde{W}$ is a Brownian motion under $\tilde{\mathbb{P}}$ and hence it is a martingale under $\tilde{\mathbb{P}}$. 
Using the SDE of part (a), together with the \ito product rule, we have 
\begin{equation*}
\begin{aligned}
d(e^{-rt}S(t)) &= e^{-rt}dS_t-r e^{-rt}S_t dt = \\
&= e^{-rt}(S_t dt + S_te^{-t}dW)-re^{-rt}S_t = \\
&= e^{-rt}S_t((1-r)dt + e^{-t}dW)) = \\
&= e^{-rt}S_t(e^{-t}\theta_t dt + e^{-t}dW)) = \\
&= e^{-t(r+1)}S_t d\tilde{W}_t
\end{aligned}
\end{equation*}
Since $e^{-rt}S(t)$ is an \ito integral, we see that the discounted price process is a martingale under $\tilde{\mathbb{P}}$.
\end{enumerate}
\end{solution}

%%%%%%%%%%%%%%%%%%%%%%%%%%%%%%%%%%%%%%%%%%%%%%%%%%%%%
\question Suppose that $X(t)$ satisfies the following SDE:
\begin{equation*}
dX_t = 0.04X_t dt + \sigma X_t dW_t
\end{equation*}
and $Y_t$ satisfies:
\begin{equation*}
dY_t = \beta Y_t dt + 0.1 Y_t dW_t
\end{equation*}
Parameters $\beta, \sigma$ are postive and both processes are driven by the same Brownian Motion $W(t)$.
For a given process
\begin{equation*}
Z_t = 2\cfrac{X_t}{Y_t}-\lambda t
\end{equation*}
with $\lambda\in\mathbb{R}^+$.
\begin{enumerate}[label=(\alph*),font=\itshape]
\item Find the SDE for $Z_t$;
\item For which values of $\beta$ and $\lambda$ is the process $Z_t$ a martingale ?
\end{enumerate}

\fillwithlines{3cm}
\begin{solution}
\begin{enumerate}[label=(\alph*),font=\itshape]
\item We have 
\begin{equation*}
\begin{gathered}
X_t = e^{\sigma W_t-\frac{\sigma^2}{2}t+0.04 t}\\
dX_t = 0.04 X_t dt + \sigma X_t dW_t\\
Y_t = e^{0.1W_t -\frac{0.01}{2}t+\beta t}\\
dY_t = \beta Y_t dt + 0.1 Y_t dW_t
\end{gathered}
\end{equation*}
Using the expressions fot $X_t$ and $Y_t$ we get
\begin{equation*}
Z_t = 2\exp\left((\sigma-0.1)W_t + (0.04+\frac{0.01}{2}-\beta-\frac{\sigma^2}{2})t\right)-\lambda t
\end{equation*}
\item A martingale process does not contain a drift term. We have 
\begin{equation*}
dZ_t = (Z+\lambda t)(0.01+0.04-\beta-0.1\sigma)dt-\lambda dt + (Z+\lambda t)(\sigma-0.1)dW_t
\end{equation*}
With $\beta$ and $\sigma$ constant and $\lambda\in\mathbb{R}^+$, the necessary conditions for a vanishing drift term are $\lambda=0$ and
\begin{equation*}
0.01+0.04-\beta-0.1\sigma=0\implies \beta=0.05-0.15\sigma
\end{equation*}
%To check this result we employ the \ito derivative rules for multivariate functions...
\end{enumerate}
\end{solution}

%%%%%%%%%%%%%%%%%%%%%%%%%%%%%%%%%%%%%%%%%%%%%%%%%%%%%
\question Suppose $B(t)$ is a standard Brownian motion. For each of the following choices of $X_t$, find an equivalent probability measure $\mathbb{Q}$ such that $X_t$ is a Brownian motion in the new measure. Assume $X_0=B_0=0$
\begin{equation*}
\begin{gathered}
dX_t = 2dt + dB_t\\
dX_t = 2dt + 6dB_t
\end{gathered}
\end{equation*}

\fillwithlines{3cm}
\begin{solution}
\begin{enumerate}[label=(\alph*),font=\itshape]
\item The Girsanov transformation tells us that for a process X driven by Brownian motion B under the original measure P and density process Z, the process defined as:

%dY_t = X_t - ∫_0^t Z_s dB_s
%
%will be a Brownian motion under the new measure Q with Radon-Nikodym derivative:
%
%dQ/dP = exp(∫_0^t Z_s dB_s - 1/2 ∫_0^t Z_s^2 ds)
%
%Since Z_t is standard normal, we can simplify the calculations.
%
%Case 1: dX_t = 2dt + dB_t
%
%Here, Z_t = 1. Plugging into the equations:
%
%dY_t = X_t - ∫_0^t dB_s = X_t - B_t
%
%dQ/dP = exp(B_t - 1/2 t)
%
%Therefore, under the measure Q defined by this Radon-Nikodym derivative, Y_t = X_t - B_t becomes a Brownian motion starting from 0.
%
%Case 2: dX_t = 2dt + 6dB_t
%
%In this case, Z_t = 6. Following the same steps:
%
%dY_t = X_t - ∫_0^t 6 dB_s = X_t - 6B_t
%
%dQ/dP = exp(6B_t - 18t)
%
%Under the measure Q defined by this Radon-Nikodym derivative, Y_t = X_t - 6B_t becomes a Brownian motion starting from 0.
\end{enumerate}
\end{solution}

%%%%%%%%%%%%%%%%%%%%%%%%%%%%%%%%%%%%%%%%%%%%%%%%%%%%%
%\question Let $f$ be a function double differentiable. Assume that $f$ is a solution of the heat equation
%\begin{equation*}
%\cfrac{\partial}{\partial t}f(t, x) = -\cfrac{1}{2}\cfrac{\partial^2}{\partial x^2}f(t, x)
%\end{equation*}
%Let $B_t$ be a standard Brownian motion.
%\begin{enumerate}[label=(\alph*),font=\itshape]
%\item Find the SDE solved by $f(t, B_t)$.
%\item Deduce that $f(t, B_t)$ is a martingale if and only if $f$ is solution of the heat equation.
%\end{enumerate}
%\fillwithlines{3cm}
%\begin{solution}
%\begin{enumerate}[label=(\alph*),font=\itshape]
%\item Apply Ito's formula to function $f(t, B_t)$ to get the differential
%\begin{equation*}
%df = \cfrac{\partial f}{\partial t} dt + \cfrac{\partial f}{\partial B}dB_t + \frac{1}{2}\cfrac{\partial^2 f}{\partial B^2}dB_t^2
%\end{equation*}
%Given that $f$ is solution of the heat equation we can simplify the differential (replacing $dB_t^2=dt$)
%\begin{equation*}
%df =\cfrac{\partial f}{\partial B}dB_t = \phi_t dB_t
%\end{equation*}
%Considering again the differential $df$,  if the function is a martingale the only surviving term should be the one in $dB_t$ hence it must hold
%\begin{equation*}
%\cfrac{\partial}{\partial t}f(t, x) = -\cfrac{1}{2}\cfrac{\partial^2}{\partial x^2}f(t, x)
%\end{equation*}
%so $f$ is a solution of the heat equation.
%\end{enumerate}
%\end{solution}


%%%%%%%%%%%%%%%%%%%%%%%%%%%%%%%%%%%%%%%%%%%%%%%%%%%%%
\question Show that the exponential SDE
\begin{equation*}
dX_t = A_t X_tdW_t,\quad X_0=x_0
\end{equation*}
has the following solution
\begin{equation*}
X_t = x_0 e^{-\frac{1}{2}\int_0^t A_0^2 ds+\int_0^t A_s dB_s}
\end{equation*}
%\textbf{Hint:} Apply \ito formula to $f(Y_t)$ where $f(x)=x_0 e^x$ and $Y_t=-\frac{1}{2}\int_0^t A_s^2 ds + \int_0^t A_s dB_s$
\fillwithlines{3cm}
\begin{solution}
The SDE can be expressed as $d(\ln(X_t)) = \cfrac{dX_t}{X_t} = A_t dW_t$ where we have changed to the variable $Y_t = \ln(X_t)$. Applying \ito formula to $Y_t$ 
\begin{equation*}
dY_t = \left[a\cfrac{\partial Y}{\partial X} + \cfrac{\partial Y}{\partial t} + \frac{1}{2}b\cfrac{\partial^2 Y}{\partial X^2}\right]+ b\cfrac{\partial Y}{\partial X}dW_t
\end{equation*}
but $a=0, \frac{\partial Y}{\partial t}=0, b=A_tX_t, \frac{\partial Y}{\partial X}=\frac{1}{X}, \frac{\partial^2 Y}{\partial X^2}=-\frac{1}{X^2}$, so
\begin{equation*}
dY_t = \left[\cancel{a\cfrac{\partial Y}{\partial X}} + \cancel{\cfrac{\partial Y}{\partial t}} - \frac{1}{2}A_t^2 \cancel{X_t^2}\cfrac{1}{\cancel{X^2}}\right]+ A_t\cancel{X_t}\cfrac{1}{\cancel{X_t}}dW_t = -\frac{1}{2}A_t^2 + A_t dW_t
\end{equation*}
Integrating and xconsidering the initial condition given by the problem
\begin{equation*}
\begin{gathered}
\ln\left(\frac{X_t}{X_0}\right) = -\frac{1}{2}\int_0^t A_s^2 ds + \int_0^tA_s dW_s \\
X_t = X_0 \exp\left(-\frac{1}{2}\int_0^t A_s^2 ds + \int_0^tA_s dW_s \right)
\end{gathered}
\end{equation*}



%dg = g_t dt + g_x dX_t + 1/2 g_{xx} d[X_t]^2
%
%Since d[X_t]^2 = X_t^2 dB_t^2 = X_t^2 dt (due to Brownian motion properties), we get:
%
%dg = (1/X_t) dX_t + 1/2 * (-1/X_t^2) * X_t^2 dt
%
%Substituting the SDE for dX_t:
%
%dg = A_t dB_t - 1/2 A_t^2 dt
%
%Integrating both sides from 0 to t:
%
%g(t, X_t) - g(0, x_0) = \int_0^t A_s dB_s - 1/2 \int_0^t A_0^2 ds
%
%Since g(0, x_0) = ln(x_0), we have:
%
%ln(X_t) - ln(x_0) = \int_0^t A_s dB_s - 1/2 \int_0^t A_0^2 ds
%
%Exponentiating both sides:
%
%X_t/x_0 = e^{\int_0^t A_s dB_s - 1/2 \int_0^t A_0^2 ds}
%
%Finally, rearranging:
%
%X_t = x_0 e^{-\frac{1}{2} \int_0^t A_0^2 ds + \int_0^t A_s dB_s}
%
%Therefore, under the mentioned conditions, the proposed solution satisfies the SDE and can be considered valid.
%
%Additional notes:
%
%This solution is known as the exponential martingale solution, as X_t remains a martingale under certain conditions.
%The solution assumes the initial condition X_0 > 0, since the logarithm is undefined for non-positive values.
%I hope this revised response is more helpful and clarifies the validity of the solution under the specified conditions!
\end{solution}

%%%%%%%%%%%%%%%%%%%%%%%%%%%%%%%%%%%%%%%%%%%%%%%%%%%%%
\question Let $W_1(t)$ and $W_2(t)$ a 2D-Brownian motion defined on a probability space $(\Omega, \mathcal{F}, \mathbb{P})$. Consider two price processes $\{S_1(t):t\geq 0\}$ and $\{S_2(t):t\geq 0\}$ with corresponding SDEs given by
\begin{equation*}
\begin{aligned}
dS_1(t) &= 2S_1(t)dW_1(t) + 3S_1(t)dW_2(t)\\
dS_2(t) &= S_2(t)dt + S_2dW_1(t)
\end{aligned}
\end{equation*}
\begin{enumerate}[label=(\alph*),font=\itshape]
\item Show that $S_1(t)S_2(t)$ is a 2D-\ito process.
\item Consider an expiration date $T$, and suppose tha interest rate is a constant $r$. Show that the market price equations have a unique solution, and determine the risk-neutral probability measure $\tilde{\mathbb{P}}$ for the process $\{S_1(t),S_2(t):0\leq t\leq T\}$.
\end{enumerate}
\fillwithlines{3cm}
\begin{solution}
\begin{enumerate}[label=(\alph*),font=\itshape]
\item We apply \ito product rule, we find
\begin{equation*}
d(S_1S_2) = S_1dS_1+S_2dS_2+dS_1 dS_2
\end{equation*}
Using 
\begin{equation*}
dS_1 = 2S_1dW_1+3S_1dW_2 \\
dS_2 = S_2dt +S_2dt +S_2dW_1
\end{equation*}
and
\begin{equation*}
dS_1 dS_2 = 2S_1 S_2 dt 
\end{equation*}
we get after simplifying
\begin{equation*}
d(S_1 S_2) = 3S_1 S_2 dt + 3S_1S_2dW_1 + 3S_1S_2dW_2 
\end{equation*}
Hence  $\{S_1(t),S_2(t):0\leq t\leq T\}$ is a 2D-\ito process.
\item Using the notation of the book, 
%we have α1 = 0, σ11 = 2, σ12 =
%3, α2 = 1, σ21 = 1, σ22 = 0. The market price equations in this case
%are given by the system,
%−r = 2θ1(t) + 3θ2(t)
%1 − r = θ1(t).
%Solving for θ1(t), θ2(t), we get
%θ1(t) = 1 − r
%θ2(t) = r − 2
%3
%.
%Setting,
%Z = exp (
%−
%Z T
%0
%(θ1(t)dW1(t) + θ2(t)dW2(t)) −
%1
%2
%Z T
%0
%
%θ
%2
%1
%(t) + θ
%2
%2
%(t)
%
%dt
%)
%= exp 
%(r − 1)W1(T) + 2 − r
%3
%W2(T) −
%1
%2
%
%(1 − r)
%2 +
%(r − 2)2
%9
%
%T
%
%,
%the risk-neutral measure is given by P˜(A) = R
%A
%ZdP. To check this,
%we set W˜
%1(t) = (1 − r)t + W1(t) and W˜
%2(t) = r−2
%3
%t + W2(t). By
%the 2-dimensional Girsanov Theorem, the process {(W˜
%1(t), W˜
%2(t) :
%0 ≤ t ≤ T} is a 2-dimensional Brownian motion under P˜. Rewritin
%5
%e
%−rtS1(t), e
%−rtS2(t) in terms of W˜
%1(t), W˜
%2(t), we get, after applying
%the Itˆo product rule,
%d(e−rtS1(t)) = e−rtS1(t)(2dW˜
%1(t) + 3dW˜
%2(t))
%d(e−rtS2(t)) = e−rtS2(t)dW˜
%1(t),
%which shows that the discounted price processes are Itˆo integrals, and
%hence martingales under P˜.
\end{enumerate}
\end{solution}

%%%%%%%%%%%%%%%%%%%%%%%%%%%%%%%%%%%%%%%%%%%%%%%%%%%%%
\question Assume we have a European call and put option, with the same expiry date $T=1/4$, i.e. exercise in three months, and strike price $K=10$. The current share price is 11, assuming a constant interest rate $R=6\%$. Determine an arbitrage opportunity if both options currently have the value $c(0)=p(0)=2.5$.

\fillwithlines{3cm}
\begin{solution}
We form two portfolios using the options, the underlying asset and a cash amount $K$, with one based on the put and the other based on the call, as follows
\begin{equation*}
\begin{aligned}
\Pi_1(t) &= p(t) + S(t)\\
\Pi_2(t) &= c(t) + Ke^{-r(T-t)}
\end{aligned}
\end{equation*}
These portfolios have same value at expiry time $T$. By the put-call parity, their value should be equal any time prior to the exercise time, as otherwise arbitrage opportunities will appear. In the case of a mismatch in value, one can buy the cheaper portfolio and sell the expensive one. At the expiry time $T$, one can trade these two portfolios without any cost, hence the initial sell-buy difference is reflected as a profit. 

Returning to the exercise and looking at the arbitrage opportunity when both options are worth 2.5, we assume this takes place at $t < T$. Using the put-call parity relation, we find the following relation for not having an arbitrage opportunity, 
\begin{equation*}
S(t) = 10e^{-0.06(0.25-t)}
\end{equation*}
Hence, at $t = 0$, assuming that the option values are 2.5, one can benefit from selling portfolio $Pi_1$ and buying $\Pi_2$. As long as $S(t) > 10 e^{
-0.06(0.25-t)}$, one can follow this strategy, when $S(t) < 10 e^{-0.06(0.25-t)}$, one should revert the strategy.
\end{solution}

%%%%%%%%%%%%%%%%%%%%%%%%%%%%%%%%%%%%%%%%%%%%%%%%%%%%%
\question Consider the process $Z_t$ whose dynamics is defined by the following SDE $dZ_t = -\phi_tZ_tdW_t, Z_0=1$, where $W_t$ is a Brownian motion under the measure $\mathbb{P}$. For any $t\geq 0$ define the a new measure $\mathbb{Q}$ according to $d\mathbb{Q}=Z_t d\mathbb{P}$.
Prove that 
\begin{equation*}
\mathbb{E}^{\mathbb{Q}}[Z_T\log(Z_T)] = \mathbb{E}^{\mathbb{Q}}\left(\frac{1}{2}\int_0^T\phi_s^2 ds\right)
\end{equation*}

\fillwithlines{3cm}
\begin{solution}
$Z$ is clearly a $\mathcal{P}$-martingale, so that $\mathcal{Q}$ defines a genuine probability measure, and therefore
\begin{equation*}
\expect{P}(Z_t\log(Z_t)) = \expect{Q}(\log(Z_t))
\end{equation*}
Now, applying \ito formula, we can write
\begin{equation*}
Z_T = \exp\left(-\frac{1}{2}\int_0^T \phi_s^2 ds - \int_0^T \phi_s dW_s\right)
\end{equation*}
From Girsanov theorem, the process $\tilde{W}$ defined pathwise as $\tilde{W}_t := W_t +\int_0^t\phi_s ds$ is a standard Brownian motion under $\mathcal{Q}$ and 
\begin{equation*}
-\frac{1}{2}\int_0^t \phi_s^2 ds - \int_0^t \phi_s dW_s = \frac{1}{2}\int_0^t \phi_s^2 ds - \int_0^t \phi_s d\tilde{W}_s
\end{equation*}
from which the result follows.
\end{solution}

%%%%%%%%%%%%%%%%%%%%%%%%%%%%%%%%%%%%%%%%%%%%%%%%%%%%%
\question Let $X_t$ be the unique solution to the following stochastic differential equation, under $\mathcal{P}$:
\begin{equation*}
dX_t = X_t(\mu_t dt + \sigma_t dW_t)
\end{equation*}
where $\mu$ and $\sigma$ are bounded and adapted processes, and $\sigma >0$ almost surely.
\begin{enumerate}[label=(\alph*),font=\itshape]
\item Show that $X_t\exp(-\int_0^t \mu_s ds)$ is a martingale.
\item Find a probability $\mathbb{Q}$, equivalent to $\mathbb{P}$ under which $X$ is a martingale.
\item Find a probability $\tilde{\mathbb{P}}$, equivalent to $\mathbb{P}$, under which the inverse process $X^{-1}$ is a martingale.
\end{enumerate}
\fillwithlines{3cm}
\begin{solution}
\begin{enumerate}[label=(\alph*),font=\itshape]
\item From \ito formula we can write for any $t\geq 0$
\begin{equation*}
X_t = \exp\left\{\int_0^t\left(\mu_s - \frac{1}{2}\sigma_s^2\right)ds + \int_0^t\sigma_s dW_s\right\}
\end{equation*}
so that (a) follows immediately. One can apply Girsanov theorem to introduce the probability measure $\mathbb{Q}$ via $d\mathbb{Q}=Z_t d\mathbb{P}$ with $dZ_t = Z_t\mu_t \sigma_t^{-1}dW_t$, such that $B_t := W_t +\mu_t\sigma^{-1}_t dt$ is a standard Brownian motion under $\mathbb{Q}$.
Finally, applying \ito formula yields
\begin{equation*}
dX^{-1}_t = -X^{-1}_t \sigma_t  \left( dW_t - \frac{\sigma_t^2 - \mu_t}{\sigma_t}dt\right)
\end{equation*}
such that (c) follows again by a direct application of Girsanov theorem.
\end{enumerate}
\end{solution}

%%%%%%%%%%%%%%%%%%%%%%%%%%%%%%%%%%%%%%%%%%%%%%%%%%%
\question Fix some $\lambda\in\mathbb{R}$, and let $X$ be a process such that
\begin{equation*}
dX_t = -\lambda X_t dt + dW_t
\end{equation*}
and introduce the process $Z$ as 
\begin{equation*}
Z_t := \exp\left\{\lambda\int_0^t X_s dW_s - \frac{\lambda^2}{2}\int_0^t X_s^2 ds\right\}
\end{equation*}
\begin{enumerate}[label=(\alph*),font=\itshape]
\item Show that $Z$ is a martingale.
\item Define the new probability measure $\mathbb{Q}$ as $d\mathbb{Q}:=Z_td\mathbb{P}$. Write the stochastic differential equation solved by the rpocess $X$ under $\mathbb{Q}$.
\item Show that 
\begin{equation*}
Z_t := \exp\left\{\lambda\int_0^t X_s dW_s + \frac{\lambda^2}{2}\int_0^t X_s^2 ds\right\}
\end{equation*}
and compute, for any $u\in\mathbb{R}$, the expectation
\begin{equation*}
\mathbb{E}^{\mathbb{P}}\left[\exp\left\{\frac{u^2}{2}\int_0^t X_s^2 ds\right\}\right]
\end{equation*}
You might need to show first that an application of \ito formula yields
\begin{equation*}
\int_0^t W_s dW_s = \cfrac{W_t^2-t}{2}
\end{equation*}
\end{enumerate}
\fillwithlines{3cm}

\begin{solution}
This is a straightforward application of Girsanov theorem: under $\mathbb{Q}$ the process $W_t -\lambda\int_0^tX_s ds$ is a standard Brownian motion. Combining this with the SDE, we obatain that 
\begin{equation*}
X_t = x -\lambda\int_0^t X_s ds + W_t
\end{equation*}
is a standard Brownian motion under $\mathbb{Q}$. Finally
\begin{equation*}
\begin{aligned}
\mathbb{E}^{\mathbb{P}}\left[\exp\left\{\frac{u^2}{2}\int_0^t X_s^2 ds\right\}\right] &= \mathbb{E}^{\mathbb{Q}}\left[Z_t^{-1}\exp\left\{\frac{u^2}{2}\int_0^t X_s^2 ds\right\}\right] \\
 &= \mathbb{E}^{\mathbb{Q}}\left[\exp\left\{-\frac{u^2+\lambda^2}{2}\int_0^t X_s^2 ds-\lambda\int_0^t X_s dX_s \right\}\right] \\
 &= \mathbb{E}^{\mathbb{Q}}\left[\exp\left\{-\frac{u^2+\lambda^2}{2}\int_0^t X_s^2 ds-\frac{\lambda}{2}(X_t^2-t)\right\}\right] \\
 &= e^{\lambda t/2}\mathcal{N}\left(\frac{\lambda}{2}, \sqrt{\lambda^2 + u^2}\right)
\end{aligned}
\end{equation*}
where $\mathcal{N}$ denote the Gaussian cumulative distribution function.
\end{solution}

%%%%%%%%%%%%%%%%%%%%%%%%%%%%%%%%%%%%%%%%%%%%%%%%%%%
\question Let $S$ be a martingale satisfying the stochastic differential equation $dS_t = \sigma S_t dW_t$, starting from $S_0 = 1$,
where $\sigma$ is a strictly positive constant.
\begin{enumerate}[label=(\alph*),font=\itshape]
\item Check that $S_t$ is strictly positive almost surely for all $t \geq 0$.
\item Compute explicitly $Xt := S_t^{-1}$.
\item Let $\mathbb{Q}$ be a new probability measure defined via $d\mathbb{Q} := S_t d\mathbb{P}$. What is the law of $X_t$ under $\mathbb{Q}$ ? 
\item Show finally the Put-Call symmetry (different from the Put-Call parity!!!!):
\begin{equation*}
\mathbb{E}^{\mathbb{P}}(S_T-K)^+ = K\mathbb{E}^{\mathbb{Q}}\left[(K^{-1}-X_T)^+\right]
\end{equation*}
\end{enumerate}
\fillwithlines{3cm}

\begin{solution}
\begin{enumerate}[label=(\alph*),font=\itshape]
\item Itˆo’s lemma implies that $S_t = \exp (-\frac{1}{2}\sigma^2_t - \sigma W_t)$ for any $t \geq 0$. Since the Brownian motion does not
explode to infinity over any finite time horizon, the result follows.
\item Using the previous representation, we immediately have
\begin{equation*}
X_t = S_t^{-1} = \exp\left(\frac{1}{2}\sigma^2 t - \sigma W_t\right)
\end{equation*}
It further satisfies the stochastic differential equation (by Itˆo’s lemma):
\begin{equation*}
dX_t = \frac{dS_t}{S_t^2} + \frac{d<S_t>}{S_t^3} = -\sigma X_t dW_t + \sigma_t dt
\end{equation*}
\item Since $S_t$ is a true strictly positive martingale, $\mathbb{Q}$ is a well-defined probability measure, equivalent to $\mathbb{P}$.
Therefore the process $B_t$ defined by $B_t := W_t - \sigma t$ is a standard Brownian motion under $\mathbb{Q}$, and so is $W^{\mathbb{Q}} := -B$, and hence 
\begin{equation*}
dX_t = -\sigma X_t(dW_t - \sigma dt) = \sigma X_t dW_t^{\mathbb{Q}}
\end{equation*}
Under $\mathbb{Q}$, the process $X$ is therefore a geometric Brownian motion.
\item Using the change of measure introduced previously, we can write, for all $K>0$,
\begin{equation*}
\mathbb{E}^{\mathbb{P}}(S_T-K)^+ = \mathbb{E}^{\mathbb{P}}\left[S_T\left(1-\frac{K}{S_T}\right)^+\right] = K \mathbb{E}^{\mathbb{Q}}\left[\left(\frac{1}{K}-\frac{1}{S_T}\right)^+\right] = K\mathbb{E}^{\mathbb{Q}}\left[(K^{-1}-X_T)^+\right]
\end{equation*}
\end{enumerate}
\end{solution}

%%%%%%%%%%%%%%%%%%%%%%%%%%%%%%%%%%%%%%%%%%%%%%%%%%%
\question Consider the process $S_t$ given by
\begin{equation*}
S_t = S_0\exp(2\mu t + 2W_t)
\end{equation*}
Show that $S$ is a submartingale whenever $\mu\geq -1$, and a supermartingale otherwise. Show then that the price of an Asian option, with payoff $\frac{1}{T}\int_0^T S_u du - K)^+$ is greater than the corresponding Call price with payoff $(S_T-K)^+$, for small enough $K\geq 0$.
\textbf{Hint:} you may want to show first that the representation $S_t = S_0 + martingale + 2(1 + \mu)\int_0^t S_u du$  holds almost surely for all $t \geq 0$, and then the trivial inequality (which follows from the convexity of the exponential function) $e^x \leq 1 + xe^x$ for any $x \in\mathbb{R}$.
\fillwithlines{3cm}

\begin{solution}
\end{solution}

%%%%%%%%%%%%%%%%%%%%%%%%%%%%%%%%%%%%%%%%%%%%%%%%%%%
\question For any $\beta\in\mathbb{R}$, consider the process $S_t$ defined as the solution to the stochastic differential equation $dS_t = \sigma S^\beta_t dW_t$, $S_0 = 1$.
\begin{enumerate}[label=(\alph*),font=\itshape]
\item What is this process when $\beta=0$ and $\beta = 1$ ?
\item Take $\sigma = 0.1$ and $\beta = 2$. Using the closed-form representation, compute, on the same plot, the functions $K\rightarrow \mathbb{E}(S_T - K)^+$, for $T\in\{0, 0.1, 1, 5\}$, and discuss the plots.
\end{enumerate}
\fillwithlines{3cm}

\begin{solution}
The given stochastic differential equation (SDE) describes the dynamics of a process $S_t$ driven by Brownian motion $dW_t$ and scaled by the current value of $S_t$ raised to the power of $\beta$. Let's see how the process behaves for different values of $\beta$:
\begin{itemize}
\item \textbf{$\beta=0$} In this case, the SDE becomes: $dS_t = \sigma dW_t$. This is simply the standard Wiener process, also known as Brownian motion. It means the change in $S_t$ depends only on the Brownian motion and is independent of the current value of $S_t$. Therefore, $S_t$ follows a normal distribution with mean $t$ and variance $\sigma^2 t$.
\item \textbf{$\beta=1$} The SDE becomes: $dS_t = \sigma S_t dW_t$. This is a geometric Brownian motion (GBM). The change in $S_t$ is now proportional to the current value of $S_t$ and the Brownian motion. Hence, $S_t$ can become larger or smaller quickly depending on the random fluctuations. The solution has the form:
\begin{equation*}
S_t = S_0 \exp\left(\sigma W_t - \frac{\sigma^2}{2}t\right) 
\end{equation*}
where $S_0$ is the initial value (1 in this case). This implies that $S_t$ follows a lognormal distribution with parameters $\mu=\frac{\sigma^2}{2}t$ and $\sigma^2 = \sigma^2 t$.
\end{itemize}
\end{solution}

%%%%%%%%%%%%%%%%%%%%%%%%%%%%%%%%%%%%%%%%%%%%%%%%%%%
\question Compute the stochastic differential $dZ$ when
\begin{enumerate}[label=(\alph*),font=\itshape]
\item $Z(t) = \exp(\alpha t)$;
\item $Z(t) = \int_0^t g_s dW_s$ where $g$ is an adapted stochastic process;
\item $Z(t) = \exp(\alpha W(t))$;
\item $Z(t) = \exp(\alpha X(t))$ where $X$ has stochastic differential $dX(t)=\mu dt + \sigma dW(t)$ with $\mu$ and $\sigma$ constants;
\item $Z(t) = X^2(t)$ where $X$ has stochastic differential $dX(t)=\alpha X(t)dt + \sigma X(t)dW(t)$.
\end{enumerate}
\fillwithlines{3cm}

\begin{solution}
\begin{enumerate}[label=(\alph*),font=\itshape]
\item $dZ_t= \alpha e^{\alpha t}dt$;
\item $dZ_t = g(t) dW_t$;
\item by \ito formula: $dZ_t = \frac{\alpha^2}{2}Z(t)dt + \alpha Z_t W_t$;
\item by \ito formula: $dZ_t = (\alpha\mu + \frac{\alpha^2\sigma^2}{2})Z(t)dt + \alpha \sigma Z_t W_t$;
\item by \ito formula: $dZ_t = (\sigma^2 + 2\alpha)Z_t dt + 2\sigma Z_t W_t$.
\end{enumerate}
\end{solution}

%%%%%%%%%%%%%%%%%%%%%%%%%%%%%%%%%%%%%%%%%%%%%%%%%%%
\question Compute the stochastic differential for $Z$ when $Z(t) = \/X(t)$ and $X$ has stochastic differential 
\begin{equation*}
dX(t) = \alpha X(t) dt + \sigma X(t) dW(t)
\end{equation*}
\fillwithlines{3cm}
\begin{solution}
$dZ(t)=(\sigma^2 - \alpha)Z(t)dt -\sigma Z(t)dW(t)$
\end{solution}

%%%%%%%%%%%%%%%%%%%%%%%%%%%%%%%%%%%%%%%%%%%%%%%%%%%
\question Compute the stochastic differential for $Z$ when
\begin{enumerate}[label=(\alph*),font=\itshape]
\item $Z(t) = \int_0^t e^{-as}\sigma dW(s)$
\item $Z(t) = e^{at}\int_0^t e^{-as}\sigma dW(s)$
\end{enumerate}
\fillwithlines{3cm}

\begin{solution}
\begin{enumerate}[label=(\alph*),font=\itshape]
\item $dZ(t) = e^{-at}\sigma dW(t)$;
\item Let $Y(t)=\int_0^t e^{-as}\sigma dW(s)$ and $F(t, y)=e^{at}y$. Then $Z(t)=F(t, Y)$ and \ito formula gives
\begin{equation*}
dZ(t) = a\left\{e^{at}\int_0^t e^{-as}\sigma dW(s)\right\}dt + \sigma dW(t)
\end{equation*}
\end{enumerate}
\end{solution}

%%%%%%%%%%%%%%%%%%%%%%%%%%%%%%%%%%%%%%%%%%%%%%%%%%%
\question Solve the n-dimensional linear equation
\begin{equation*}
\begin{aligned}
dX(t) &= AX(t) dt + \sigma dW(t)\\
X(0) &= x_0
\end{aligned}
\end{equation*}
where $B$ is a d-dimensional Brownian motion, $\sigma$ is a $n\times d$-matrix and $A$ is an $n\times n$-matrix
\fillwithlines{3cm}

\begin{solution}
Using \ito formula one can check that the solution is given by
\begin{equation*}
X(t) = e^{At}x_0 + \int_0^te^{A(t-s)}\sigma dW(s)
\end{equation*}
where we have used the matrix exponential $e^{At}$ defined by
\begin{equation*}
e^{At} = \sum_{k=0}^{\infty} \cfrac{A^k}{k!}t^k
\end{equation*}
\end{solution}

%%%%%%%%%%%%%%%%%%%%%%%%%%%%%%%%%%%%%%%%%%%%%%%%%%%
\question Let $W$ be a Brownian motion and $\mathcal{F}$ the natural filtration generated by $W$. Show, using stochastic calculus, that the following processes are martingales,
\begin{enumerate}[label=(\alph*),font=\itshape]
\item $W^2(t) - t$;
\item $\exp\{\lambda W(t) - \lambda^2\frac{t}{2}\}$, where $\lambda$ is a fixed real number.
\end{enumerate}
\fillwithlines{3cm}

\begin{solution}
\ito formula gives
\begin{enumerate}[label=(\alph*),font=\itshape]
\item $d(W^2(t) - t) = 2W(t)dW(t)$;
\item $d(\exp\{\lambda W(t) - \lambda^2\frac{t}{2}\}) = \lambda\exp\{\lambda W(t) - \lambda^2t/2\}dW(t)$.
\end{enumerate}
We know that $\int_0^tX(s)dW(s)$ is a martingale if $X$ is adapted (and square integrable) so we see that the process are martingales.
\end{solution}

%%%%%%%%%%%%%%%%%%%%%%%%%%%%%%%%%%%%%%%%%%%%%%%%%%%
\question Check whether the following processes $X(t)$ are martingales (w.r.t. the natural filtration generated by $W$),
\begin{enumerate}[label=(\alph*),font=\itshape]
\item $X(t) = W(t) + 4t$;
\item $X(t) = W^2(t)$;
\item $X(t) = t^2W(t) - 2\int_0^tsW(s)ds$;
\item $X(t) = W_1(t) + W_2(t)$ where $W_1$ and $W_2$ is a 2-dimensional Brownian motion.
\end{enumerate}
\fillwithlines{3cm}

\begin{solution}
\ito formula gives the stochastic differentials 
\begin{enumerate}[label=(\alph*),font=\itshape]
\item $dX(t) = 4dt + dW(t) $;
\item $dX(t) = dt+2W(t)dW(t)$;
\item $dX(t) = t^2dW(t)$;
\item $dX(t) = W_1(t)dW_2(t) + W_2(t)dW_1(t)$.
\end{enumerate}
Since the drift term has to be zero in order for the process to be a martingale we see that the process in (a) and (b) are not martingales, while the processes in (c) and (d) are.
\end{solution}

%%%%%%%%%%%%%%%%%%%%%%%%%%%%%%%%%%%%%%%%%%%%%%%%%%%
\question Supoose that $X$ satisfies the SDE
\begin{equation*}
dX(t) = \alpha X(t)dt + \sigma X^{\beta}(t)dW(t)
\end{equation*}
where $\alpha, \sigma, \beta$ are constants and $W$ is a Brownian motion.
\begin{enumerate}[label=(\alph*),font=\itshape]
\item Determine the constants $a, b, c$ such that $Y(t) = \exp\{-a X^b(t) + ct\}$ is a martingale;
\item compute $\mathbb{E}[X(t)]$ when $\beta =1/2$;
\item compute $\text{Var}[X(t)]$ when $\beta =1/2$.
\end{enumerate}
\fillwithlines{3cm}

\begin{solution}
\begin{enumerate}[label=(\alph*),font=\itshape]
\item $a=2\alpha/(b\sigma^2)$, $b=2(1-\beta)$ and $c=ab(b-1)\sigma^2/2$;
\item $\mathbb{E}[X(t)] = x_0e^{\alpha t}$;
\item $\text{Var}[X(t)] = \frac{x_0\sigma^2}{\alpha}(e^{2\alpha t}-e^{\alpha t})$.
\end{enumerate}
\end{solution}

%%%%%%%%%%%%%%%%%%%%%%%%%%%%%%%%%%%%%%%%%%%%%%%%%%%
\question Let $h(t)$ be a deterministic function and define the process $X$ by
\begin{equation*}
X(t) = \int_0^t h(s) dW(s)
\end{equation*}
Show that $X(t)$ is normally distributed with 
\begin{equation*}
\mathbb{E}[X(t)] = 0, \quad \text{Var}[X(t)] = \int_0^t h^2(s) ds
\end{equation*}
by showing that 
\begin{equation*}
\mathbb{E}[\exp\{i u X(t)\}] = \exp\left\{-\frac{1}{2}u^2 \int_0^t h^2(s) ds\right\}
\end{equation*}
\fillwithlines{3cm}

\begin{solution}
Fix $u$ and put $Z(t)=\exp (iuX(t))$. Using \ito formula gives
\begin{equation*}
Z(t) = Z_0 - \int_0^t \cfrac{1}{2}u^2h_s^2Z_s ds + \int_0^t iuh_sZ_sdW_s
\end{equation*}
Taking expected value and solving the resulitng ODE for $m(t)=\mathbb{E}[Z(t)]$:
\begin{equation*}
\frac{dm}{dt} = -{1}{2}u^2h^2(t)m(t), \quad m(0)=1
\end{equation*}
yields the result.
\end{solution}

%%%%%%%%%%%%%%%%%%%%%%%%%%%%%%%%%%%%%%%%%%%%%%%%%%%
\question Let $X$ and $Y$ staisfy the following system of SDE's
\begin{equation*}
\begin{cases}
dX(t) = \alpha X(t) dt + Y(t)dW(t), \quad X(0) = x_0\\
dY(t) = \alpha Y(t) dt - X(t)dW(t), \quad Y(0) = x_0\\
\end{cases}
\end{equation*}
where $x_0$ and $y_0$ are deterministic constants.
\begin{enumerate}[label=(\alph*),font=\itshape]
\item Show that $R$, defined by $R(t) = X^2(t)+Y^2(t)$ is deterministic;
\item compute $\mathbb{E}[X(t)]$ and $\text{Cov}(X(t), Y(t))$, and determine for which value of $\alpha$ the process $R$ becomes a deterministic constant.
\end{enumerate}
\fillwithlines{3cm}

\begin{solution}
\begin{enumerate}[label=(\alph*),font=\itshape]
\item \ito formula gives
\begin{equation*}
dR(t)=(2\alpha + 1)R(t)dt,\quad R(0)=x_o^2 + y_0^2
\end{equation*}
Hence $R(t) = (x_0^2 + y_0^2)e^{(2\alpha + 1)t}$.
\item First compute $\mathbb{E}[X(t)] = x_0e^{\alpha t}$ and  $\mathbb{E}[Y(t)] = y_0e^{\alpha t}$. The let $Z=XY$ and use \ito formula. Taking expected value gives
\begin{equation*} 
\mathbb{E}[Z(t)] = x_0y_0 + \int_0^t(2\alpha - 1)\mathbb{E}[Z(s)]ds
\end{equation*}
Solving the ODE for $m(t) = \mathbb{E}[Z(t)]$ gives 
\begin{equation*} 
m(t) = x_0y_0e^{(2\alpha - 1)t}
\end{equation*}
Using $\text{Cov}(X(t)Y(t))=\mathbb{E}[X(t)Y(t)]-\mathbb{E}[X(t)]\mathbb{E}[Y(t)]$ gives the answer
\begin{equation*} 
\text{Cov}(X(t), Y(t)) = x_oy_0e^{2\alpha t}(e^t - 1)
\end{equation*}
\end{enumerate}

\end{solution}

%%%%%%%%%%%%%%%%%%%%%%%%%%%%%%%%%%%%%%%%%%%%%%%%%%%
\question Let $X$ and $Y$ be given by
\begin{equation*}
\begin{cases}
dX(t) = \alpha X(t) dt + \beta X(t)dW_2(t), \quad X(0) = x_0\\
dY(t) = \gamma Y(t) dt + \sigma X(t)dW_1(t), \quad Y(0) = x_0\\
\end{cases}
\end{equation*}
where $\alpha, \beta, \gamma$ are constant and $W_1$ and $W_2$ are independent Brownian motions.
Compute $\mathbb{E}[X(t)Y(t)]$.
\fillwithlines{3cm}
\begin{solution}
With $Z=XY$ we have by \ito formula that
\begin{equation*}
dZ(t) = (\gamma+\alpha\sigma)Z(t)dt + (\alpha + \sigma)Z(t)dW_1(t) + \beta Z(t)dW_2(t)
\end{equation*}
Integrating and taking expectation gives
\begin{equation*}
\mathbb{E}[Z(t)] = Z_0 + (\gamma+\alpha\sigma)\int_0^t\mathbb{E}[Z(s)]ds
\end{equation*}
Put $m(t)=\mathbb{E}[Z(t)]$ we get the ODE for $m$:
\begin{equation*}
\cfrac{dm}{m} = (\gamma+\alpha\sigma)m(t),\quad m(0)=x_0y_0
\end{equation*}
Solving this ODE we get $\mathbb{E}[Z(t)]=x_0y_0e^{(\gamma+\alpha\sigma)t}$
\end{solution}

When it comes to analyzing callable bonds, investors have a variety of tools and techniques at their disposal. These tools and techniques enable investors to assess the value of callable bonds and make informed investment decisions. In this section, we will discuss some of the most commonly used callable bond analysis tools and techniques.

1. Yield to Call (YTC)

YTC is a measure of the return an investor can expect if a callable bond is called by the issuer. It takes into account the call price, the coupon rate, and the time to call. YTC is an important metric for investors because it allows them to compare the potential return of a callable bond to that of a non-callable bond. However, it is important to note that YTC assumes that the issuer will call the bond at the first opportunity, which may not always be the case.

2. Yield to Worst (YTW)

YTW is a measure of the return an investor can expect if a callable bond is not called by the issuer. It takes into account the yield to maturity and the yield to call, whichever is lower. YTW is important for investors because it provides a more conservative estimate of the potential return of a callable bond. It is also important to note that YTW assumes that the bond will not be called, which may not always be the case.

3. option-Adjusted spread (OAS)

OAS is a measure of the yield spread between a callable bond and a risk-free bond of the same maturity. It takes into account the optionality of the callable bond, which allows the issuer to call the bond at any time. OAS is important for investors because it provides a more accurate estimate of the credit risk associated with a callable bond.

4. Duration

Duration is a measure of the sensitivity of a bond's price to changes in interest rates. It takes into account the bond's cash flows, maturity, and yield. Duration is important for investors because it allows them to assess the risk associated with a callable bond. Callable bonds typically have shorter durations than non-callable bonds, which means that they are less sensitive to changes in interest rates.

5. Monte Carlo Simulation

Monte Carlo simulation is a technique that uses random variables to simulate the potential outcomes of an investment. It takes into account a range of possible interest rate scenarios and calculates the potential return of a callable bond under each scenario. Monte Carlo simulation is important for investors because it allows them to assess the potential risk and return of a callable bond in a variety of market conditions.

A callable bond is a bond in which the issuer reserves the right to redeem the bond at different discrete times, possibly for different redemption values. When a callable bond is bought, no one knows when the bond will be called, however it can nly be called at the times agreed upon at issue.

The most common questions asked with callable bonds involve either

1. determining the maximum price that can be paid for the bond to guarantee a certain yield rate, or
2. detemining the minimum yield rate for a bond that was bought at a certain price.

The recommende strategy for solving callable bond problems is to crete a two column table with the first represeint ghe timme at which the bond can be called.

For type 1. problems, the second column is the corresponding price in order to receive the desired yield rate, keeping in mind the redemption value may be different at the different redemption dates. Then think through what happens whine the bond is bought for the prices in the table but redeemed at other times in the table, an answer the question accrodgingly.

For type 2. rpboems the second column of the table is the yield rate tha would produce the given price, again keeping in mind the redemption value may be different at the different edemption dates.


\question A 1000 face value 20-years callable bond with 5\% annual coupons is selling 1150. The bond can be redeemed at the end of 18 years for 950, at the end of 19 years for 975, or at the end of 20 years for 1000. Determine the minimum annual yield rate that a buyer will earn on this bond.
\fillwithlines{3cm}
\begin{solution}
3.647\%

18 3.831\%
19 3.87\%
20 3.91\%


\end{solution}

\question A 1000 face value 20-year callable bond with 3\% annual coupons can be redeemed according to the following schedule:
\begin{itemize}
	\item 1000 at the end of years 10 through 14;
	\item 1075 at the end of years 15 through 17;
	\item 1125 at the end of years 18 through 20.
\end{itemize}
Determine the maximum price a buyer should pay in order to earn an annual yield of at least 5\%.
\fillwithlines{3cm}
\begin{solution}
def ytm(y, c, t, f, red):
    ...:     val = sum([c*f/(1+y)**k for k in range(1, t+1)])
    ...:     return val + red/(1+y)**t
    ...: 

In [28]: ytm(0.05, 0.03, 10, 1000)

10 845.57
11 833.87
NOTE: The prices will systematically increase or decrease (decrease in this problem) when the redemption value remains the same.
Therefore, we only need to check endpoints for the same redemption values.
14 802.03
15 828.48
17 807.24
18 818.15
20 797.87 which is the answer

Imagine you bought the bond for 797.87. Determine the yield if the bond is redeemed at the end of year 16.

In [51]: def find_ytm(y, c, t, f, red, price):
    ...:     val = sum([c*f/(1+y)**k for k in range(1, t+1)])
    ...:     return val + red/(1+y)**t - price

n [55]: brentq(find_ytm, 0, 0.1, args=(0.03, 16, 1000, 1075, 797.87))
Out[55]: 0.05204043575983794

\end{solution}

\question A 1000 face value 10-year callable bond with 8\% semiannual coupons is bought for 1050.  The bond can be redeemed at the end of any year starting with year 7.  Determine the minimum annual yield for this bond.
\fillwithlines{3cm}
\begin{solution}
In [70]: def b_ytm(price, fv, call_price, T, coup, freq=2, guess=0):
    ...:     freq = float(freq)
    ...:     periods = T*freq
    ...:     coupon = coup*fv/freq
    ...:     dt = [(i+1)/freq for i in range(int(periods))]
    ...:     ytm_func = lambda y: sum([coupon/(1+y/freq)**(freq*t) for t in dt]) + fv/(1+y/freq)**(freq*max(dt)) - price
    ...:     return newton(ytm_func, guess)

There is no redemption value stated or implied. For all bond problems, when this is the case, assume the bond is redeemable at par. 
14 3.54\%
16 3.58\%
18 3.62\%
20 3.64\%
NOTE: The interest rates will systematically increase or decrease (increase in this problem) when the redemption value remains the same. Therefore, we only need to check endpoints for the same redemption values.
\end{solution}

\question A 1000 face value 20-years callable bong, reddeemable at 1200, with 5\% annual coupons can be redeemed at the end of year 18, 19, 20. Determine the maximum price a buyer is willing to pay in order to earn an annual yield of at least 3\%
1392.55
In [116]: price(1000, 1200, 19, 0.05, 0.03, 1)
Out[116]: 1286.475982125383

In [117]: price(1000, 1200, 18, 0.05, 0.03, 1)
Out[117]: 1275.0702615891446

In [118]: price(1000, 1200, 20, 0.05, 0.03, 1)
Out[118]: 1297.5494972091096


\question A 1000 face value 20-years callable bond with 4\% annual coupons can be redeemed according to the following schedule:
1000 at end 12-14, 1025 at end 15-17, 975 at end 18-20.
Determine the maximum price a buyer is willing to pay in order to earn an annual yield of at least 4\%
987.66

\question A 1000 face value 10\% annual coupon bond is redeemable as follows: 1100 at 15, 16,17, 1000 at 18,19,20. A buyer pays 1500 for this bond, Determine the buyer minimum annual yield.
In [70]: def b_ytm(price, fv, call_price, T, coup, freq=2, guess=0):
    ...:     freq = float(freq)
    ...:     periods = T*freq
    ...:     coupon = coup*fv/freq
    ...:     dt = [(i+1)/freq for i in range(int(periods))]
    ...:     ytm_func = lambda y: sum([coupon/(1+y/freq)**(freq*t) for t in dt]) + fv/(1+y/freq)**(freq*max(dt)) - price
    ...:     return newton(ytm_func, guess)
5.474%
In [96]: b_ytm(1500, 1000, 1100, 15, 0.1, 1)
Out[96]: 0.051378853239277274

In [97]: b_ytm(1500, 1000, 1100, 16, 0.1, 1)
Out[97]: 0.05290592139352415

In [98]: b_ytm(1500, 1000, 1100, 17, 0.1, 1)
Out[98]: 0.05423627236846406

In [99]: b_ytm(1500, 1000, 1000, 18, 0.1, 1)
Out[99]: 0.05540277776325789

In [100]: b_ytm(1500, 1000, 1000, 19, 0.1, 1)
Out[100]: 0.05643146238150476

In [101]: b_ytm(1500, 1000, 1000, 20, 0.1, 1)
Out[101]: 0.05734319789419295




\question A 20-year 10\% 1000 bond that pays interest half-yearly is redeemable (callable) in twelve years at a buy-back (call) price of 1150. The bond's current yield to maturity is 9.50\% annually. You are required to determine 
\begin{enumerate}[label=(\alph*),font=\itshape]
	\item the yield to call;
	\item the yield to call if the buy-back price is only 1100;
	\item the yield to call if instead of twelve years the bond can be called in eight years, the buy-back price being 1150.
\end{enumerate}
\fillwithlines{3cm}
\begin{solution}

def price(y, c, t, f, red):
    ...:     val = sum([c*f/(1+y)**k for k in range(1, t+1)])
    ...:     return val + red/(1+y)**t

def price(fv, call_price, T, coup, y, freq=2):
     freq = float(freq)
     periods = T*freq
     coupon = coup*fv/freq
     dt = [(i+1)/freq for i in range(int(periods))]
	return sum([coupon/(1+y/freq)**(freq*t) for t in dt]) + fv/(1+y/freq)**(freq*max(dt)) 

A callable bond is a debenture issued with a call provision, in which this bond can be redeemed earlier than its initial maturity. The call price is determined at the issuance.

Given the number of periods $N$, the par 1000, the semiannual coupon = $1000\cdot 0.1 \cdot 0.5 = 50$, and the semiannual yield $I=0.0475$ determine the current price of the bond
\begin{equation*}
P_0 = coupon \frac{1-(1+I)^{-N}}{I} + \frac{Par}{(1+I)^N} = 50 \frac{1-(1+0.0475)^{-40}}{0.0475}+\cfrac{1000}{(1+0.0475)^40} = 1044.41
\end{equation*}
\begin{enumerate}[label=(\alph*),font=\itshape]
\item To determine the yield to call if the price is 1150, 
FV = 1,150
PV = 1,044.41
PMT = 50
N = 12 x 2 = 24
CPT I = 5.01
YTC = 5.01% x 2 = 10.02%
\item Determine the yield to call if the call price is Rs 1,100; using the financial calculator:

FV = 1,100
PV = 1,044.41
PMT = 50
N = 12 x 2 = 24
CPT I = 4.91
YTC = 4.91% x 2 = 9.82%
\item Determine the yield to call if the call price is Rs 1,150; and this bond will be called in 8 yearsl of ; using the financial calculator:

FV = 1,150
PV = 1,044.41
PMT = 50
N = 8 x 2 = 16
CPT I = 5.21
YTC = 5.21% x 2 = 10.42%
\end{solution}



\end{questions}
\end{document}
