\documentclass{article}
\usepackage{amsmath}
\usepackage{amssymb}
\usepackage{geometry}
\geometry{a4paper, margin=1in}
\usepackage{enumerate}
\usepackage{hyperref}

\title{Advanced Financial Modeling - Exam}
\author{M. Sani, A. Carapelli}
\date{07/06/2024}

\begin{document}

\maketitle

\section{Exercises}
\subsection{Extended Short Rate Models}
Define the HJM dynamics of $f(t,T)$ under the risk neutral measure $\mathbb{Q}$
\begin{equation*}
    df(t,T) = \left( \sigma_f(t,T) \int_t^T  \sigma_f(t,u) du \right) dt + \sigma_f(t,T)dW_t
\end{equation*}
Where $\sigma_f$ is the instantaneous forward volatility and $W$ is a Brownian motion. Recall the change of numeraire equality:
\begin{equation*}
    \frac{d \mathbb{Q}^T}{d \mathbb{Q}} = \frac{P(t,T)}{P(0,T)} \frac{B(0)}{B(t)}
\end{equation*}
\begin{itemize}
    \item[a)] Show that  $f(t,T)$ is a martingale under the $ \mathbb{Q}^T$ forward measure, $dB_t$ is the process of the money market account (deterministic solution) and $dP_t$ is a log-Normal process for the zero coupon bond dynamics.
    \item[b)]  What is the difference between forward measure $ \mathbb{Q}^T$ and terminal forward measure $ \mathbb{Q}^{T_f}$?
    \item[c)] What is the advantage of "extending" short rate models? 
\end{itemize}

\subsection{xVA}
\begin{enumerate}[a)]
    \item Define briefly Counterparty Credit Risk and xVA. Show that, at valuation time $t_0$, with $\tau > t_0$ (no default happened yet), the price of a derivative under counterparty risk is:
    $$ V_D(t) = V(t) - \mathbb{E}^Q \left[ 1_{\tau\leq T} \cdot \text{LGD}(\tau) \cdot D(t,\tau) \cdot \text{V}(\tau) \right] $$
    % \item Suppose you want to estimate future exposure for a linear interest rate derivative (assume that the only risk factor is interest rate) using the Hull and White model.
    % The integrated process reads:
    % \begin{equation*}
    %     r(t) = r(s)e^{-\kappa(t-s)} + \alpha (t) - \alpha(s) e^{-\kappa(t-s)} + \sigma \int_{s}^{t} e^{-\kappa(t-u)}dW(u),
    % \end{equation*}
    % where $\kappa$ and $\sigma$ are positive constants and:
    % \begin{equation*}
    %     \alpha (t) = f^M (0,t) + \frac{\sigma ^2}{2\kappa^2}(1- e^{-\kappa t})^2,
    % \end{equation*}
    % Compute conditional expectation under the risk neutral measure $\mathbb{E}^\mathbb{Q}[r(t) \vert r(s)]$ and  variance of the process ($r(t)$ conditional on $r(s)$ with $t>s$). Which is the statistical distribution of the process? 
    \item Explain the workflow for CVA and DVA calculation using Monte Carlo simulation. 
    
    
\end{enumerate}



\newpage
\section{Solutions}

\subsection{Forward HJM Process}
\begin{itemize}
    \item[a)] The goal is to define another Brownian under the new probability and establish a connection with the previous. Recall the change of numeraire
\begin{equation*}
    \frac{d \mathbb{Q}^T}{d \mathbb{Q}} = \frac{P(t,T)}{P(0,T)} \frac{B(0)}{B(t)}
\end{equation*}
We know the dynamics of $B(t)$, then apply Ito to $ln P(t,T)$ to get the needed dynamics
\begin{align*}
    \frac{P(t,T)}{P(0,T)} &= e^{\int_0^t (r_u - \frac{1}{2}\sigma_P^2(u,T))du + \int_0^t \sigma_P(u,T)dW_u} \\
    \frac{B(0)}{B(t)} &= e^{-\int_0^t r_u du}
\end{align*}
now substitute to get 
\begin{equation}\label{eq:RN_derivative_HJM}
     \frac{d \mathbb{Q}^T}{d \mathbb{Q}} = e^{- \frac{1}{2} \int_0^t \sigma_P^2(u,T)du + \int_0^t \sigma_P(u,T)dW_u}
\end{equation}
Recall that by the \textbf{Girsanov} theorem, if we have a Brownian motion $W_t$ under $\mathbb{Q}$, and we introduce a new process $y(t) = \int_0^t y_u du$ then $W^T = W_t - \int_0^t y_u du$ is a Brownian motion under $\mathbb{Q}^T$ defined via the R-N derivative
\begin{equation}\label{eq:RN_derivative}
    \frac{d \mathbb{Q}^T}{d \mathbb{Q}} = e^{-\frac{1}{2} \int_0^t y_u^2 du + \int_0^t y_u dW_u}
\end{equation}
in differential form
\begin{equation*}
    dW_t^T = dW_t - y_t dt
\end{equation*}
If we compare the RN derivative [\ref{eq:RN_derivative}] with the equation [\ref{eq:RN_derivative_HJM}] we see that we established a connection as  $y_u = \sigma_P(u,T)$ and we can write the new Brownian as
\begin{equation*}
    dW_t^T = dW_t - \sigma_P(t,T)dt = dW_t + \int_t^T \sigma_f(t,u)du dt
\end{equation*}
now we can replace the Brownian in the Risk Neutral HJM and get the dynamics under the new Forward probability measure
\begin{equation*}
        df(t,T) = \left( \sigma_f(t,T) \int_t^T  \sigma_f(t,u) du \right) dt + \sigma_f(t,T) \left( dW_t^T - \int_t^T \sigma_f(t,u)du dt \right)
\end{equation*}
the drift term simplifies as $f(t,T)$ is a martingale under the $ \mathbb{Q}^T$ forward measure
\begin{equation*}\label{eq:FWD_HJM}
        df(t,T) =  \sigma_f(t,T) dW_t^T 
\end{equation*}

    \item[b)]  Typically, you can use the \textbf{longest maturity $P(t,T_f)$} where $T_f > T$ where $T_f$ is the numeraire of the \textbf{Terminal forward measure}. From the Girsanov theorem we get
\begin{equation*}
    dW_t^{T_f} = dW_t - \sigma_P^2(t,T_f)dt = dW_t + \int_t^{T_f} \sigma_f(t,u)du dt
\end{equation*}
if we substitute in the Risk Neutral equation [\ref{eq:RN_HJM}] as we did before we get
\begin{equation*}\label{eq:TERM_HJM}
    df(t,T) = - \left( \sigma_f(t,T) \int_T^{T_f}  \sigma_f(t,u) du \right) dt + \sigma_f(t,T)dW^T_t
\end{equation*}
we can see now the difference between probability measures:
\begin{itemize}
    \item $\mathbb{Q} \to $ the drift integral will be defined with the remaining maturity;
    \item $\mathbb{Q}^T \to $ drift will be zero;
    \item $\mathbb{Q}^{T_f} \to $ the drift integral will be defined between $T$ (maturity of the modelled forward) and $T_f$ (maturity of the terminal numeraire)
\end{itemize}
No matter which is the probability measure, volatility drives the dynamics of $f(t,T)$.
 \item[c)] Extended models provide perfect fit to the term structure in $t_0$ (curve is an input of the model) and better fit to volatility structures with respect to standard affine models.
\end{itemize}
%%%%%%%%%%%%%%%

\subsection{xVA Simulation}
\begin{itemize}
    \item[a)] If the default of a counterparty happens after the final payment of derivative $T$, the value at time $t$ is simply $$1_{\tau > T}V(t,T)$$.
 If the default occurs before the maturity time $\tau < T$:
\begin{enumerate}
    \item We receive/pay all the payments until the default time: $1_{\tau \leq T}V(t, \tau)$;
    \item Depending on the counterparty, we may be able to recover some of the future payments, assuming the recovery fraction to be $R$ the value yields: $1_{\tau \leq T}R \max(V(\tau;T), 0)$;
    \item On the other hand, if we owe the money to the counterparty that has defaulted we cannot keep the money but we need to pay it completely back: $1_{\tau \leq T} \min(V(\tau;T), 0)$.
\end{enumerate}
Thus, when including all the components, a price of a \textit{risky} derivative is given by:
\begin{align*}
V_D(t_0, T) = \mathbb{E}^Q& \big[ 1_{\tau > T}V(t_0,T) +
1_{\tau \leq T}V(t_0, \tau) \\
&+ D(t_0, \tau) \, 1_{\tau \leq T}R \max(V(\tau,T), 0) \\
&+ D(t_0, \tau) \, 1_{\tau \leq T} \min(V(\tau, T), 0) \,|\, \mathcal{F}_t \big]
\end{align*}
Since \( x = \max(x, 0) + \min(x, 0) \), the simplified equation reads:
\begin{align*}
V_D(t_0, T) = \mathbb{E}^Q& \big[ 1_{\tau > T}V(t_0,T) +
1_{\tau \leq T}V(t_0, \tau) \\
&+ D(t_0, \tau) \, 1_{\tau \leq T} V(\tau;T)\\
&+ D(t_0, \tau) \, 1_{\tau \leq T}(R-1) \max(V(\tau;T), 0) \,|\, \mathcal{F}_t \big]
\end{align*}

We immediately note that the first three terms in the expression above yield:
\begin{align*}
    &\mathbb{E}^Q \big[ 1_{\tau > T}V(t_0,T) +
1_{\tau \leq T}V(t_0, \tau) + D(t_0, \tau)1_{\tau \leq T} V(\tau,T) \big] \\
    &= \mathbb{E}^Q \big[ 1_{\tau > T}V(t_0,T) +
1_{\tau \leq T}V(t_0, T) \big] \\
    &= V(t_0).
\end{align*}
The value of the risky derivative $V_D(t)$ is:
\begin{align*}
    V_D(t_0) &=  V(t_0) + \mathbb{E}^Q \left[1_{ \tau \leq T} \, (\text{R}(\tau) -1) D(t, \tau) \,  V(\tau)^+ \,|\, \mathcal{F}_t \right] \\
    &= V(t_0) - \mathbb{E}^Q \left[1_{ \tau \leq T} \, \text{LGD}(\tau) D(t, \tau) \,  V(\tau)^+ \,|\, \mathcal{F}_t \right]  \\
    &= V(t_0) - \text{uCVA}(t_0).
\end{align*}

\item[b)] \begin{itemize}
    \item part1, definition of the contract and risk factor involved for pricing
    \item part2, simulation of the risk factors necessary for p1, as we need future paths. Need to define model, time grid discretization.
    \item part3, price the derivative across all paths and timesteps
    \item  part4, compute exposures profiles.
    \item part5, bootstrap credit and funding curve.
    \item part6, aggregate exposures and compute xVA.
\end{itemize}

%     \item[b)] Starting from the uCVA equation:
%         \begin{align*}
%             \text{uCVA}(t) = 1_{\tau > t} \lim_{n \to \infty} \sum_{i=1}^n \mathbb{E}^Q \left[  \left( e^{-\int_{t}^{t_{i-1}} \lambda(s)  ds} - e^{-\int_{t}^{t_i} \lambda(s) ds} \right)  \text{LGD}(t_{i-1}) \cdot D(t, t_{i-1}) \cdot \text{PV}^+(t_{i-1}) \right]. 
%     \end{align*}
%     Assuming : 
%     \begin{itemize}
%         \item finite number of timesteps N,
%         \item constant loss given default,
%         \item independence between default rates and interest rates,
%         \item  deterministic hazard rates in $t_0$.
%     \end{itemize} 
%     \begin{align*}
%         \text{uCVA}_{sw}(t_0) &=  \text{LGD} \sum_{i=1}^N   
%         \left( e^{-\int_{t}^{t_{i-1}} \lambda(s)  ds} - e^{-\int_{t}^{t_i} \lambda(s) ds} \right)
%         \mathbb{E}^Q \left[   D(t, t_{i-1})  \text{PV}^+(t_{i-1}) \right] 
%     \end{align*}
%     with $ \mathbb{E}^Q \left[   D(t, t_{i-1})  \text{PV}^+(t_{i-1}) \right] $ being the swaption part.
% \end{itemize}

\end{document}
